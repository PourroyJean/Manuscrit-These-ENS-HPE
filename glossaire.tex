\newglossaryentry{domain-knowledge}{%
  name={domain knowledge},%
  description={valid knowledge used to refer to an area of human endeavour, an autonomous computer activity, or other specialized discipline}}

\newacronym{tla}{TLA}{Three Letter Acronym}




\newglossaryentry{latex}
{
    name=latex,
    description={Is a mark up language specially suited for scientific documents}
}

\newglossaryentry{maths}
{
    name=mathematics,
    description={Mathematics is what mathematicians do}
}

\newglossaryentry{formula}
{
    name=formula,
    description={A mathematical expression}
}



\newglossaryentry{duck}{name=DUUuck,
  description={a waterbird with webbed feet}}

\newglossaryentry{parrot}{name=Parrot,
  description={mainly tropical bird with bright plumage}}
  
  %%% The glossary entry the acronym links to   
\newglossaryentry{apig}{name={API},
    description={An Application Programming Interface (API) is a particular set
of rules and specifications that a software program can follow to access and
make use of the services and resources provided by another particular software
program that implements that API}}

%%% define the acronym and use the see= option
\newglossaryentry{api}{type=\acronymtype, name={API}, description={Application
Programming Interface}, first={Application
Programming Interface (API)\glsadd{apig}}}



  %%% The glossary entry the acronym links to   
\newglossaryentry{hpcg}{name={HPC},
    description={Le but du HPC est de paralléliser des applications scientifiques à destination de ressources informatiques telles que les supercalculateurs}}

%%% define the acronym and use the see= option
\newglossaryentry{hpc}{type=\acronymtype, name={HPC}, description={High Performance Computing ou Calcul Haute Performance}, first={Calcul Haute Performance (HPC)\glsadd{hpcg}}}




%\newglossaryentry{hpc}{%
%  name={Calcul Haute Performance},%
%  description={Le but du HPC est de paralléliser des applications scientifiques à destination de ressources informatiques telles que %les supercalculateurs}
%}
