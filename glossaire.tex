\newglossaryentry{domain-knowledge}{%
  name={domain knowledge},%
  description={valid knowledge used to refer to an area of human endeavour, an autonomous computer activity, or other specialized discipline}}

\newacronym{tla}{TLA}{Three Letter Acronym}




\newglossaryentry{latex}
{
    name=latex,
    description={Is a mark up language specially suited for scientific documents}
}

\newglossaryentry{maths}
{
    name=mathematics,
    description={Mathematics is what mathematicians do}
}

\newglossaryentry{formula}
{
    name=formula,
    description={A mathematical expression}
}



\newglossaryentry{duck}{name=DUUuck,
  description={a waterbird with webbed feet}}

\newglossaryentry{parrot}{name=Parrot,
  description={mainly tropical bird with bright plumage}}
  
  %%% The glossary entry the acronym links to   
\newglossaryentry{apig}{name={API},
    description={An Application Programming Interface (API) is a particular set
of rules and specifications that a software program can follow to access and
make use of the services and resources provided by another particular software
program that implements that API}}

%%% define the acronym and use the see= option
\newglossaryentry{api}{type=\acronymtype, name={API}, description={Application
Programming Interface}, first={Application
Programming Interface (API)\glsadd{apig}}}


%%
%%
%%%%%%%%%%%%%%%%%%%%%% DEFINITION %%%%%%%%%%%
%%
%%



    
\newglossaryentry{exascale}{
    name={exascale},
    description={désigne la nouvelle génération de plateforme capable d'exécuter un exaflops ($10^{18}$ opérations en virgule flottante par seconde)}
}
    

\newglossaryentry{thread}{
    name={thread},
    description={ou processus léger ou tâche est similaire à un processus. Les threads d'un même processus se partagent le même espace mémoire.},
    plural={threads}
}
    

\newglossaryentry{hotspot}{
    name={hot spot},
    description={ Désigne une région d'un programme où une grande proportion d'instructions sont exécutées pendant l'exécution d'une application.},
    plural={hot spots}
}


\newglossaryentry{benchmark}{
    name={benchmark},
    description={ Code, ou un ensemble de codes, permettant de mesurer la performance d'une solution et d'en vérifier ses fonctionnalités.},
    plural={benchmarks},
}


\newglossaryentry{framework}{
    name={framework},
    description={ infrastructure logicielle désignant un ensemble de composants logiciels établissant les fondations d'un logiciel.},
    plural={frameworks},
}






%%
%%
%%%%%%%%% ACCRONYME %%%%%%%%%%
%%
%%
% --> liens avec \glsadd

%%% define the acronym and use the see= option


%% GLOBAL %%
    
    \newglossaryentry{hpc}{type=\acronymtype, name={HPC}, description={High Performance Computing ou Calcul Haute Performance}, first={Calcul Haute Performance (HPC)\glsadd{hpcg}}}
    
    \newglossaryentry{hpcg}{name={HPC},
    description={Le but du HPC est de paralléliser des applications scientifiques à destination de ressources informatiques telles que les supercalculateurs}}

    \newglossaryentry{FLOP}{type=\acronymtype, name={FLOP}, description={Floating Point Operation ou opérations en virgule flottante}, first={opération en virgule flottante (FLOP))}}
    
    \newglossaryentry{FLOPS}{type=\acronymtype, name={FLOPS}, description={Floating Point Operation per second ou nombre d'opérations en virgule flottante par seconde (FLOP/s)}, first={opérations en virgule flottante par seconde (FLOPS)}}
    
    \newglossaryentry{exaFLOPS}{type=\acronymtype, name={exaFLOPS}, description={$10^{18}$ FLOPS}, first={$10^{18}$ FLOPS (un exaFLOPS)}}
    
    

%% ARCHITECTURE %%
    
    \newglossaryentry{gpu}{type=\acronymtype, name={GPU}, description={Graphics Processing Unit ou processeur graphique}, first={processeur graphique (GPU)}}
    
    \newglossaryentry{fpga}{type=\acronymtype, name={FPGA}, description={Field Programmable Gate Arrays ou réseaux logiques programmables}, first={réseaux logiques programmables (FPGA)}}
    
    \newglossaryentry{dsp}{type=\acronymtype, name={DSP}, description={Digital Signal Processor ou processeur de signal numérique}, first={processeur de signal numérique (DSP)}}
    
    \newglossaryentry{asic}{type=\acronymtype, name={ASIC}, description={application-specific integrated circuit ou circuit intégré propre à une application}, first={circuit intégré propre à une application (ASIC))}}

%% LOGICIEL %%

    %\newglossaryentry{mpig}{name={MPI},description={standard de communication pour des programmes parallèles sur des systèmes à mémoire distribuée}}
    
    %\newglossaryentry{mpi}{type=\acronymtype, name={MPI}, description={Message Passing Interface}, first={interface de passage de message (MPI))\glsadd(mpig)}}
  

    
    \newglossaryentry{mpi}{type=\acronymtype, name={MPI}, description={Message Passing Interface ou interface de passage de message}, first={interface de passage de message (MPI)\glsadd{mpig}}}
    
    \newglossaryentry{mpig}{name={MPI},
    description={Standard de communication pour des programmes parallèles sur des systèmes à mémoire distribuée}}
    
  


%\newglossaryentry{hpc}{%
%  name={Calcul Haute Performance},%
%  description={Le but du HPC est de paralléliser des applications scientifiques à destination de ressources informatiques telles que %les supercalculateurs}
%}
