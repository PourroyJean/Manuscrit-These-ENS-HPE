\iffalse 
%%%%%%%%%%%%%%%%%%%%%%%%%%%%%%%%%%%%%%%%%%%%%%%%%%%%%%%%%%%%%%%%%%%%%%%%%%%%%%%%%%%%%%%%
%%%%%%%%%%%%%%%%%%%%%%%%%%%%%%%%%%%%%%%%%%%%%%%%%%%%%%%%%%%%%%%%%%%%%%%%%%%%%%%%%%%%%%%%
%%_______ .___________.    ___      .______    _______                       __       %%
%%|   ____||           |   /   \     |   _  \  |   ____|                    /_ |      %%
%%|  |__   `---|  |----`  /  ^  \    |  |_)  | |  |__          ______        | |      %%
%%|   __|      |  |      /  /_\  \   |   ___/  |   __|        |______|       | |      %%
%%|  |____     |  |     /  _____  \  |  |      |  |____                      | |      %%
%%|_______|    |__|    /__/     \__\ | _|      |_______|                     |_|      %%
%%%%%%%%%%%%%%%%%%%%%%%%%%%%%%%%%%%%%%%%%%%%%%%%%%%%%%%%%%%%%%%%%%%%%%%%%%%%%%%%%%%%%%%%
%%%%%%%%%%%%%%%%%%%%%%%%%%%%%%%%%%%%%%%%%%%%%%%%%%%%%%%%%%%%%%%%%%%%%%%%%%%%%%%%%%%%%%%%
\fi


%%%%%%%%%%%%%%%%%%%%%%%%%%%%%%%%%%%%%%%%%%%%%%%%%%%%%%%%%%%%%%%%%%%
\section{Étape 1: Veille technologique}
%%%%%%%%%%%%%%%%%%%%%%%%%%%%%%%%%%%%%%%%%%%%%%%%%%%%%%%%%%%%%%%%%%%

%%%%%%%%%%%%%%%%%
\subsection{Motivation et objectifs}
%%%%%%%%%%%%%%%%%

Grâce à Gen-Z, de nouvelles technologies utilisable par les super-calculateurs vont apparaître régulièrement. Comme expliqué précédemment, il est crucial pour les utilisateurs de porter leur code et d'investir du temps et de l'argent dans les bonnes plate-formes. Aujourd'hui ce choix se limité à quelques architectures, principalement les CPU (Intel, AMD, IBM) et les accélérateurs (NVIDIA, AMD, Xeon Phi). Demain, prendre la bonne entre toutes les nouvelles technologies sera beaucoup plus difficile.

Même si elle n'est pas une étape de la méthodologie en soit, le premier travail de développeur est d'être en constante recherche des dernières innovations technologiques. Cela peut être de nouveau processeurs, de nouvelles mémoire ou bien de nouveaux algorithmes ou optimisations. Il est très important de se tenir à l'état de l'art ou même en avance pour anticiper les nouveautés. 

Le but principal de cette étape est de répertorier toutes les plate-formes et technologies potentiellement intéressantes pour le calcul haute performance. Certaines caractéristiques clés sont calculées à partir des spécificités techniques des architectures.


%%%%%%%%%%%%%%%%%
\subsection{Processeurs et accélérateurs}
%%%%%%%%%%%%%%%%%

Dans notre vision, la majorité des lignes de codes continueront d'être exécutées sur des architectures semblables à celles d'aujourd'hui (x86 et PowerPC). Seulement les kernels de calculs seront déportés sur les accélérateurs adéquats. Il est donc nécessaire de continuer à s'y interesser et à les caractériser. Les processeurs présent dans les architectures de demain pourraient alors être bien différent de ce d'aujourd'hui car si les kernels des applications n'y sont plus exécutés, les caractéristiques recherchés seront différentes. 

Les accélérateurs actuels continueront d'avoir leur rôle à jouer. Les GPU se sont montrés extrêmement efficace pour les algorithmes d'apprentissage par machine et d'intelligence artificielle.



%%%%%%%%%%%%%%%%%
\subsection{Mémoires}
%%%%%%%%%%%%%%%%%

Grâce à Gen-Z, la totalité de l'architecture sera \textit{composable}, pas seulement au niveau des processeurs mais aussi au niveau des mémoires. Grâce à sa sémantique d'accès \textit{load/store}, Gen-Z va permettre au processeur d'accéder à toute la mémoire visible dans le super-calculateur. En fonction des jeux de données, la quantité de mémoire doit être calculée pour de pas en manquer et risquer d'effondrer les performances, ou de surestimer le besoin et perdre en rendement économique. 



%%%%%%%%%%%%%%%%%
\subsection{Caractéristiques clefs}
%%%%%%%%%%%%%%%%%

Pour la suite de l'analyse, il est nécessaire de récupérer des caractéristiques clefs pour chaque architectures. Cette phase étant indépendante de l'application, il ne faut pas négliger certaines architectures qui pourraient s'avérer intéressante ensuite. La majorité des applications aient besoin d'un bus mémoire très performant. Seulement, certaines partie du code, séquentielles ou utilisant seulement les unités arithmétiques et logiques, auront d'autres besoins qui pourra être interessant de porter sur des architectures différentes.

Nous avons isolé X caractéristiques qu'il est nécessaire d'avoir pour la suite de l'analyse:
\begin{itemize}
    \item La bande passante économique dont qui mesure le nombre de gigabyte de données transférable par seconde pour le prix de la plate-forme ($GB/seconde/dollar$). Deux facteurs très important dans le choix de la plate-forme entre ici en jeu: la bande passante disponible, très importante pour la majorité des codes, ainsi que l'économie qui est souvent le facteur de décision ultime. On cherchera les plate-forme avec la plus grande bande passante économique.
    \item L'équilibre arithmétique de la plate-forme est mesurée nombre d'opération réalisable pour chaque gigabyte de données transféré  depuis la mémoire en une seconde mesuré en  $flops/GB/s)$. Cette valeur permet d'estimer l'équilibre entre le calcul et le débit mémoire d'une plate-forme. Une grande valeur signifiera que la plate-forme est plutôt destinée à des codes intensifs en calcul. A l'inverse, une valeur faible signifiera que la plate-forme est adapté à des codes nécessitant beaucoup d'accès mémoire. Pour la majorité des applications on cherchera à obtenir une valeur petite.
    \item L'efficience énergétique mesure le rapport d'opérations flottante par watt d'énergie consommée ($flops/watt$). Comme exposé dans la partie \ref{X}, la consommation électrique du super-calculateur est une contrainte majeur pour le projet Exascale. Il est donc important de privilégier des architectures avec les meilleurs rendements énergétiques. On cherche ici à obtenir la plus grande valeur possible.
\end{itemize}




%%%%%%%%%%%%%%%%%
\subsection{Calculs des caractéristiques}
%%%%%%%%%%%%%%%%%

Le calculs des caractéristiques par les données techniques des architectures a l'avantage de permettre d'évaluer rapidement leur potentiel sans y avoir accès. Cette partie présente comment calculer certaines caractéristiques comme la bande passante ou la puissance crête d'un processeur.

%%%%%%%%%%%%%%%%%
\subsubsection{La bande passante}
Le calcul de la bande passante mémoire, nécessite de connaître la fréquence de la mémoire mesurée en $MHz$. La fréquence de la mémoire RAM correspond au nombre de cycle d'horloge par seconde. Il existe différentes technologies mémoire permettant d'écrire entre une, deux ou quatre fois par cycle sur chaque ligne du bus. On parle alors de mémoire Single Data Rate (SDR), Double Data Rate (DDR) et Quad Data Rate (DQR). La fréquence seule ne permet donc pas d'indiquer combien de transferts peuvent être réalisés par seconde, il faut aussi connaître le débit de données. Pour éviter les confusions, on parle aussi de Mega Transferts par seconde ($MT/s$) que l'on notre $MTS$. Ces deux unités sont souvent mélangés, par les constructeurs eux même, peut être à des fins marketing. Par exemple la DDR4-2666, signifie que la RAM a une fréquence de 1333 MHz. Il faut ensuite connaître le nombre de ligne reliant la mémoire au processeur ou au GPU que l'on note $bus\_width$ mesuré en byte. Les architectures x86 récentes utilisent des bus de 64 bits. Pour augmenter la bande passante mémoire disponible, les architectures utilisent plusieurs canaux mémoire noté $nb\_channels$.
La bande passante maximum théorique, $MEMORY_{peak}$, peut alors être calculée avec la formule suivante:

\begin{equation}
\label{eq:bw}
    MEMORY_{peak} = MTS \times bus\_width \times nb\_channels
\end{equation}


%%%%%%%%%%%%%%%%%
\subsubsection{La puissance de calculs}
La deuxième valeur qui nous intéresse dans notre analyse est la performance crête de calcul mesurée en $GFlops$. Pour la calculer, nous adaptons la notation proposée dans \cite{dolbeau2015theoretical}.
La première information nécessaire est le nombre maximale de \textit{flop} exécutable par cycle, notée $FLOP_{cycle}$ mesuré en $\frac{flop}{cycle}$. 
Les processeurs sont capables de réaliser des opérations sur plusieurs données à la fois. La taille de ces instructions (SIMD) et leur disponibilité dépend de l'architecture. Elle est mesurée en $\frac{flop}{operation}$.
Pour atteindre la performance crête, il faut utiliser les instructions permettant de faire le maximum d'opération en une exécution. Sur les processeurs modernes ce sont les instructions Fused Multiply Add (FMA) qui sont capables d'exécuter une multiplication et une addition en un cycle mesuré en $\frac{operations}{instruction}$. Enfin, les processeurs superscalaire sont capables d'exécuter plusieurs instructions en un seul cycle, mesurée en $\frac{instructions}{cycle}$. $FLOP_{cycle}$ peut ainsi être calculé grâce à la formule suivante:

\begin{equation}
\label{eq:floc}
    FLOP_{cycle} = \frac{flop}{operation} \times \frac{operations}{instruction} \times \frac{instructions}{cycle}
\end{equation}

Une fois la performance maximale de la micro-architecture calculée, il faut calculer la performance maximale atteignable par le processeur, notée $FLOP_{seconde\_peak}$ mesurée en $\frac{flop}{seconde} $. Pour cela, il faut connaître la fréquence atteignable par le processeurs lorsque sont exécutés les instructions SIMD utilisées pour calculer $FLOP_{cycle}$. En effet, pour éviter des problèmes de surchauffe, le processeur doit abaisser sa fréquence lorsqu'il utilise de telles instructions. Cette fréquence est notée $\frac{cycle}{seconde}$. Enfin, il faut avoir le nombre de coeurs disponibles sur le processeurs.
\begin{equation}
\label{eq:flops}
    FLOP_{seconde\_peak} = FLOP_{cycle} \times \frac{cycle}{seconde} \times nombre\ de\ coeurs
\end{equation}





%%%%%%%%%%%%%%%%%
\subsection{Application au processeur Intel Xeon 6148}
%%%%%%%%%%%%%%%%%

Pour illustrer la présentation de la méthodologie, nous utilisons l'exemple d'un processeur Intel Xeon Skylake 6148 possédant 20 coeurs à une fréquence de base de 2.4 GHz. Une configuration à deux processeurs est présenté sur la \autoref{fig:skylake_gold}. \textbf{Mémoire: } le processeur étudié possède 6 canaux mémoire le connectant à 6 barrettes mémoire cadencé à 2666 MT/s. \textbf{ALU:} les processeurs de la gamme Xeon Gold 6 possèdent tous deux unités AVX-512 capables d'exécuter chacune 2 instructions vectorielles de 512 bits (AVX-512), dont les Fused Multiple Add (FMA). \textbf{Fréquence:} pour une même architecture, dans notre exemple Skylake, chaque modèle de processeur a ses propres plages de fréquences utilisables qui peuvent être consultées en ligne \cite{Wikichipa}. La fréquence utilisable dépend essentiellement de la consommation électrique du processeur et de sa température (dépendant de la qualité du système de refroidissement). Ainsi, les fréquences soutenables par le processeurs dépendent du nombre de coeurs utilisés, de la taille des instructions  exécutées (normal, AVX-2 ou AVX-512) et de la disponibilité du Turbo. Pour l'exécution d'instruction SISD (Single Instruction Single Data) avec le turbo actif sur les 20 coeurs, la fréquence maximale atteignable est de 3.1 GHz.  Le benchmark étudié a la possibilité d'utiliser des instructions vectorielles, le processeur Skylake 6148 peut utiliser des fréquence allant de 1.6 GHz à 2.2GHz \cite{Wikichipa}. 

\begin{figure}
    \center
    \includegraphics[width=10cm]{images/skylake_gold.png}
    \caption{\label{skylake_gold} Architecture d'une plate-forme avec deux processeurs Xeon Skylake (source \cite{Aspsys})}
\end{figure}



%%%%%%%%%%%%%%%%%
\subsubsection{Performances mémoire théoriques}
Le processeur Xeon Skylake 6148 possède 6 canaux mémoire pour accéder, dans notre expérimentation, à une mémoire DDR4-2666. En appliquant l'\autoref{eq:bw} nous obtenons une bande passante maximale de : $2666 \times 8 \times 6 = 128\ GB/s$. A cause de la loi de Little \cite{little2008little}, le processeur doit être capable de générer parallèlement suffisamment de chargement (\textit{outstanding load}) pour atteindre cette performance.


%%%%%%%%%%%%%%%%%
\subsubsection{Performances de calculs théoriques}
Le processeur étudié est un processeur superscalaire capable d'éxécuter jusqu'à 4 instructions par cycle dont deux opérations flottantes. Ces opération pouvant être des instruction FMA vectorielles de 512 bits. Il est donc possible de calculer sur chaque ALU, une multiplication et une addtion par cycle sur 8 éléments simultanément. On peut ainsi calculer la performances crête de ce processeur en appliquant l'\autoref{eq:flops}. Suivant la fréquence utilisable par le processeur (dépendant de la température) la performance crête théorique, $FLOPS_{peak}$ est comprise entre $8 \times 2 \times 2 \times 1.6 \times 20 = 1024\ GFLOP/s$ et  $8 \times 2 \times 2 \times 2.2 \times 20 = 1408\ GFLOP/s$. 

Cependant, pour comparer la performance de l'application avec ce résultat, il faut que la nature du code puisse utiliser des instructions FMA vectorisées. Il peut être intéressant de disposer d'une fourchette de performance lorsque la totalité du parallélisme est utilisé ou non. Quand le processeur n'utilisent pas d'instruction AVX-512, le processeur est capable d'atteindre 3.1 GHz lorsque les 20 coeurs sont actifs. En reprenant l'\autoref{eq:flops}, la performance optimale d'une telle application serait: $FLOPS_{SISD} = 1 \times 1 \times 2 \times 3.1 \times 20 = 124 \ GFLOP/s$.

%%%%%%%%%%%%%%%%%
\subsubsection{Équilibre arithmétique}
L'équilibre arithmétique du processeur permet d'évaluer s'il est approprié pour un code nécessitant une grande bande passante ou plutôt de bonnes performances de calculs. En réutilisant les deux caractéristiques précédemment calculées, on peut calculer $\text{EQUILIBRE}_{non\_avx}$ et $\text{EQUILIBRE}_{avx\_512}$ qui bornent la performances inférieure et supérieure de ce processeur. On obtient ainsi $\text{EQUILIBRE}_{non\_avx} = \frac{124}{128} = 0.97\ flop/byte$ et $EQUILIBRE_{avx\_512} = \frac{1408}{128} = 11\ \text{flopbyte}$. 
 

%%%%%%%%%%%%%%%%%
\subsubsection{L'efficacité énergétique}

La valeur  $EQUILIBRE_{avx\_512}$ est la plus pertinente des deux pour réaliser le choix de l'architecture. Les composants permettant de faire du calcul vectoriel sont compris dans le prix d'achat, qu'ils soient utilisés ou non. Si l'application portée est loin du ratio de 11 opérations flottantes pour 1 byte de données transféré, il faudra se tourner vers d'autres architecture, plus efficace. La notion d'efficacité se retrouve dans le ratio $flops/watt$.
\\
\textbf{tode}