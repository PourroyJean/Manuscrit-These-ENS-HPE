%%%%%%%%%%%%%%%%%%%%%%%%%%%%%%%%%%%%%%%%%%%%%%%%%%%%%%%%%%%%%%%%%%%
\section{Étape 1: Rechercher les dernières innovations technologiques}\label{sec:methodo_step1}
%%%%%%%%%%%%%%%%%%%%%%%%%%%%%%%%%%%%%%%%%%%%%%%%%%%%%%%%%%%%%%%%%%%


%%%%%%%%%%%%%%%%%
\subsection{Introduction}
%%%%%%%%%%%%%%%%%
    
    \subsubsection{Motivations}
    %%%%%%%%%%%%%%%%%%%%%%%%%%%
    
        Le domaine du calcul haute performance est très concurrentiel. Les industries ayant recours à ces plateformes doivent être à la pointe de la technologie au risque de se faire dépasser par un concurrent. 
        Comme vu dans la \autoref{sec:methodo_intro_hetero} \textbf{todo ref plus a jour}, la grande majorité des plateformes utilisent les mêmes architectures. Jusqu'à aujourd'hui, peu d'entreprises se sont démarquées par l'utilisation d'un nouvel accélérateur.
      
        Aujourd'hui, les choix d'architectures disponibles se limitent à quelques architectures: les processeurs (Intel, AMD, IBM) et les \gls{GPU} (NVIDIA, AMD, Xeon Phi). Les technologies émergentes présentées dans la \autoref{sec:oppo}, vont permettre le développement de nouvelles architectures. Grâce au protocole \verb=Gen-Z=, il sera plus facile de les utiliser ensemble au sein d'une même plateforme. Il est donc indispensable pour les industries, mais aussi pour les constructeurs tels que \verb=Hewlett Packard Enterprise=, de connaître et d'utiliser les meilleures d'entre elles. Porter son application sur une nouvelle plateforme nécessite d'investir du temps et de l'argent. Cette décision très importante se complexifie avec l'augmentation du nombre d'architectures différentes.
   
   \subsubsection{Objectifs}
   %%%%%%%%%%%%%%%%%%%%%%%%%
        
        Le premier travail des développeurs et des architectes de plateformes \gls{HPC} est d'être en constante recherche des dernières innovations technologiques. Cela peut être l'annonce d'un nouveau processeur, d'une nouvelle technologie mémoire ou bien de nouveaux algorithmes ou de nouvelles optimisations. Il est très important de se tenir à l'état de l'art ou même en avance pour anticiper les nouveautés. 

        Le but principal de cette étape est de répertorier toutes les plateformes et technologies potentiellement intéressantes pour le calcul haute performance. Certaines caractéristiques clés sont calculées à partir des spécificités techniques des architectures, d'autres sont mesurées dans l'étape suivante lorsque l'accès aux plateformes est disponible.
    

        %\paragraph{Processeurs et accélérateurs.} 
        Dans notre vision, la majorité des lignes de codes continueront d'être exécutées sur des architectures semblables à celles d'aujourd'hui (x86 et PowerPC). Seuls les \glspl{kernel} seront déportés sur les accélérateurs adéquats. Il est donc nécessaire de continuer à s'y intéresser et à les caractériser. Les processeurs présents dans ces nouvelles générations de plateformes pourraient alors être bien différents de ceux utilisés actuellement. En effet, si les kernels des applications ne sont plus exécutés sur ces processeurs, les caractéristiques recherchées seront différentes. Les accélérateurs actuels continueront d'avoir leur rôle à jouer. Les GPU se montrent extrêmement efficaces pour les algorithmes d'apprentissage par machine et d'intelligence artificielle. L'objectif de l'industrie est de trouver des accélérateurs aussi efficaces pour l'exécution d'autres types d'algorithmes.

        %\paragraph{Mémoires.} Grâce à Gen-Z, la totalité de l'architecture sera \textit{composable}, pas seulement au niveau des processeurs, mais aussi au niveau des mémoires. Grâce à sa sémantique d'accès \textit{load/store}, Gen-Z va permettre au processeur d'accéder à toute la mémoire visible dans le supercalculateur. En fonction des jeux de données, la quantité de mémoire doit être calculée pour de pas en manquer et risquer d'effondrer les performances, ou de surestimer le besoin et perdre en rendement économique. 


\subsection{Caractéristiques des architectures à analyser}
%%%%%%%%%%%%%%%%%
    \textbf{TODO reprendre ce paragraphe pour mieux présenter les trois section suivante}
    Pour la suite de l'analyse, il est nécessaire de récupérer des caractéristiques clefs pour chaque architecture. La majorité des applications ont besoin d'un bus mémoire très performant. Cependant, certaines parties du code, séquentielles ou utilisant seulement les unités arithmétiques et logiques, auront d'autres besoins et devront être portées sur des architectures différentes. Il ne faut donc négliger aucune architecture qui pourrait s'avérer intéressante pour une partie du code ou une autre application. 
    
    \textbf{todo liste a puces des 5}
    
    \textbf{et chaque paragraphe pour les caractéristiques}
    
 
    %%%%%%%%%%%%%%%%%
    \subsubsection{La bande passante mémoire}
        Le calcul de la bande passante mémoire, noté \gls{memorypeak} et mesuré en GB/s, nécessite de connaître plusieurs caractéristiques. Il existe différentes technologies mémoires permettant d'écrire entre une, deux ou quatre fois par cycle sur chaque ligne du bus. On parle alors de mémoire Single Data Rate (SDR), Double Data Rate (DDR) et Quad Data Rate (QDR). La fréquence seule ne permet donc pas d'indiquer combien de transferts peuvent être réalisés par seconde, il faut aussi connaître le débit de données. Pour éviter les confusions, on parle alors de \textit{Mega Transferts} par seconde ($MT/s$) noté \verb|MTS|. La fréquence et le MTS des mémoires sont deux grandeurs différentes qui sont souvent confondues, par les constructeurs eux-mêmes. Par exemple la DDR4-2666, signifie que la RAM a une fréquence de 1333 MHz. La DDR4 étant une mémoire DDR, une mémoire DDR4-2666 aura un débit de 2666 \verb|MTS|. Pour calculer le débit mémoire disponible, il faut ensuite connaître le nombre de lignes reliant la mémoire au processeur, que l'on note $bus\_width$. Les architectures x86 récentes utilisent des canaux (ou \textit{memory channels}) de 64 bits. Pour obtenir une grande bande passante mémoire, les architectures utilisent plusieurs canaux mémoire notés $nb\_channels$. Ainsi, la bande passante maximum théorique \gls{memorypeak} peut alors être calculée avec la formule suivante:
        \begin{equation}
        \label{eq:bw}
            MEMORY_{peak} = MTS \times bus\_width \times nb\_channels
        \end{equation}
    
    
    %%%%%%%%%%%%%%%%%
    
    \subsubsection{La puissance de calcul}
        
        La deuxième valeur qui nous intéresse dans notre analyse est la performance crête de calcul, mesurée en nombre d'\gls{FLOPS} et notée \gls{flopspeak}. Pour la calculer, nous adaptons la notation proposée dans de précédents travaux \cite{dolbeau2015theoretical}.
        
        Pour calculer la performance maximale théorique d'un processeur, nous commençons par calculer le nombre maximal d'\gls{FLOP} exécutables par cycle. Cette performance est notée \gls{flopcycle} et mesurée en \verb=flop/cycle=. Pour la calculer, plusieurs données techniques sont nécessaires. 
        Tout d'abord, il est nécessaire de connaître la taille des instructions vectorielles. La taille de ces instructions (SIMD) et leur disponibilité dépend de l'architecture. Elle est mesurée en \verb=flop/operation=.
        Ensuite, il faut connaître le nombre maximal d'opérations exécutées par opération, mesuré en \verb=operation/instruction=. Sur les processeurs modernes, ce sont les instructions Fused Multiply Add (\gls{FMA}) qui sont capables d'exécuter deux opérations en un cycle.
        Enfin, les processeurs étant généralement superscalaires (voir \aref{sec:superscalar}), il faut obtenir le nombre d'instructions pouvant être exécutées en un cycle. Cette valeur est mesurée en \verb=instructions/cycle= et peut être trouvé à l'aide de la documentation des \gls{FPU} (voir \autoref{sec:fpu}).
        À l'aide de ces trois caractéristiques techniques, \gls{flopcycle} peut être calculé grâce à la formule suivante:
        
        \begin{equation}
        \label{eq:floc}
            FLOP_{cycle} = \frac{flop}{operation} \times \frac{operations}{instruction} \times \frac{instructions}{cycle}
        \end{equation}
        
        Une fois la performance maximale de la microarchitecture obtenue, il faut calculer la performance crête théorique atteignable par le processeur, notée \gls{flopspeak} mesurée en \gls{FLOPS}. Pour cela, il faut connaître la fréquence atteignable par le processeur lors de l'exécution des instructions SIMD utilisées pour calculer \gls{flopcycle}. Pour éviter des problèmes de surchauffe, le processeur doit abaisser sa fréquence lorsqu'il utilise de telles instructions. Un tableau de correspondance entre le type d'instructions utilisées et la fréquence soutenable est généralement donné par le constructeur. Enfin, il faut avoir le nombre de coeurs disponibles sur le processeur. La performance maximale théorique \gls{flopspeak} peut alors être calculée à l'aide de la formule suivante:
        \begin{equation}
        \label{eq:flops}
            FLOPS_{peak} = FLOP_{cycle} \times \frac{cycle}{seconde} \times nombre\ de\ coeurs
        \end{equation}
    
    

%%%%%%%%%%%%%%%%%
    
    \subsubsection{Trois caractéristiques importantes}
    %%%%%%%%%%%%%%%%%%%%%%%%%%%%%%%%%%%%%%%%%%%%%%%%%%

        Nous avons isolé trois caractéristiques qu'il est nécessaire d'avoir pour la suite de l'analyse:
        \begin{itemize}
            \item La bande passante économique mesurée en $GB/seconde/dollar$ représente le débit de données transférables par seconde pour le prix de la plateforme. Deux facteurs importants dans le choix de la plateforme entrent ici en jeu: la bande passante disponible, facteur limitant pour la majorité des codes, ainsi que l'économie qui est souvent l'élément de décision ultime. On cherchera les plateformes avec la plus grande bande passante économique.
            \item L'équilibre arithmétique mesuré en $flops/GB/s$ représente le nombre d'opérations réalisables pour chaque donnée transférée  depuis la mémoire. Cette valeur permet d'estimer l'équilibre entre le calcul et le débit mémoire d'une plateforme. Une grande valeur signifiera que la plateforme est plutôt destinée à des codes intensifs en calcul. À l'inverse, une valeur faible signifiera que la plateforme est adaptée à des codes nécessitant beaucoup d'accès mémoire. Pour la majorité des applications, on cherchera à obtenir une valeur petite.
            \item L'efficience énergétique mesurée $flop/seconde/watt$ représente le rapport d'opérations flottantes par watt d'énergie consommée. Comme discuté dans la \autoref{sec:edl_chal_energie}, la consommation électrique du supercalculateur est une contrainte majeure pour le projet Exascale. Il est donc important de privilégier des architectures avec les meilleurs rendements énergétiques. On cherche ici à obtenir la plus grande valeur possible.
        \end{itemize}
            
    Le calcul des caractéristiques par les données techniques des architectures a l'avantage de permettre d'évaluer rapidement leur potentiel sans y avoir accès. Dans la suite de cette partie, nous présentons comment certaines caractéristiques, comme la bande passante ou la puissance crête d'un processeur, peuvent être calculées.
    

\subsection{Application au processeur Intel Xeon 6148}
%%%%%%%%%%%%%%%%%
    
    Pour illustrer la présentation de la méthodologie, nous utilisons l'exemple d'un processeur Intel Xeon Skylake 6148 possédant 20 coeurs et une fréquence de base de 2,4 GHz. Une configuration à deux processeurs est présentée sur la \autoref{pic:skylake_gold}. Ce modèle de processeur possède les caractéristiques suivantes:
    \begin{itemize}
        \item \textbf{Mémoire: } le processeur étudié possède 6 canaux mémoire le connectant à 6 barrettes mémoires cadencées à 2666 MT/s. 
        \item \textbf{ALU:} les processeurs de la gamme Xeon Gold 6 possèdent tous deux unités AVX-512 capables d'exécuter chacune 2 instructions vectorielles de 512 bits (AVX-512), dont \gls{FMA}.
        \item \textbf{Fréquence:} pour une même architecture (ici Intel Skylake), chaque modèle de processeur a ses propres plages de fréquences utilisables qui peuvent être consultées en ligne \cite{Wikichipa}. La fréquence utilisable dépend essentiellement de la consommation électrique du processeur et de sa température (dépendant de la qualité du système de refroidissement). Ainsi, les fréquences soutenables par le processeur dépendent du nombre de coeurs utilisés, de la taille des instructions  exécutées (normal, AVX-2 ou AVX-512) et de la disponibilité du Turbo. Pour l'exécution d'instruction SISD (Single Instruction Single Data) avec le turbo actif sur les 20 coeurs, la fréquence maximale atteignable est de 3,1 GHz.    En fonction de l'utilisation du turbo et de la qualité du système de refroidissement, le processeur Skylake 6148 peut utiliser des fréquences allant de 1,6 GHz à 2,2 GHz \cite{Wikichipa}. 
    \end{itemize}
       
    
 
    
    \begin{figure}
        \center
        \includegraphics[width=10cm]{images/skylake_gold.png}
        \caption{\label{pic:skylake_gold} Architecture d'une plateforme avec deux processeurs Xeon Skylake (source \cite{Aspsys})}
    \end{figure}
    
    
    

    \subsubsection{Performances mémoires théoriques}
    %%%%%%%%%%%%%%%%%
        Le processeur Xeon Skylake 6148 possède 6 canaux mémoire pour accéder, dans notre expérimentation, à une mémoire DDR4-2666. En appliquant l'\autoref{eq:bw} nous obtenons une bande passante maximale de : $2666 \times 8 \times 6 = 128\ GB/s$. À cause de la loi de Little \cite{little2008little}, le processeur doit être capable de lancer plusieurs transactions simultanément (\textit{outstanding load}) pour saturer le bus mémoire et  atteindre cette performance. Nous avons montré dans la \autoref{sec:dml_saturation} que ce processeur nécessitait d'avoir au moins 15 coeurs actifs pour y parvenir.
        
    

    \subsubsection{Puissance de calcul théoriques}\label{sec:cal_teo_intel_etape1}
    %%%%%%%%%%%%%%%%%
        Le processeur étudié est un processeur superscalaire capable d'exécuter jusqu'à 4 instructions par cycle, dont deux opérations à virgule flottante. Ces opérations pouvant être des instructions \gls{FMA} vectorielles de 512 bits. Il est donc possible de calculer sur chaque ALU, une multiplication et une addition par cycle sur 8 éléments simultanément. On peut ainsi calculer la performance crête de ce processeur en appliquant l'\autoref{eq:flops}. Suivant la fréquence utilisable par le processeur (dépendant de la température) la performance crête théorique (\gls{flopspeak}) est comprise entre $8 \times 2 \times 2 \times 1.6 \times 20 = 1024$ GFLOPS et  $8 \times 2 \times 2 \times 2.2 \times 20 = 1408$ GFLOPS. 
        
        Cependant, pour comparer la performance de l'application avec ce résultat, il faut que la nature du code puisse utiliser des instructions FMA vectorisées. Il peut être intéressant de disposer d'une fourchette de performance lorsque la totalité du parallélisme est utilisée ou non. Quand le processeur n'utilise pas d'instruction AVX-512, le processeur est capable d'atteindre 3.1 GHz lorsque les 20 coeurs sont actifs. En reprenant l'\autoref{eq:flops}, la performance optimale d'une telle application serait: \gls{flopsmax} $= 1 \times 1 \times 2 \times 3.1 \times 20 = 124$ GFLOPS.
    

    \subsubsection{Équilibre arithmétique}
    %%%%%%%%%%%%%%%%%
        L'équilibre arithmétique du processeur permet d'évaluer s'il est approprié pour un code nécessitant une grande bande passante ou plutôt de bonnes performances de calculs. En réutilisant les deux caractéristiques précédemment calculées, on peut calculer $\text{EQUILIBRE}_{non\_avx}$ et $\text{EQUILIBRE}_{avx\_512}$ qui bornent la performance inférieure et supérieure de ce processeur. On obtient ainsi $\text{EQUILIBRE}_{non\_avx} = \frac{124}{128} = 0.97\ flop/byte$ et $EQUILIBRE_{avx\_512} = \frac{1408}{128} = 11\ \text{flopbyte}$. L'équilibre arithmétique est utilisé pour construire le modèle du Roofline présenté dans la \autoref{sec:roofline}.