\section{Conclusion et perspectives}\label{sec:methodo_conclusion}

\subsection{Conclusion}
%%%%%%%%%%%%%%%%%%%%%%%%%%%%%%%%%%%%%%%%%%

Cette méthodologie nous a montré qu'avec des outils très simple, le programmeur peut mener à bien une analyse très complexes et précise de son application.
En combinant les outils ont améliore notre expertise

Grâce à cette méthodologie, plusieurs deals HPC stratégiques ont pu être remportés. L’un d’entre eux fut lors de l’appel d’offres d’un client de défense français qui autorisait le choix de plateformes spécialisées et la transformation du code. Grâce à notre méthodologie, la performance maximale théorique de l’algorithme pour la plateforme sélectionnée a été atteinte. Plus récemment, nos outils ont permis de comprendre et d’isoler un bug majeur dans une nouvelle architecture prometteuse (alors inconnu par le fabricant) ne délivrant pas la performance attendue. Cette méthode a également été appliquée pour remporter un concours d’optimisation d’applications. Ce concours sponsorisé par Intel avait pour objectif d’optimiser une application de calcul distribué MPI pour la dynamique de systèmes particulaires. En utilisant nos outils, notamment ceux de profiling de performance, une accélération d’un facteur 10 a pu être réalisée.


\subsection{Perspectives}
%%%%%%%%%%%%%%%%%%%%%%%%%%%%%%%%%%%%%%%%%%%




        La puissance vient de la méthodologie
            Notre methodologie montre qu'avec des outils très simple, le programmeur peut mener à bien une analyse très complexes et précise de son application
            En combinant les outils ont améliore notre expertise
            Image de la méthodologie
        
