\chapter{Méthodologie simple pour l'analyse de performance et le portage de code}
\label{chap:methodo}
\minitoc

Le chapitre précédent présente les différents outils développés durant le travail de thèse. Ces outils ont pour objectif d'aider le programmeur dans sa démarche de caractérisation d'architecture et d'optimisation d'applications. Dans ce chapitre nous proposons une méthodologie à suivre pour réaliser ce travail. La \autoref{sec:methodo_intro} présente les motivations qui nous ont poussé à développer une telle méthodologie. Les cinq sections suivante correspondent aux cinq étapes de la méthodologie



\section{Introduction}\label{sec:methodo_intro}

À cause des pressions énergétique et économique, les plateformes de calculs haute performance doivent être repensées. De nouvelles architectures optimisées pour certaines applications devront être utilisées rendues possibles grâce au protocole Gen-Z. En l'absence de méthode de caractérisation fine de la performance des codes, ces architectures inovantes sont potentiellement condamnées puisque peu d'experts savent les valoriser. Cependant, ces nouvelles architectures très différentes de celles utilises aujourd'hui (x86, GPU) doivent être caractérisées pour prédire le gain de performance pour les applications. Pour pouvoir profiter de ces technologies et les utiliser de façon optimale, nous avons présenté dans le chapitre précédent une suite de logiciels de caractérisation et d'analyse de performance. L'objectif de ce chapitre est de présenter une méthodologie adaptée,  permettent de réaliser cette caractérisation ainsi que le portage des applications sur ces nouveaux accélérateurs.



\subsection{La révolution de l'hétérogénéité}
%%%%%%%%%%%%%%%%%%%%%%%%%%%%%%%%%%%%%%%%%%%%%%%

    Le protocole Gen-Z, présenté dans le chapitre \ref{sec:edl_hpc}, va révolutionner le monde de l'informatique comme peu de technologies auparavant. Entre tous les bénéfices apportés par ce protocole, la faculté de rendre facile l'hétérogénéité dans les super-calculateurs est sans doute la plus importante. 
    
    L'hétérogénéité sera à la fois entre des accélérateurs spécialisés pour différentes workload, mais aussi dans les architectures elles mêmes permettant de mixer et d'adapter chaque coprocesseur. En 2018, plus de 96\% des processeurs des super-calculateurs du Top500 ont une architecture x86 et la majorité d'entre eux (91\%) proviennent du constructeur Intel. Seulement 28\% des 500 clusters sont associés à des accélérateurs dont 92\% sont des GPU NVIDIA. Nous remarquons donc que l'architecture des super-calculateurs est très similaire et que l'utilisation d'accélérateurs adaptés n'est encore qu'à ses débuts. Aujourd'hui l'utilisation d'architectures différentes est souvent perçu négativement car elle implique d'adapter les codes, d'utiliser plusieurs langages ou d'obtenir de mauvaises performance car l'architecture n'est pas adaptée à la totalité de l'application.
    
    Par analogie, nous comparons cette opportunité avec celle des moteurs d'avion à réaction qui ont révolutionné l'économie et le domaine de l'aviation. Si certains constructeurs et utilisateurs continuent d'utiliser la même stratégie d'ajouter des serveurs \textit{standards} comme jusqu'à aujourd'hui, ils seront dépassés par ceux ayant commencé à investir ces nouvelles technologies plusieurs années avant eux. Il est donc cruciale de s'y préparer en ayant la bonne méthodologie et les bons outils pour pouvoir en profiter. L'accès à des plate-forme exascale et à ces nouvelles architectures va aussi ouvrir de nouveaux marchés, inaccessible aujourd'hui à cause de plusieurs contraintes: le prix, la bande passante nécessaire, la sécurité ou encore la consommation électrique. 
    
    Les gains de performance ne viendront pas seulement par l'utilisation d'accélérateurs puissants, mais de leur diversité et de la capacité des programmeurs de bien les utiliser. Pour une même application, plusieurs accélérateurs spécialisés seront souvent nécessaires. On peut en imaginer certains adaptés à la lecture et à la décompression du jeu de données. Un fois réalisé, des accélérateurs spécialisés dans le calcul demandé pourront être utilisés (ASIC, FGPA ou DSP). Enfin pour la visualisation des données, des GPU seront alors nécessaires. L'hétérogénéité est à la fois un challenge majeur des plate-formes Exascale, mais aussi une grande opportunité. 
    
    

\subsection{Développement}
%%%%%%%%%%%%%%%%%%%%%%%%%%%%%%%%%%%%%%%%%%%%%%%

    Comme présente dans la section \ref{X}, l'analyse de performance peut se faire à plusieurs niveaux. En fonction de l'objectif défini, le niveau et les outils utilisés doivent être adaptés. Notre analyse porte sur la performance d'une partie du code intéressante, un hot spot. Les outils développé permettent pour le moment de mettre en avant des problèmes de performances dus au système mémoire ou au processeur. Les problèmes de performances liés au réseaux ou au système d'exploitation ne sont pas la priorité de la méthodologie, bien qu'ils puissent être décelé.
    
    Du fait de la complexité des architectures, le travail de portage et d'optimisation peut être très difficile. Ainsi, notre démarche s'adresse aux programmeurs ayant de solides connaissances des micro-architectures. Pour améliorer la précision de l'analyse et des outils il est préférable d'avoir accès au code source. 
    
    Contrairement à des solutions existante comme VTune \cite{vtune}, nous avons choisi de développer plusieurs outils indépendant répondant chacun à une question précise. Nous espérons qu'en réduisant la complexité de l'outillage, l'adoption des outils auprès des programmeurs sera plus grande. Les outils n'ont pas vocation d'automatiser entièrement les tâches du programmeur. La puissance de ces outils vient de leur utilisation complémentaire. 
    
    Les outils utilisés sont disponibles en Open Source et ne sont pas exhaustifs. Certains outils utilisés ont été développé durant la thèse, d'autres répondant à nos critères n'ont pas eu à être développés de nouveau. Cette méthodologie et les outils l'accompagnant sont présentés pour partager notre philosophie d'analyse de performance, mais le travail doit être poursuivi pour ajouter de nouveaux outils. De plus, le contexte de leur utilisation est de profiter de l'hétérogénéité arrivant dans les centre de données, ils nécessiteront d'être portés sur ces architectures.



    \begin{figure}
        \center
        \includegraphics[width=10cm]{images/analyse.png}
        \caption{\label{pic_analyse} Délimitation de l'analyse proposée.}
    \end{figure}

    
    
\subsection{Contributions}
%%%%%%%%%%%%%%%%%%%%%%%%%%%%%%%%%%%%%%%%%%%%%%%

    Dans ce chapitre, nous présentons une méthodologie simple en 5 étapes permettant aux utilisateurs de modéliser les performances de leur code, de les projeter sur de nouvelles architectures et de les optimiser (voir \autoref{pic:methodologie_step_2}). Nous proposons un modèle de performance simple basé sur les caractéristiques du sous-système mémoire. L'objectif est de créer pour chaque \textit{hot spot} un modèle de ses performances dans le but de projeter ses performances sur d'autres architectures mais aussi de valider ses performances. Nous cherchons à prouver la bonne utilisation ou non du système mémoire, ressource critique pour la performance des applications sur les architectures modernes. Enfin, lorsque les performances de l'application ne sont pas celles attendues par notre modèle, nous proposons un cheminement pour comprendre, optimiser et transformer le code pour parvenir aux performances ultimes.

    \begin{figure}
    \center
    \includegraphics[width=14cm]{images/methodologie_step.png}
    \caption{\label{pic:methodologie_step_2} Méthodologie en 5 étapes pour caractériser et optimiser une application sur une nouvelle architecture.}
    \end{figure}
            
    
    Pour illustrer les différentes étapes, nous appliquons la méthodologie à l'étude des performances de la fonction \textit{triadd} (voir extrait de code \ref{lst:triadd}) du benchmark Stream \cite{McCalpin1995} sur un processeur Intel\textit{ Xeon Gold 6148} possédant 20 coeurs. Les matrices utilisées mesurent chacune 19.6 GB. Cet exercice nous permet de montrer que même pour un code aussi simple et en apparence optimisée, l'approche et les outils utilisés permettent de comprendre et d'optimiser ses performances.
    
\begin{lstlisting}[language=c,caption=Fonction Triadd extraite du benchmark Stream \ref{McCalpin1995},label={lst:triadd}, 
  basicstyle=\footnotesize, frame=tb,
  xleftmargin=.065\textwidth, xrightmargin=.065\textwidth]
for (j=0; j < STREAM_ARRAY_SIZE; j++)
    A[j] = B[j] + scalar * C[j];
\end{lstlisting}


%%%%%%%%%%%%%%%%%%%%%%%%%%%%%%%%%%%%%%%%%%%%%%%%%%%%%%%%%%%%%%%%%%%
\section{Étape 1: Rechercher les dernières innovations technologiques}\label{sec:methodo_step1}
%%%%%%%%%%%%%%%%%%%%%%%%%%%%%%%%%%%%%%%%%%%%%%%%%%%%%%%%%%%%%%%%%%%


%%%%%%%%%%%%%%%%%
\subsection{Introduction}
%%%%%%%%%%%%%%%%%
    
    \subsubsection{Motivations}
    %%%%%%%%%%%%%%%%%%%%%%%%%%%
    
        Le domaine du calcul haute performance est très concurrentiel. Les industries ayant recours à ces plateformes doivent être à la pointe de la technologie au risque de se faire dépasser par un concurrent. 
        Comme vu dans la \autoref{sec:methodo_intro_hetero} \textbf{todo ref plus a jour}, la grande majorité des plateformes utilisent les mêmes architectures. Jusqu'à aujourd'hui, peu d'entreprises se sont démarquées par l'utilisation d'un nouvel accélérateur.
      
        Aujourd'hui, les choix d'architectures disponibles se limitent à quelques architectures: les processeurs (Intel, AMD, IBM) et les \gls{GPU} (NVIDIA, AMD, Xeon Phi). Les technologies émergentes présentées dans la \autoref{sec:oppo}, vont permettre le développement de nouvelles architectures. Grâce au protocole \verb=Gen-Z=, il sera plus facile de les utiliser ensemble au sein d'une même plateforme. Il est donc indispensable pour les industries, mais aussi pour les constructeurs tels que \verb=Hewlett Packard Enterprise=, de connaître et d'utiliser les meilleures d'entre elles. Porter son application sur une nouvelle plateforme nécessite d'investir du temps et de l'argent. Cette décision très importante se complexifie avec l'augmentation du nombre d'architectures différentes.
   
   \subsubsection{Objectifs}
   %%%%%%%%%%%%%%%%%%%%%%%%%
        
        Le premier travail des développeurs et des architectes de plateformes \gls{HPC} est d'être en constante recherche des dernières innovations technologiques. Cela peut être l'annonce d'un nouveau processeur, d'une nouvelle technologie mémoire ou bien de nouveaux algorithmes ou de nouvelles optimisations. Il est très important de se tenir à l'état de l'art ou même en avance pour anticiper les nouveautés. 

        Le but principal de cette étape est de répertorier toutes les plateformes et technologies potentiellement intéressantes pour le calcul haute performance. Certaines caractéristiques clés sont calculées à partir des spécificités techniques des architectures, d'autres sont mesurées dans l'étape suivante lorsque l'accès aux plateformes est disponible.
    

        %\paragraph{Processeurs et accélérateurs.} 
        Dans notre vision, la majorité des lignes de codes continueront d'être exécutées sur des architectures semblables à celles d'aujourd'hui (x86 et PowerPC). Seuls les \glspl{kernel} seront déportés sur les accélérateurs adéquats. Il est donc nécessaire de continuer à s'y intéresser et à les caractériser. Les processeurs présents dans ces nouvelles générations de plateformes pourraient alors être bien différents de ceux utilisés actuellement. En effet, si les kernels des applications ne sont plus exécutés sur ces processeurs, les caractéristiques recherchées seront différentes. Les accélérateurs actuels continueront d'avoir leur rôle à jouer. Les GPU se montrent extrêmement efficaces pour les algorithmes d'apprentissage par machine et d'intelligence artificielle. L'objectif de l'industrie est de trouver des accélérateurs aussi efficaces pour l'exécution d'autres types d'algorithmes.

        %\paragraph{Mémoires.} Grâce à Gen-Z, la totalité de l'architecture sera \textit{composable}, pas seulement au niveau des processeurs, mais aussi au niveau des mémoires. Grâce à sa sémantique d'accès \textit{load/store}, Gen-Z va permettre au processeur d'accéder à toute la mémoire visible dans le supercalculateur. En fonction des jeux de données, la quantité de mémoire doit être calculée pour de pas en manquer et risquer d'effondrer les performances, ou de surestimer le besoin et perdre en rendement économique. 


\subsection{Caractéristiques des architectures à analyser}
%%%%%%%%%%%%%%%%%
    \textbf{TODO reprendre ce paragraphe pour mieux présenter les trois section suivante}
    Pour la suite de l'analyse, il est nécessaire de récupérer des caractéristiques clefs pour chaque architecture. La majorité des applications ont besoin d'un bus mémoire très performant. Cependant, certaines parties du code, séquentielles ou utilisant seulement les unités arithmétiques et logiques, auront d'autres besoins et devront être portées sur des architectures différentes. Il ne faut donc négliger aucune architecture qui pourrait s'avérer intéressante pour une partie du code ou une autre application. 
    
    \textbf{todo liste a puces des 5}
    
    \textbf{et chaque paragraphe pour les caractéristiques}
    
 
    %%%%%%%%%%%%%%%%%
    \subsubsection{La bande passante mémoire}
        Le calcul de la bande passante mémoire, noté \gls{memorypeak} et mesuré en GB/s, nécessite de connaître plusieurs caractéristiques. Il existe différentes technologies mémoires permettant d'écrire entre une, deux ou quatre fois par cycle sur chaque ligne du bus. On parle alors de mémoire Single Data Rate (SDR), Double Data Rate (DDR) et Quad Data Rate (QDR). La fréquence seule ne permet donc pas d'indiquer combien de transferts peuvent être réalisés par seconde, il faut aussi connaître le débit de données. Pour éviter les confusions, on parle alors de \textit{Mega Transferts} par seconde ($MT/s$) noté \verb|MTS|. La fréquence et le MTS des mémoires sont deux grandeurs différentes qui sont souvent confondues, par les constructeurs eux-mêmes. Par exemple la DDR4-2666, signifie que la RAM a une fréquence de 1333 MHz. La DDR4 étant une mémoire DDR, une mémoire DDR4-2666 aura un débit de 2666 \verb|MTS|. Pour calculer le débit mémoire disponible, il faut ensuite connaître le nombre de lignes reliant la mémoire au processeur, que l'on note $bus\_width$. Les architectures x86 récentes utilisent des canaux (ou \textit{memory channels}) de 64 bits. Pour obtenir une grande bande passante mémoire, les architectures utilisent plusieurs canaux mémoire notés $nb\_channels$. Ainsi, la bande passante maximum théorique \gls{memorypeak} peut alors être calculée avec la formule suivante:
        \begin{equation}
        \label{eq:bw}
            MEMORY_{peak} = MTS \times bus\_width \times nb\_channels
        \end{equation}
    
    
    %%%%%%%%%%%%%%%%%
    
    \subsubsection{La puissance de calcul}
        
        La deuxième valeur qui nous intéresse dans notre analyse est la performance crête de calcul, mesurée en nombre d'\gls{FLOPS} et notée \gls{flopspeak}. Pour la calculer, nous adaptons la notation proposée dans de précédents travaux \cite{dolbeau2015theoretical}.
        
        Pour calculer la performance maximale théorique d'un processeur, nous commençons par calculer le nombre maximal d'\gls{FLOP} exécutables par cycle. Cette performance est notée \gls{flopcycle} et mesurée en \verb=flop/cycle=. Pour la calculer, plusieurs données techniques sont nécessaires. 
        Tout d'abord, il est nécessaire de connaître la taille des instructions vectorielles. La taille de ces instructions (SIMD) et leur disponibilité dépend de l'architecture. Elle est mesurée en \verb=flop/operation=.
        Ensuite, il faut connaître le nombre maximal d'opérations exécutées par opération, mesuré en \verb=operation/instruction=. Sur les processeurs modernes, ce sont les instructions Fused Multiply Add (\gls{FMA}) qui sont capables d'exécuter deux opérations en un cycle.
        Enfin, les processeurs étant généralement superscalaires (voir \aref{sec:superscalar}), il faut obtenir le nombre d'instructions pouvant être exécutées en un cycle. Cette valeur est mesurée en \verb=instructions/cycle= et peut être trouvé à l'aide de la documentation des \gls{FPU} (voir \autoref{sec:fpu}).
        À l'aide de ces trois caractéristiques techniques, \gls{flopcycle} peut être calculé grâce à la formule suivante:
        
        \begin{equation}
        \label{eq:floc}
            FLOP_{cycle} = \frac{flop}{operation} \times \frac{operations}{instruction} \times \frac{instructions}{cycle}
        \end{equation}
        
        Une fois la performance maximale de la microarchitecture obtenue, il faut calculer la performance crête théorique atteignable par le processeur, notée \gls{flopspeak} mesurée en \gls{FLOPS}. Pour cela, il faut connaître la fréquence atteignable par le processeur lors de l'exécution des instructions SIMD utilisées pour calculer \gls{flopcycle}. Pour éviter des problèmes de surchauffe, le processeur doit abaisser sa fréquence lorsqu'il utilise de telles instructions. Un tableau de correspondance entre le type d'instructions utilisées et la fréquence soutenable est généralement donné par le constructeur. Enfin, il faut avoir le nombre de coeurs disponibles sur le processeur. La performance maximale théorique \gls{flopspeak} peut alors être calculée à l'aide de la formule suivante:
        \begin{equation}
        \label{eq:flops}
            FLOPS_{peak} = FLOP_{cycle} \times \frac{cycle}{seconde} \times nombre\ de\ coeurs
        \end{equation}
    
    

%%%%%%%%%%%%%%%%%
    
    \subsubsection{Trois caractéristiques importantes}
    %%%%%%%%%%%%%%%%%%%%%%%%%%%%%%%%%%%%%%%%%%%%%%%%%%

        Nous avons isolé trois caractéristiques qu'il est nécessaire d'avoir pour la suite de l'analyse:
        \begin{itemize}
            \item La bande passante économique mesurée en $GB/seconde/dollar$ représente le débit de données transférables par seconde pour le prix de la plateforme. Deux facteurs importants dans le choix de la plateforme entrent ici en jeu: la bande passante disponible, facteur limitant pour la majorité des codes, ainsi que l'économie qui est souvent l'élément de décision ultime. On cherchera les plateformes avec la plus grande bande passante économique.
            \item L'équilibre arithmétique mesuré en $flops/GB/s$ représente le nombre d'opérations réalisables pour chaque donnée transférée  depuis la mémoire. Cette valeur permet d'estimer l'équilibre entre le calcul et le débit mémoire d'une plateforme. Une grande valeur signifiera que la plateforme est plutôt destinée à des codes intensifs en calcul. À l'inverse, une valeur faible signifiera que la plateforme est adaptée à des codes nécessitant beaucoup d'accès mémoire. Pour la majorité des applications, on cherchera à obtenir une valeur petite.
            \item L'efficience énergétique mesurée $flop/seconde/watt$ représente le rapport d'opérations flottantes par watt d'énergie consommée. Comme discuté dans la \autoref{sec:edl_chal_energie}, la consommation électrique du supercalculateur est une contrainte majeure pour le projet Exascale. Il est donc important de privilégier des architectures avec les meilleurs rendements énergétiques. On cherche ici à obtenir la plus grande valeur possible.
        \end{itemize}
            
    Le calcul des caractéristiques par les données techniques des architectures a l'avantage de permettre d'évaluer rapidement leur potentiel sans y avoir accès. Dans la suite de cette partie, nous présentons comment certaines caractéristiques, comme la bande passante ou la puissance crête d'un processeur, peuvent être calculées.
    

\subsection{Application au processeur Intel Xeon 6148}
%%%%%%%%%%%%%%%%%
    
    Pour illustrer la présentation de la méthodologie, nous utilisons l'exemple d'un processeur Intel Xeon Skylake 6148 possédant 20 coeurs et une fréquence de base de 2,4 GHz. Une configuration à deux processeurs est présentée sur la \autoref{pic:skylake_gold}. Ce modèle de processeur possède les caractéristiques suivantes:
    \begin{itemize}
        \item \textbf{Mémoire: } le processeur étudié possède 6 canaux mémoire le connectant à 6 barrettes mémoires cadencées à 2666 MT/s. 
        \item \textbf{ALU:} les processeurs de la gamme Xeon Gold 6 possèdent tous deux unités AVX-512 capables d'exécuter chacune 2 instructions vectorielles de 512 bits (AVX-512), dont \gls{FMA}.
        \item \textbf{Fréquence:} pour une même architecture (ici Intel Skylake), chaque modèle de processeur a ses propres plages de fréquences utilisables qui peuvent être consultées en ligne \cite{Wikichipa}. La fréquence utilisable dépend essentiellement de la consommation électrique du processeur et de sa température (dépendant de la qualité du système de refroidissement). Ainsi, les fréquences soutenables par le processeur dépendent du nombre de coeurs utilisés, de la taille des instructions  exécutées (normal, AVX-2 ou AVX-512) et de la disponibilité du Turbo. Pour l'exécution d'instruction SISD (Single Instruction Single Data) avec le turbo actif sur les 20 coeurs, la fréquence maximale atteignable est de 3,1 GHz.    En fonction de l'utilisation du turbo et de la qualité du système de refroidissement, le processeur Skylake 6148 peut utiliser des fréquences allant de 1,6 GHz à 2,2 GHz \cite{Wikichipa}. 
    \end{itemize}
       
    
 
    
    \begin{figure}
        \center
        \includegraphics[width=10cm]{images/skylake_gold.png}
        \caption{\label{pic:skylake_gold} Architecture d'une plateforme avec deux processeurs Xeon Skylake (source \cite{Aspsys})}
    \end{figure}
    
    
    

    \subsubsection{Performances mémoires théoriques}
    %%%%%%%%%%%%%%%%%
        Le processeur Xeon Skylake 6148 possède 6 canaux mémoire pour accéder, dans notre expérimentation, à une mémoire DDR4-2666. En appliquant l'\autoref{eq:bw} nous obtenons une bande passante maximale de : $2666 \times 8 \times 6 = 128\ GB/s$. À cause de la loi de Little \cite{little2008little}, le processeur doit être capable de lancer plusieurs transactions simultanément (\textit{outstanding load}) pour saturer le bus mémoire et  atteindre cette performance. Nous avons montré dans la \autoref{sec:dml_saturation} que ce processeur nécessitait d'avoir au moins 15 coeurs actifs pour y parvenir.
        
    

    \subsubsection{Puissance de calcul théoriques}\label{sec:cal_teo_intel_etape1}
    %%%%%%%%%%%%%%%%%
        Le processeur étudié est un processeur superscalaire capable d'exécuter jusqu'à 4 instructions par cycle, dont deux opérations à virgule flottante. Ces opérations pouvant être des instructions \gls{FMA} vectorielles de 512 bits. Il est donc possible de calculer sur chaque ALU, une multiplication et une addition par cycle sur 8 éléments simultanément. On peut ainsi calculer la performance crête de ce processeur en appliquant l'\autoref{eq:flops}. Suivant la fréquence utilisable par le processeur (dépendant de la température) la performance crête théorique (\gls{flopspeak}) est comprise entre $8 \times 2 \times 2 \times 1.6 \times 20 = 1024$ GFLOPS et  $8 \times 2 \times 2 \times 2.2 \times 20 = 1408$ GFLOPS. 
        
        Cependant, pour comparer la performance de l'application avec ce résultat, il faut que la nature du code puisse utiliser des instructions FMA vectorisées. Il peut être intéressant de disposer d'une fourchette de performance lorsque la totalité du parallélisme est utilisée ou non. Quand le processeur n'utilise pas d'instruction AVX-512, le processeur est capable d'atteindre 3.1 GHz lorsque les 20 coeurs sont actifs. En reprenant l'\autoref{eq:flops}, la performance optimale d'une telle application serait: \gls{flopsmax} $= 1 \times 1 \times 2 \times 3.1 \times 20 = 124$ GFLOPS.
    

    \subsubsection{Équilibre arithmétique}
    %%%%%%%%%%%%%%%%%
        L'équilibre arithmétique du processeur permet d'évaluer s'il est approprié pour un code nécessitant une grande bande passante ou plutôt de bonnes performances de calculs. En réutilisant les deux caractéristiques précédemment calculées, on peut calculer $\text{EQUILIBRE}_{non\_avx}$ et $\text{EQUILIBRE}_{avx\_512}$ qui bornent la performance inférieure et supérieure de ce processeur. On obtient ainsi $\text{EQUILIBRE}_{non\_avx} = \frac{124}{128} = 0.97\ flop/byte$ et $EQUILIBRE_{avx\_512} = \frac{1408}{128} = 11\ \text{flopbyte}$. L'équilibre arithmétique est utilisé pour construire le modèle du Roofline présenté dans la \autoref{sec:roofline}.
\section{Étape 2: Caractériser les architectures}\label{sec:methodo_step2}
%%%%%%%%%%%%%%%%%%%%%%%%%%%%%%%%%%%%%%%%%%%%%%%%%%%%%%%%%%%%%%%%%%%

Lors de la première étape, les caractéristiques des architectures ont été rassemblées pour sélectionner celles avec le meilleur potentiel. Pour l'étape deux, il est nécessaire d'avoir accès aux différentes architectures pour réaliser une caractérisation fine de leur comportement et de leurs performances.

\subsection{Introduction}
%%%%%%%%%%%%%%%%%

    %\subsubsection{Performance de référence}
    %%%%%%%%%%%%%%%%%%%%%%%%%
        
        Pour pouvoir estimer la bonne ou mauvaise performance d'un code sur une plateforme, il est nécessaire d'avoir une performance de référence avec laquelle la comparer. Grâce à cette référence, il est ensuite possible d'estimer les gains de performance dont pourrait profiter une application suite à son optimisation ou à son portage sur une nouvelle architecture. Un modèle largement utilisé dans le domaine de l'analyse de performance est celui du Roofline, présenté dans la \autoref{sec:roofline}. Pour sa construction, il faut disposer de deux caractéristiques de l'architecture: le débit mémoire (mesuré en GB/s) et le débit de calcul (mesuré en \gls{FLOP}). Pour les obtenir, deux méthodes sont possibles (voir \autoref{sec:methodo_step2}). La première est de les calculer à partir des données techniques de l'architecture et la deuxième est de la mesurer.
        
        \begin{figure}[h!]
        \center
        \includegraphics[width=14cm]{images/methodo_step2.png}
        \caption{\label{pic:methodo_step2} Les performances d'une architecture peuvent être calculées à l'aide des caractéristiques techniques (\gls{flopspeak}, \gls{memorypeak}) ou mesurées à l'aide d'applications spécialisées (\gls{flopsmax}, \gls{memorymax}).}
        \end{figure}


    \subsubsection{Performance maximale théorique}
    %%%%%%%%%%%%%%%%%%%%%%%%%%%
        
        L'implémentation \textit{naïve} \cite{Williams2008a} du modèle du Roofline utilise les performances théoriques de la microarchitecture. Celle du système mémoire est notée (\gls{memorypeak}) et celle du processeur (\gls{flopspeak}). Malheureusement, à cause de la complexité des architectures, il est très rare qu'une application atteigne une performance égale à la performance théorique. Même des applications spécialisées telles que \verb=STREAM= \cite{McCalpin1995} et \verb=HPL= \cite{Dongarra2003} n'y parviennent pas. Il est donc très rare de voir des applications industrielles atteindre la performance maximale théorique des processeurs.
        
        

        Lors de l'étape 1, nous avons utilisé les caractéristiques techniques des architectures pour réaliser un premier tri, sans même y avoir accès physiquement. L'objectif de cette deuxième étape est d'obtenir des valeurs de références pour pouvoir projeter et apprécier les performances d'une application.
    
        
    \subsubsection{Performance maximale mesurée}
    %%%%%%%%%%%%%%%%%%%%%%%%%%%
    
        Afin de connaître précisément les performances d'une architecture, il est conseillé d'utiliser des applications spécialisées pour chaque caractéristique. Une application industrielle peut être difficile à porter sur une nouvelle architecture. Il est donc peu recommandé de réaliser ce portage dans le simple but de caractériser cette dernière. Il est préférable d'utiliser des \glspl{benchmark}, plus courts et plus facilement portables. Après le portage de l'application sur la plateforme choisie, il sera possible de comparer la performance mesurée de l'application (performance \textit{maximale}\protect\footnotemark), avec la performance théorique atteignable (performance \textit{crête}\textsuperscript{\ref{note1}}).

    
        Pour obtenir les deux caractéristiques qui nous intéressent ici (notées \gls{memorymax} et \gls{flopsmax}), il est courant d'utiliser des benchmarks de références. Par exemple pour mesurer la bande passante mémoire maximale atteignable, on peut utiliser le benchmark \textit{Stream} \cite{McCalpin1995}. Afin d'obtenir les latences des différents niveaux de caches, \textit{lmbench} \cite{Staelin2002} peut alors être utilisé. Pour mesurer le nombre maximum d'opérations sur un nombre flottant exécutable par seconde, on peut utiliser le benchmark HPL.
        
        L'avantage de cette approche est sa facilité. Il suffit de compiler le benchmark voulu et de l'exécuter pour obtenir les résultats. L'inconvénient est que la mesure est dépendante de la qualité du code et du compilateur. Il peut arriver qu'en voulant mesurer spécifiquement un composant, les performances soient dégradées par une autre partie de la microarchitecture, par exemple, si l'on cherche à mesurer la bande passante maximale atteignable par un seul coeur avec un code simple lisant un tableau de données. Sur un processeur récent tel qu'un Xeon Skylake, on s'attendrait à obtenir une valeur proche du maximum théorique de 128 GB/s, calculé dans lors de l'étape précédente. Cependant, à cause de la Loi de Little \cite{little2008little} et de la taille de la queue de chargement (\textit{outstanding load queue}), il faut plus de quinze coeurs actifs pour saturer la bande passante \cite{JohnMcCalpin2010}. Il faut donc une certaine expérience des outils et des microarchitectures pour apprécier les résultats mesurés. Il peut ainsi être nécessaire d'avoir plusieurs codes de benchmark à exécuter pour valider de différentes façons le comportement de la microarchitecture.
        
        \footnotetext{\label{note1}Le classement du Top500 utilise les notations $\text{R}_{max}$ et $\text{R}_{peak}$ pour   désigner la performance maximale atteignable et la performance crête (\textit{peak}) théorique d'un supercalculateur.}


  
%%%%%%%%%%%%%%%%%
\subsection{Application au processeur Intel Xeon 6148}
%%%%%%%%%%%%%%%%%
    
    Nous présentons ici comment les outils développés durant la thèse permettent de caractériser la microarchitecture du processeur Intel Xeon Skylake 6148. Nous mesurons le débit mémoire maximale du bus mémoire (\gls{memorymax}) à l'aide de \verb=DML_MEM= (voir \autoref{sec:yamb}) et la puissance de calcul du processeur (\gls{flopsmax}) grâce au \verb=Kernel Generator= (voir \autoref{sec:kg}).


    \subsubsection{Débit mémoire maximal}
    %%%%%%%%%%%%%%%%%%%%%%%%%%%%%%%%%%%%
        
        Pour caractériser le système mémoire du processeur, nous paramétrons l'outil \verb=DML_MEM= pour réaliser des accès mémoire avec des sauts en mémoire de la taille d'une ligne de cache.
        Pour nous assurer de la capacité du processeur à générer suffisamment de transactions mémoires pour saturer le bus (voir Loi de Little \autoref{sec:loidelittle}), nous exécutons le benchmark sur les 20 coeurs disponibles. La commande résultante pour cette expérimentation est la suivante:
\begin{lstlisting}
mpirun -np 20 numactl dml_mem  --steplog 0 --unroll 8 
                               --type read --stride 64 --matrixsize 5000
\end{lstlisting}
        Lors de l'étape 1, nous avions calculé une performance crête théorique $\text{MEMORY}_{peak}$ de 128 GB/s. A l'aide de \verb=DML_MEM=, nous obtenons une bande passante mémoire \gls{memorymax} de 105 GB/s.
        
    
    
    \subsubsection{Performance de calcul}
    %%%%%%%%%%%%%%%%%%%%%%%%%%%%%%%%%%%% 
    
        Pour mesurer la performance maximale atteignable (\gls{flopsmax}) par le processeur, nous utilisons le générateur de benchmark \verb|Kernel Generator|. Nous avons paramétré le générateur pour évaluer le nombre maximum d'opérations \gls{FMA} AVX-512 réalisable sur un nombre flottant à double précision. Pour cela, nous avons utilisé la commande présentée ci-dessous. Nous générons un kernel de 14 instructions FMA pour réduire le coût de gestion de la boucle (incrémentation et comparaison) et masquer la latence des instructions.

\begin{lstlisting}
./kg -W 512 -O ffffffffffffff -P double -S 100 -L 120000000
\end{lstlisting}
        
        Le benchmark généré ne possède pas de version multicoeur, nous lançons donc indépendamment 20 exécutions du binaire qui sont accrochées à un coeur différent grâce à un paramètre du benchmark. Pour cette expérience, nous avons commencé par désactiver le turbo du processeur, et limiter sa fréquence à 1,6 GHz. La performance mesurée est de 998,57 GFLOPS. Ensuite nous avons activé le turbo, et laissé le processeur choisir lui même sa fréquence. L'\autoref{code:kg_512_output} montre les résultats donnés par le benchmark pour l'exécution sur un des 20 coeurs. Le benchmark mesure sa fréquence effective et trouve effectivement la valeur de 2,2 GHz renseignée par Intel dans sa documentation. La performance maximale d'un coeur mesurée est de 69,2 GFLOPS, approchant le maximum théorique de 70 GFLOPS calculé dans l'étape 1 (voir \autoref{sec:cal_teo_intel_etape1}).\\
        
\begin{minipage}{\linewidth}
\begin{lstlisting}[caption=Résultat de l'exécution du benchmark sur un coeur avec le turbo activé, label={code:kg_512_output},
  basicstyle=\footnotesize, frame=tb,
  xleftmargin=.005\textwidth, xrightmargin=.005\textwidth]
------------------  INSTRUCTIONS SUMMARY ------------------------------
_label_|   NB INSTRUCTIONS      Time    FREQUENCY    inst/sec       IPC
_value_|      168000000000      38.8          2.2     4.33e+9      2.01
----------------------  FLOP SUMMARY  ---------------------------------
 PRECISION     FLOP/cycle         FLOP/second
    Single              0                   0
    Double           32.0            6.92e+10
-----------------------------------------------------------------------
\end{lstlisting}
\end{minipage}
        Le benchmark du \verb=Kernel Generator=, ne possède pas de version multicoeurs et ne présente que les résultats de la performance d'un coeur. Pour nous assurer de l'exécution du code sur les différents coeurs, nous avons utilisé un outil développé chez HPE appelé \textit{mygflops.sh}. Il permet de compter les instructions flottantes simples et doubles précisions exécutées sur un processeur. Le résultat est présenté dans l'\autoref{code:mygflops_512_output}. La performance maximale mesurée du processeur \gls{flopsmax} est de 1372.78 GFLOPS, proche du maximum théorique $\text{FLOPS}_{peak}$ de 1408 GFLOPS, calculé lors de l'étape 1  (voir \autoref{sec:cal_teo_intel_etape1}).
        
\begin{lstlisting}[caption=Résultat de l'outil \textit{myflops.sh} utilisé pour compter les instructions flottantes exécutées sur un processeur., label={code:mygflops_512_output},
  basicstyle=\footnotesize, frame=tb,
  xleftmargin=.005\textwidth, xrightmargin=.005\textwidth]
Single-precision SSE/AVX :       0.00 GFlop/s  --   0.0% of Flops
Double-precision SSE/AVX :     (*\bfseries 1372.78 GFlop/s*)  -- 100.0% of Flops
   0.0% scalar  64-bit SSE/AVX instructions (  0.0% of fp instructions)
   0.0% packed 128-bit SSE/AVX instructions (  0.0% of fp instructions)
   0.0% packed 256-bit AVX instructions     (  0.0% of fp instructions)
 100.0% packed 512-bit AVX instructions     (100.0% of fp instructions)
\end{lstlisting}
%%%%%%%%%%%%%%%%%%%%%%%%%%%%%%%%%%%%%%%%%%%%%%%%%%%%%%%%%%%%%%%%%%%%%%%%%%%%%%%%%%%%%%%%
%%%%%%%%%%%%%%%%%%%%%%%%%%%%%%%%%%%%%%%%%%%%%%%%%%%%%%%%%%%%%%%%%%%%%%%%%%%%%%%%%%%%%%%%
%%_______ .___________.    ___      .______    _______                      ___       %%
%%|   ____||           |   /   \     |   _  \  |   ____|                   |___ \     %%
%%|  |__   `---|  |----`  /  ^  \    |  |_)  | |  |__          ______        __) |    %%
%%|   __|      |  |      /  /_\  \   |   ___/  |   __|        |______|      |__ <     %%
%%|  |____     |  |     /  _____  \  |  |      |  |____                     ___) |    %%
%%|_______|    |__|    /__/     \__\ | _|      |_______|                   |____/     %%
%%%%%%%%%%%%%%%%%%%%%%%%%%%%%%%%%%%%%%%%%%%%%%%%%%%%%%%%%%%%%%%%%%%%%%%%%%%%%%%%%%%%%%%%
%%%%%%%%%%%%%%%%%%%%%%%%%%%%%%%%%%%%%%%%%%%%%%%%%%%%%%%%%%%%%%%%%%%%%%%%%%%%%%%%%%%%%%%%


%%%%%%%%%%%%%%%%%%%%%%%%%%%%%%%%%%%%%%%%%%%%%%%%%%%%%%%%%%%%%%%%%%%
\section{Étape 3: Extraction de kernels et modélisation de leur performance} \label{sec:methodo_step3}
%%%%%%%%%%%%%%%%%%%%%%%%%%%%%%%%%%%%%%%%%%%%%%%%%%%%%%%%%%%%%%%%%%%


%%%%%%%%%%%%%%%%%
\subsection{Motivations et objectifs}
%%%%%%%%%%%%%%%%%



Les applications que nous ciblons dans notre analyse sont des codes dont la majorité du temps d'exécution se déroule seulement dans quelques pourcentages des lignes de codes. Nous appelons ces zones des points chauds, \textit{hot spots} ou encore \textit{kernels}. Une application possédant des hot spots est un gage d'un potentiel d'amélioration des performances. Les applications réelles utilisées en production dépassent souvent les dizaines de milliers de lignes de codes. Porter et optimiser la totalité d'une application serait complexe et contre-productif. De plus, le portage et l'optimisation des codes est une tâche difficile, même aujourd'hui alors que le nombre d'accélérateurs différents disponibles est réduit. Ensuite, pour optimiser la phase de transformation du code, nous modélisons la performances de ces kernels pour éliminer les plate-formes inadaptées. Ce nombre peut encore être réduite grâce au caractéristiques récupérées lors de l'étape 2.  


L'objectif de cette étape est d'identifier ces zones du code clés et de modéliser leur performances en fonction des performances de la bande passante mémoire (GB/s) ou du processeur (FLOP/s) en calculant leur intensité opérationnelle $\text{OI}_{kernel}$. La majorité des codes étant limité par la performance de la mémoire, la thèse présente un modèle de performance basé sur ses performances. 


%%%%%%%%%%%%%%%%%
\subsection{Identification des kernels}
%%%%%%%%%%%%%%%%%

Si une application passe 99\% de ses cycles dans l'exécution d'une fonction, une amélioration d'un facteur 10 de celle-ci entraînera une amélioration du même facteur de l'application. L'identification des ces fonctions est donc primordiale. 

De nombreux travaux sont réalisés pour identifier et extraire les \textit{hot spots} d'une application \cite{castro2015cere, brunst2013custom}. L'outil de profilage \textit{perf} \cite{de2010new} permet d'extraire un sommaire de l'exécution d'une application en représentant son arbre d'appel (voir \autoref{perf_example}). 



\begin{lstlisting}[caption=Exemple d'utilisation de l'out perf avec la commande \textit{perf record  -g  -F 97}. Le rapport d'exécution est obtenu avec la commande \textit{perf report --stdio}, float,floatplacement=H, label={perf_example}]
# Samples: 116K of event 'cycles:ppp'
# Event count (approx.): 2744862582690
#
# Children      Self  Command          Shared Object       Symbol                                                                       
# ........  ........  ...............  .................. .......
    99.52%     0.00%  Stream.SKL.128   [unknown]           [k] 0000000000000000
            |          
             --99.52%--0
                       |          
                        --99.16%--0xadf96
                                  |
                                  |--25.35%--tuned_STREAM_Add   
                                  |--25.34%--tuned_STREAM_Triad
                                  |--20.36%--tuned_STREAM_Copy
                                  |--17.27%--tuned_STREAM_Scale
                                   --10.74%--main

\end{lstlisting}



%%%%%%%%%%%%%%%%%
\subsection{Modélisation de l'équilibre des kernels}
%%%%%%%%%%%%%%%%%

Chaque kernel peut être porté sur un accélérateur différent, et leur analyse doit se faire indépendamment les uns des autres. L'objectif de la modélisation est de comprendre les performances de l'application: si elles limitées par les performances du sous système mémoire ou par la capacité de calcul du processeur. La modélisation des performances permet de réduire le nombre de plate-forme envisagées pour le portage de l'application. Cette étape permet d'éviter d'investir du temps et de l'argent dans des solutions inefficace pour l'application étudiée. La modélisation du kernel nécessite de lire son code pour compter les différents accès mémoire et les opérations réalisées.

La majorité des codes HPC exécutées sur des architectures modernes voient leurs performances limitées par celle de la bande passante mémoire. S'il est très rare de voir un code limité par la puissance de calcul, il peut être nécessaire de faire cette vérification. 

Pour cela, le \textit{Roof Line Model}, permet de réaliser cette représentation des performances. Il est présenté dans la section \ref{sec:roofline}. 

Il est nécessaire de calculer l'intensité opérationnelle, $\text{OI}_{kernel}$, mesurée en $flop/byte$. Elle représente le nombre d'opération réalisable par le processeur pour chaque donnée transférée depuis la mémoire. Ce calcul se fait à partir de la lecture de code source, motivant le désire des programmeurs d'identifier individuellement les kernels de calculs. 


%%%%%%%%%%%%%%%%%
\subsection{Simple Memory Model: Modélisation de la performance mémoire} \label{sec:smm}
%%%%%%%%%%%%%%%%%

La majorité des codes étant limité par la performance du système mémoire, nous avons développé un modèle de performance simple, permettant de modéliser et valider les performances d'un code facilement. Pour réaliser cette modélisation le développeur doit avoir accès au code source de l'application à porter. Pour un hot spot donné, il faut compter le nombre d'accès mémoire en distinguant les accès en lecture et ceux en écriture. Il est important de distinguer les accès en lecture et en écriture car nous utiliserons leur ratio pour valider le bon comportement de la micro-architecture avec l'outil YAMB. En effet, nous montrons dans notre expérience que la saturation du bus mémoire n'est pas un indicateur suffisant pour conclure de l'efficacité ou non d'un code.

Cette modélisation est faisable seulement si les kernels du code ont été identifiés, l'appliquer sur la totalité de l'application serait trop long. Si la taille des jeux de données $\text{DATA}_{size}$ est connue, il est alors possible de calculer la quantité de donnée minimale qui doit être transférée sur le bus mémoire. Grâce à l'étape 2, nous connaissons les performances maximales théorique $\text{MEMORY}_{peak}$ et réelle $\text{MEMORY}_{max}$ de la micro-architecture. Il est donc possible de calculer la durée optimale pour exécuter fonction étudiée, $\text{TEMPS}_{optimal}$, mesurée en seconde. Le modèle assume que le code utilise un algorithme parfait (utilisation de la localité des données), que sa compilation du code à été réalisée avec un compilateur parfait et qu'il est exécuté sur une plate-forme parfaite. L'objectif n'est pas d'atteindre exactement cette performance, mais de s'en approcher le plus possible. Généralement, lorsque qu'un défaut apparaît à un des niveaux énuméré précédemment, la performance s'éloigne radicalement de la performance optimale.

\begin{equation}
    \text{TEMPS}_{optimal}\ = \frac{\text{DATA}_{size}}{\text{MEMORY}_{max}}
\end{equation}





%%%%%%%%%%%%%%%%%
\subsection{Application des modèles roofline et SMM au benchmark Stream}
%%%%%%%%%%%%%%%%%

L'\autoref{perf_example} montre le profil de l'exécution du benchmark \textit{Stream}. Il comporte quatre fonctions utilisées pour stresser la mémoire par différent type d'accès. Nous choisissons arbitrairement de consacrer notre analyse sur un des quatre \textit{hot spot} de Stream: la fonction \textit{triad} dont le code peut être vu dans l'\autoref{code:triad}. Cette fonction est intéressante car ce motif d'accès est très courant dans les applications HPC (algorithmes RTM).

\subsubsection{Modèle du roofline}
%%%%%%%%%%%%%%%%%
La première étape est de modéliser l'équilibre entre processeur et mémoire des besoins de l'application. 

\begin{lstlisting}[language=c,caption= La fonction triad du benchmark Stream utilise trois matrices: deux en lecture et une en écriture,label={code:triad}, 
  basicstyle=\footnotesize, frame=tb,
  xleftmargin=.065\textwidth, xrightmargin=.065\textwidth]
for (j=0; j < STREAM_ARRAY_SIZE; j++)
    A[j] = B[j] + scalar * C[j];
\end{lstlisting}


Concernant les données, il faut, à chaque itération, charger 3 éléments en double précision, soit 24 bytes. En effet, or optimisation, une ligne de cache doit être chargée avant d'être écrite, même si aucune des données n'est utilisées en lecture par le processeur. Cette fonction a donc une intensité arithmétique $\text{OI}_{kernel} = \frac{2}{24} = 0.083\ flop/byte$.
Pour comparaison, les processeurs récents ont un ratio proche de $10\ flop/byte$. Cette simple modélisation montre le déséquilibre qu'il y a entre la performance de la mémoire et celle des processeurs. Elle permet de guider le choix de la plate-forme sur laquelle cette fonction devra être portée. La \autoref{pic:roofline_stream} montre l'application du roofline à l'étude de la fonction triadd et au processeur étudié. Cette fonction ayant un faible $\text{OI}_{kernel}$, ses performances théorique mesurée en flop sont elles aussi très faible.

\begin{figure}
    \center
    \includegraphics[width=10cm]{images/roofline_stream.png}
    \caption{\label{pic:roofline_stream} Modèle du \textit{Roofline} appliqué à la fonction \textit{triadd} du benchmark \textit{Stream} et un processeur Xeon Skylake 6148.}
\end{figure}



\subsubsection{Simple Memory Model} 
%%%%%%%%%%%%%%%%%
L'analyse de la fonction avec le modèle du \textit{roof line} indique que sur l'architecture ciblée, la performance du code sera limité par la performance de la bande passante.
A chaque itération de boucle, deux opérations doivent être réalisées, une addition et une multiplication. 

Une fois assuré que les performances de l'application sont limitées par le système mémoire, le Simple Memory Model peut être appliqué. Dans notre éxpérimentation, nous utilisons trois matrices de 19.6 GB ($3 \times 10^9 \times sizeof(double)$). Pour une exécution optimale, le bus mémoire devrait être utilisé pour charger une fois les deux matrices en lectures (matrices B et C)  et pour l'écriture de la matrice A. Le traffic mémoire total serait alors de 58.8 GB. En utilisant les résultats mesurés lors de l'étape 2, on peut estimer le temps optimale pour l'exécution de cette fonction: $\text{TEMPS}_{optimal} = \frac{58.8}{128} = 0.56$ seconde.
%%%%%%%%%%%%%%%%%%%%%%%%%%%%%%%%%%%%%%%%%%%%%%%%%%%%%%%%%%%%%%%%%%%
\section{Étape 4: Sélectionner la plateforme adaptée} \label{sec:methodo_step4}
%%%%%%%%%%%%%%%%%%%%%%%%%%%%%%%%%%%%%%%%%%%%%%%%%%%%%%%%%%%%%%%%%%%


    Les deux premières étapes ont permis de trouver et de caractériser de potentielles architectures. À l'aide de plusieurs benchmarks, certaines caractéristiques clés des architectures ont pu être obtenues. Lors de l'étape 3, l'extraction et la modélisation des \glspl{hotspot} ont été faites.  Cette quatrième étape consiste à réaliser le choix de l'architecture la plus adaptée pour l'exécution de chaque hot spot. Ce choix est réalisé en prenant en considération la performance des architectures, les besoins de l'application étudiée ainsi que d'autres critères (voir \autoref{pic:methodo_step4}).
    
    \begin{figure}
        \center
        \includegraphics[width=14cm]{images/methodo_step4.png}
        \caption{\label{pic:methodo_step4}L'étape 4 consiste à choisir les plateformes adaptées aux différents noyaux en fonction de plusieurs critères.}
    \end{figure}



%%%%%%%%%%%%%%%%%
\subsection{Le coût}
%%%%%%%%%%%%%%%%%
    
   Le prix des architectures est un élément important pour choisir la ou les architectures utilisées pour la construction de la plateforme. Cependant, utiliser le prix unitaire des accélérateurs choisis n'est pas suffisant et il est nécessaire de prendre en compte d'autres paramètres dans le calcul du prix appelé coût total de possession (TCO). Il prend en compte tous les coûts engendrés par le centre de données durant son cycle de vie: construction des bâtiments, consommation électrique, etc.
   %Le budget pour la construction du premier centre exascale européen est de 500 millions d'euros \cite{SergiGirona2018}.
    
    \paragraph{Le prix du matériel} 
        Le  prix des accélérateurs et du matériel nécessaires à la construction du supercalculateur constituent une part significative dans le calcul du coût. En fonction du prix de l'accélérateur, une solution moins performante pourra lui être préférée. Les cartes FPGA sont un bon exemple d'architectures très performantes, mais peu utilisées dans les supercalculateurs. Bien que la complexité de programmation y participe, le prix des cartes FPGA est une raison majeure de leur faible utilisation dans les plateformes modernes.
    
    \paragraph{La consommation électrique}
        La consommation électrique de la plateforme finale est devenue un critère très important dans le choix du matériel. La consommation des centres de données est un réel investissement qui doit être mesuré et anticipé lors de l'achat du matériel. Un supercalculateur consommant 10 MW engendrera une facture de plusieurs millions d'euros chaque année. Le prix de l'électricité et l'enveloppe énergétique disponible pour le calcul varient avec l'emplacement choisi pour installer le centre de calcul. La majorité des centres ayant des lignes électriques déjà construites ne peuvent acheminer qu'une quantité limitée de courant. L'enveloppe énergétique disponible est alors une contrainte forte pouvant favoriser l'utilisation d'une architecture plus efficace énergétiquement.
        
        En comparant  l'intensité opérationnelle d'un kernel (\gls{oikernel}) et l'équilibre arithmétique de l'architecture (\gls{equilibrearchi}) il est possible d'estimer la pertinence d'un accélérateur pour un noyau. Des valeurs proches indiquent que l'architecture ciblée est adaptée au code étudié et que l'accélérateur choisi aura un meilleur rendement énergétique. Un code faisant peu de calculs flottants ne nécessitera pas l'utilisation de coeurs complexes réalisant plusieurs dizaines de \gls{FLOP} par cycle. Bien que non utilisés, ces composants impactent le prix de la solution, mais aussi sa consommation électrique. 
        
                %La chaleur de la zone géographique de son installation impactant le refroidissement nécessaire. Ainsi en 2018, Microsoft a décidé de couler ses serveurs au fond de l'océan pour profiter d'un refroidissement gratuit \cite{ChristineHall2018}.
            
    \paragraph{La taille du centre de données} 
        Souvent les centres de données sont déjà existants et la taille disponible pour la création d'un supercalculateur ou l'ajout de nouveaux serveurs est une contrainte forte. Certaines solutions prenant plus ou moins de place pour être installées, le ratio $\frac{flop}{m^2}$ peut alors être calculé pour évaluer la densité des serveurs pour s'adapter aux contraintes du lieu. Si le bâtiment doit être construit pour accueillir le supercalculateur, son coût doit entrer dans le calcul de la solution finale.




%%%%%%%%%%%%%%%%%
\subsection{La performance}
%%%%%%%%%%%%%%% \cite{Rodero2012}%%
    D'autres critères concernant la performance et l'optimisation du code doivent être pris en compte pour choisir la ou les architectures utilisées pour accélérer l'exécution de l'application.
    
    \paragraph{Performance du noyau}
        Bien sûr, le gain de performance obtenu après le portage d'un \gls{kernel} sur un accélérateur est un critère important.
        La performance du noyau a un impact financier, car s'il est exécuté plus rapidement, d'autres applications pourront alors accéder à la plateforme. Les clients de supercalculateurs ont généralement un budget fixe et de multiples applications à exécuter. Ils cherchent alors à optimiser l'utilisation des ressources informatiques par les différents codes. Bien que l'on considère qu'il est nécessaire de porter individuellement chaque noyau sur l'accélérateur le plus adapté, il faut aussi prendre en considération le reste de l'application, mais aussi les applications des autres utilisateurs. Si une seule des applications ne bénéficie pas des performances d'un accélérateur, il serait plus pertinent d'opter pour un accélérateur adapté à plusieurs applications exécutées sur la plateforme. 
    
     
    \paragraph{Difficulté du portage}
        Les transformations de codes nécessaires pour porter le code d'un kernel sur une architecture doivent être évaluées. Celles-ci peuvent avoir un impact sur les performances finales (incapacité à réaliser les optimisations nécessaires) ou sur le coût (recours à des programmeurs expérimentés).  Suivant l'architecture choisie, il faudra peut-être coder l'application avec un nouveau langage, utiliser de nouvelles librairies ou de nouveaux modèles de programmation. Lorsque l'implémentation d'une optimisation est décidée, le risque de ne pas parvenir à obtenir les performances espérées doit lui aussi être mesuré. Le temps optimal, noté \gls{tempsoptimal}, pour exécuter l'application considère que l'application utilise de façon optimale l'architecture. Cependant pour atteindre ces performances, le noyau peut nécessiter l’utilisation d'optimisations complexes, pouvant être difficile à implémenter sur des applications industrielles. Il peut arriver qu'une optimisation moins performante soit préférée, car la transformation du code est plus facile. Le temps et le nombre de programmeurs nécessaires à son implémentation entrent alors aussi en considération dans le prix de la solution.
    
                           
%%%%%%%%%%%%%%%%%%%%%%%%%%%%%%%%%%%%%%%%%%%%%%%%%%%%%%%%%%%%%%%%%%%
\section{Étape 5: Porter et optimiser le code} \label{sec:methodo_step5}
%%%%%%%%%%%%%%%%%%%%%%%%%%%%%%%%%%%%%%%%%%%%%%%%%%%%%%%%%%%%%%%%%%%

L'étape 4 a permis de choisir une ou plusieurs architectures pour porter les différents \glspl{kernel} d'une application. L'étape 5 consiste à apporter les transformations au code pour porter chaque kernel sur une architecture et à valider les performances atteintes.


%%%%%%%%%%%%%%%%%
\subsection{Motivations}
%%%%%%%%%%%%%%%%%
    
    Suite au choix de l'architecture lors de l'étape 4, le travail de portage doit être réalisé. L'effort nécessaire pour adapter le code à cette nouvelle architecture a dû être évalué lors de l'étape précédente et peut impliquer certaines difficultés:
    \begin{itemize}
        \item Utiliser un nouveau langage de programmation ou un nouveau paradigme de programmation.
    
        \item Vérifier que les performances obtenues après le portage sont celles attendues par les prédictions. 
    
        \item Avoir à sa disposition des outils compatibles avec l'architecture pour réaliser l'analyse et la validation des performances.
    \end{itemize}
    Dans la suite de cette section, nous montrons comment les outils développés durant ces travaux de thèse sont utilisés pour valider la performance du kernel établie lors de l'étape 3.


    
    \paragraph{Développement d'une première version.} 
    
        Afin de s'assurer que les premières versions du code atteignent des performances acceptables, le programmeur doit rechercher les librairies existantes pouvant être utilisées. En effet, les constructeurs de l'accélérateur peuvent avoir développé des librairies pouvant être quasi-optimales et qui nécessiteraient une grande expertise pour être développées. Pour atteindre le maximum de performance, le code doit profiter au maximum de la parallélisation. Pour cela, le maximum de coeurs doit être utilisé, grâce aux paradigmes de programmation à mémoire partagée et mémoire distribuée. Les processeurs étant généralement superscalaires, voir \autoref{sec:superscalar}, il faut essayer d'exécuter le maximum d'instructions à chaque cycle. Le dernier niveau de parallélisme est apporté par l'utilisation d'instructions vectorielles. Lorsque c'est possible, des instructions telles que les \gls{FMA} doivent être utilisées. Les différentes pistes listées peuvent nécessiter l'utilisation d'un simple drapeau lors de la compilation ou au contraire nécessiter de lourdes modifications du code et des jeux de données.
    
    
\subsection{Analyse de performance}
%%%%%%%%%%%%%%%%%

    Lors de l'étape 3, un modèle de performance du noyau a été établi (voir \autoref{sec:smm}). Une fois celui-ci porté sur la nouvelle architecture, il convient de vérifier que les performances obtenues sont proches de celles attendues par le modèle développé. La \autoref{pic:methodo_step5} propose un cheminement à suivre pour la validation et l'optimisation des performances.

\begin{figure}[h!]
    \center
    \includegraphics[width=14cm]{images/methodo_step5.png}
    \caption{\label{pic:methodo_step5}Les différentes étapes à suivre pour valider la performance du noyau.}
\end{figure}




    \subsubsection{Vérification du modèle de performance}
    %%%%%%%%%%%%%%%%%
    
        La première étape consiste à vérifier que la performance atteinte par le code est proche de celle calculée par notre modèle SMM. Dans le cas contraire, cela permet de quantifier l'écart par rapport à l'optimum théorique (\gls{tempsoptimal}). 
        
        Pour mesurer le temps d'exécution du noyau, noté \texttt{TEMPS}$_\texttt{mesure}$, nous proposons d'utiliser la fonction \verb|gettimeofday ()| \cite{Linux}. Cette fonction est disponible sur la totalité des systèmes Linux et permet de récupérer l'heure actuelle avec une précision allant jusqu'à la microseconde. Le but de la méthodologie présentée est de porter les codes sur de nouvelles architectures. Baser l'analyse de performance sur des compteurs matériels trop complexes aurait réduit la portabilité de notre démarche. Il est alors possible de mesurer  \texttt{TEMPS}$_\texttt{mesure}$ en plaçant deux appels à la fonction \verb|gettime()| (avant et après le noyau) présentée dans l'\autoref{lst:gettime}.
        
\begin{minipage}[]{\linewidth}%
         \begin{lstlisting}[language=c,caption=Fonction utilisée pour obtenir l'heure actuelle avec une précision allant jusqu'à la microseconde,label={lst:gettime}, 
          basicstyle=\footnotesize, frame=tb,
          xleftmargin=.065\textwidth, xrightmargin=.065\textwidth]
        double gettime()
        {
          struct timeval tp;
          struct timezone tzp;
          int i;
          i = gettimeofday(&tp,&tzp);
          return ( (double) tp.tv_sec + (double) tp.tv_usec * 1.e-6 );
        }
        \end{lstlisting}
\end{minipage}

        Si \texttt{TEMPS}$_\texttt{mesure}$ est proche de \gls{tempsoptimal}, alors le noyau est proche d'avoir des performances optimales. Dans le cas contraire, l'analyse de performance doit se poursuivre pour définir la cause de cette mauvaise performance. La mesure \gls{tempsoptimal} permet alors d'avoir un objectif de performances à atteindre et de savoir quand le travail d'optimisation est terminé.
        
    
    \subsubsection{Profilage des performances}
    %%%%%%%%%%%%%%%%%

        Lorsque la mesure de \texttt{TEMPS}$_\texttt{mesure}$ est inférieure à l'optimal \gls{tempsoptimal}, le programmeur doit réaliser les modifications appropriées pour améliorer la performance du \gls{kernel} étudié. Il doit pour cela avoir à sa disposition des outils lui permettant de mener à bien cette l'analyse. Utilisés de façon méthodique, les deux outils présentés dans cette thèse, \verb|OProfile++| et \verb|YAMB|, permettent de répondre à beaucoup de questions et mener à bien le travail d'optimisation du code.
        Même si l'analyse par le modèle du \textit{Roofline} a montré que la performance maximale atteignable était limitée par le débit mémoire, d'autres facteurs peuvent affecter les performances (mauvaise compilation, dépendances de données, mauvaise utilisation de la localité). Le programmeur doit alors suivre une démarche logique lui permettant d'identifier la raison de cette mauvaise performance (voir \autoref{pic:analyse_bigpicture}).
        
        \begin{figure}
            \center
            \includegraphics[width=12cm]{images/analyse_bigpicture.png}
            \caption{\label{pic:analyse_bigpicture} Flux de travail pour l'utilisation des outils \texttt{OProfile++} et \texttt{YAMB}. En fonction des observations amenées par les outils, des réponses différentes sont proposées au programmeur pour optimiser son code.}
        \end{figure}

        La première mesure à réaliser est celle du nombre d'\gls{IPC} des boucles critiques du kernel. Pour cela, nous avons présenté l'outil \verb=Oprofile++= dans la \autoref{sec:oprofile} (voir \autoref{lst:oprofileex}). Si l'outil indique un IPC faible, il y a de fortes chances que le système mémoire soit responsable de la mauvaise performance. Un IPC élevé indique quant à lui que le problème vient généralement du processeur.

\begin{minipage}[]{\linewidth}%
        \begin{lstlisting}[language={},caption={L'outil \texttt{OProfile++} permet d'extraire le code assembleur et d'y associer le compteur de cycles},label={lst:oprofileex}, 
  frame=tb]
================================================================
Analysis from the app name (horner1_long) hot spot from the symbole name (f1(double)) which takes 50.1878% 
================================================================
CYCLES       INSTS      ADDRESS    disassembly
----------------------------------------------------------------
     5           11      401be0    vmovupd 0xd1b8(%rip),%ymm1
    39           79      401bf0    vfnmadd231pd 0xd1c7(%rip),%ymm11,%ymm1
     3           14      401bf9    vfmadd213pd %ymm2,%ymm11,%ymm1
...
   760         1652      401c3e    cmp    %eax,%edx
    48           99      401c40    jb     401be0 <_Z8range_f1ddi+0xa0>
----------------------------------------------------------------
LOOP from 401c40 to 401be0  size= 96 sum(cycles)= 2287 
     IPC= 2.12462 cycles/LOOP= 10.3548 flop/cycle = 1.42 
----------------------------------------------------------------
\end{lstlisting}
\end{minipage}
        
        \subsubsection{Profilage de la mémoire}
        %%%%%%%%%%%%%%%%%
            Lorsque la première analyse indique que la mauvaise performance du noyau est due au système mémoire, il est nécessaire de réaliser l'analyse de son activité. La première vérification à faire est de s'assurer que le bus mémoire est saturé. Pour y répondre, nous proposons d'utiliser l'outil \verb|YAMB| (voir \autoref{sec:yamb}) qui affiche l'activité du bus mémoire (lecture, écriture et utilisation totale). 
            
            Pour confirmer sa saturation, il est nécessaire d'avoir caractérisé la performance du bus mémoire lors de la première étape et d'en connaître la performance maximale. De plus, il est important de posséder une fréquence d'échantillonnage suffisamment élevée. En effet, avec une fréquence trop faible, il peut arriver que le graphique montre une saturation avec une ligne droite atteignant la performance maximale. En augmentant la fréquence et en grossissant certaines parties du graphique, il est possible de voir que le bus n'est pas totalement utilisé pendant des périodes très courtes, et saturé quelques cycles après. Ce phénomène peut être dû à des dépendances entre plusieurs instructions. Des techniques de déroulement de boucles sont alors très efficaces si la nature de l'application le permet (voir \autoref{sec:dml_unroll}). Un autre facteur peut venir de la mauvaise gestion du préchargement des données. Il peut alors être intéressant de réaliser le préchargement manuellement, en démarrant les chargements de données plusieurs instructions avant qu'elles soient utilisées. D'autres techniques, telles que la fusion de différentes boucles (ou noyau) nous ont permis d'améliorer la performance de plusieurs applications.
            
            Si le bus est bien saturé, nous utilisons le modèle SMM pour comparer les ratios de lecture/écriture avec celui calculé dans l'étape 1 (voir \autoref{sec:smm}). Un mauvais ratio peut provenir d'un nombre de lectures plus élevé que prévu. En effet, il est très rare de réaliser plus d'écriture que nécessaire. Ceci est causé par un large éventail de problèmes potentiels, par exemple, des conflits dans le cache, la lecture de données inutiles, une décomposition incorrecte des données ou des structures de mémoire inappropriées. Pour y remédier, des techniques de \textit{blocking} peuvent être utilisées pour améliorer l'utilisation de la localité des données.  L'outil \verb|Oprofile++| peut alors aider à trouver si des groupes d'instructions sont plus longs à exécuter que d'autres, pouvant être expliqués par une dépendance entre les données. 
        
        
        
        \subsubsection{Profilage du processeur}
        %%%%%%%%%%%%%%%%%
        
            Si la première analyse montrait que la mauvaise performance n'était pas due au système mémoire, il faut alors affiner l'analyse de l'exécution du code donnée par \verb=Oprofile++= (voir \autoref{lst:oprofileex}).
            La première vérification à faire est de compter le nombre d'opérations flottantes réalisées par la boucle. Pour cela, nous comptons manuellement le nombre d'opérations flottantes à partir des instructions assembleurs utilisées. Si le nombre d'opérations flottantes et l'\gls{IPC} sont élevés, cela indique que le code est optimal. 
            
            Un nombre d'opérations flottantes faible associé à un IPC élevé indique que la parallélisation n'est pas correctement utilisée. Il est alors conseillé de regarder le type d'instructions exécutées.
                Un compilateur de mauvaise qualité aura tendance à générer de nombreuses instructions supplémentaires simples à exécuter. Cela fait augmenter l'IPC de la boucle sans en améliorer la performance. Un mauvais compilateur aura tendance à générer plus d'instructions qu'un bon compilateur pour le calcul d'adressage par exemple. La performance du code peut alors être limitée par le matériel responsable du calcul d'adresses et non par la \gls{FPU}.
                Un autre exemple couramment rencontré est l'exécution d'instructions de la librairie \gls{MPI} utilisée pour synchroniser les processus. Celles-ci sont exécutées à chaque cycle pour vérifier l'état des autres processus, faisant augmenter l'IPC de la boucle. Il faut alors utiliser des outils tels que \textit{Extrae} \cite{Rodriguez}, \textit{Paraver} ou Vampire \cite{Nagel1996, brunst2013custom} pour vérifier que la répartition du travail est bien réalisée. Un noeud de calcul ayant un matériel défectueux affecte alors les autres serveurs, et de nombreux cycles sont perdus dans ces instructions de synchronisation. Les pannes sont très difficiles à identifier sans les outils adaptés, car ils peuvent venir d'une multitude d'endroits: barrette mémoire défectueuse, surchauffe d'un processeur ou d'un disque cassé.
                Si la boucle contient beaucoup d'instructions de calcul à virgule flottante, il faut vérifier qu'il s'agit d'instructions vectorielles les plus larges possible.
            
            Il est donc important de ne pas baser son analyse seulement sur la lecture de l'IPC, mais de vérifier que les instructions exécutées sont des instructions de calculs flottants. Les transformations du code nécessaires pour y parvenir peuvent être difficiles à réaliser et demander l'utilisation d'un autre algorithme ou de restructurer le jeu de données.
            


%%%%%%%%%%%%%%%%%
\subsection{Application au benchmark STREAM}
%%%%%%%%%%%%%%%%%

    Nous appliquons notre méthodologie sur le benchmark \verb|STREAM| exécuté sur un processeur Intel Skylake. Cet exercice nous permet de montrer que même sur un code aussi simple, apparemment optimal, notre analyse nous permet de comprendre sa performance et de l'optimiser.
    
        
    \subsubsection{Vérification du modèle SMM}
    %%%%%%%%%%%%%%%%%
    
        Nous avons montré précédemment que la performance du kernel était limitée par la bande passante mémoire. Nous avons appliqué notre modèle de performance SMM et calculé \gls{tempsoptimal} égale à $\frac{58.8}{128} = 0.56$ seconde. Suite à l'exécution du noyau étudié (kernel \textit{triad}), et à l'utilisation de la fonction présentée dans l'\autoref{lst:gettime}, nous mesurons $\text{TEMPS}_{mesure} = 0.79$ seconde.
        L'application n'est donc pas optimale. Ceci est également validé par le résultat du benchmark \verb|STREAM| qui annonce une bande passante mémoire de 80,13 GB/sec, inférieure à la performance du bus mémoire mesurée à l'aide de \verb=DML_MEM=. Dans l'étape 2, nous avions mesuré un débit maximal $\text{MEMORY}_{max}$ de 105 GB/s.
        
        Pour comprendre ce résultat, nous avons utilisé \verb|YAMB| pour vérifier que le bus mémoire était bien saturé. Le graphique de la \autoref{fig:stream_before} nous montre que le bus mémoire est utilisé au maximum de son potentiel de 104 GB/s. La saturation du bus n'est donc pas la cause de la mauvaise performance du noyau. Il est alors nécessaire de regarder les rapports lecture/écriture. Lors de l'étape 3, nous avions calculé un ratio optimal pour ce noyau  d'une écriture pour deux lectures. La \autoref{fig:stream_before} montre la répartition mesurée lors de l'exécution: 26 GB/s en écriture pour 78 GB/s en lecture, pour un total de 104 GB/s, ce qui correspond à un ratio de 1 écriture pour 3 lectures. Ce ratio exact d'une écriture pour trois lectures n'est pas une coïncidence. Un tableau complet est lu alors qu'il ne devrait pas l'être.
        
        \begin{figure}[htb]
        {
        \centering
        \includegraphics[width=0.80\textwidth]{images/stream_before.png}
        \caption{Profil mémoire donné par l'outil \texttt{YAMB} pour plusieurs exécutions de noyau \textit{triadd} du benchmark \texttt{STREAM}. }\label{fig:stream_before}
        }
        \end{figure}
        \textbf{todo trop petit}


        
        \subsubsection{Optimisation du noyau \textit{triadd}}
        %%%%%%%%%%%%%%%%%
            
            Les deux tableaux B et C doivent nécessairement être lus une fois. La lecture supplémentaire provient du tableau en écriture. Ce comportement est dû au processeur qui, avant d'écrire une donnée, charge la ligne de cache correspondante pour la mettre à jour. Cependant, le noyau étudié a la particularité d'écrire toute la ligne de cache. Il n'est donc pas nécessaire de charger les données initialement présentes. Une option peut être utilisée avec le compilateur Intel (ICC) pour permettre au compilateur d'éviter ce chargement inutile: \verb|-qopt- streaming-stores=always|. Cette option permet au CPU d'écrire toute la ligne de cache sans avoir à la charger.
            
        
        \subsubsection{Performance du noyau optimisé}
        %%%%%%%%%%%%%%%%%
        
            Nous avons compilé \verb|STREAM| avec l'option \verb|-qopt- streaming-stores=always| et mesuré le temps et la bande passante de la même manière que pour la première exécution. Les résultats sont résumés dans le \autoref{table:stream_res}. Le temps passé dans le noyau est de 0,59 seconde, beaucoup plus proche du maximum théorique de 0,56 seconde. En analysant la bande passante mémoire, on constate que le rapport lecture/écriture est maintenant de 2 pour 1, les données échangées étant de 34 GB/s en mode écriture pour 68 GB/s en mode lecture. La bande passante totale atteint 104 GB/s proche de $\text{MEMORY}_{max}$.
            
            %\renewcommand{\arraystretch}{1.2}
            %\setlength{\tabcolsep}{8pt}
            
            \begin{table}[htbp]
            \centering
                        \begin{tabular}{l|c|c|c|c|c|}
            \cline{2-6}
            & $\text{TEMPS}_{mesure} (s)$ & Stream   & $\text{YAMB}_{read}$  & $\text{YAMB}_{write}$  & $\text{YAMB}_{total}$  \\ \hline
            \multicolumn{1}{|l|}{Version originale}  & 0.79   & 80.13  & 26        & 78         & 104        \\ \hline
            \multicolumn{1}{|l|}{Version optimisée} & 0.59   & 101.84 & 35        & 69         & 104        \\ \hline
            \end{tabular}
            \caption{Performance du noyau \textit{triadd} du benchmark \texttt{STREAM} avant et après optimisation. Le temps (seconde) après optimisation est proche de l'optimum théorique \gls{tempsoptimal}  calculé à 0,56 sec. L'utilisation du bus mémoire est la même entre les deux versions du code. L'option de compilation optimise le rapport lecture/écriture (GB/s) et améliore les performances du code de 25\%.}
            \label{table:stream_res}
            \end{table}
    

\subsection{Travaux liés}
\url{https://tel.archives-ouvertes.fr/tel-00984791v4/document}


\subsection{Conclusion}
%%%%%%%%%%%%%%%%%

\textbf{TODO}
\begin{itemize}
    \item Through this experiment we wanted to show that having a solid analytic methodology is essential to understand the performance of a code
    \item Only measuring the memory throughput is not enough to conclude that the code is optimal. 
\end{itemize}


\printbibliography[heading=references,segment=\therefsegment]
