\section{Introduction}\label{sec:methodo_intro}

À cause des pressions énergétique et économique, les plateformes de calculs haute performance doivent être repensées. De nouvelles architectures optimisées pour certaines applications devront être utilisées rendues possibles grâce au protocole Gen-Z. En l'absence de méthode de caractérisation fine de la performance des codes, ces architectures inovantes sont potentiellement condamnées puisque peu d'experts savent les valoriser. Cependant, ces nouvelles architectures très différentes de celles utilises aujourd'hui (x86, GPU) doivent être caractérisées pour prédire le gain de performance pour les applications. Pour pouvoir profiter de ces technologies et les utiliser de façon optimale, nous avons présenté dans le chapitre précédent une suite de logiciels de caractérisation et d'analyse de performance. L'objectif de ce chapitre est de présenter une méthodologie adaptée,  permettent de réaliser cette caractérisation ainsi que le portage des applications sur ces nouveaux accélérateurs.



\subsection{La révolution de l'hétérogénéité}
%%%%%%%%%%%%%%%%%%%%%%%%%%%%%%%%%%%%%%%%%%%%%%%

    Le protocole Gen-Z, présenté dans le chapitre \ref{sec:edl_hpc}, va révolutionner le monde de l'informatique comme peu de technologies auparavant. Entre tous les bénéfices apportés par ce protocole, la faculté de rendre facile l'hétérogénéité dans les super-calculateurs est sans doute la plus importante. 
    
    L'hétérogénéité sera à la fois entre des accélérateurs spécialisés pour différentes workload, mais aussi dans les architectures elles mêmes permettant de mixer et d'adapter chaque coprocesseur. En 2018, plus de 96\% des processeurs des super-calculateurs du Top500 ont une architecture x86 et la majorité d'entre eux (91\%) proviennent du constructeur Intel. Seulement 28\% des 500 clusters sont associés à des accélérateurs dont 92\% sont des GPU NVIDIA. Nous remarquons donc que l'architecture des super-calculateurs est très similaire et que l'utilisation d'accélérateurs adaptés n'est encore qu'à ses débuts. Aujourd'hui l'utilisation d'architectures différentes est souvent perçu négativement car elle implique d'adapter les codes, d'utiliser plusieurs langages ou d'obtenir de mauvaises performance car l'architecture n'est pas adaptée à la totalité de l'application.
    
    Par analogie, nous comparons cette opportunité avec celle des moteurs d'avion à réaction qui ont révolutionné l'économie et le domaine de l'aviation. Si certains constructeurs et utilisateurs continuent d'utiliser la même stratégie d'ajouter des serveurs \textit{standards} comme jusqu'à aujourd'hui, ils seront dépassés par ceux ayant commencé à investir ces nouvelles technologies plusieurs années avant eux. Il est donc cruciale de s'y préparer en ayant la bonne méthodologie et les bons outils pour pouvoir en profiter. L'accès à des plate-forme exascale et à ces nouvelles architectures va aussi ouvrir de nouveaux marchés, inaccessible aujourd'hui à cause de plusieurs contraintes: le prix, la bande passante nécessaire, la sécurité ou encore la consommation électrique. 
    
    Les gains de performance ne viendront pas seulement par l'utilisation d'accélérateurs puissants, mais de leur diversité et de la capacité des programmeurs de bien les utiliser. Pour une même application, plusieurs accélérateurs spécialisés seront souvent nécessaires. On peut en imaginer certains adaptés à la lecture et à la décompression du jeu de données. Un fois réalisé, des accélérateurs spécialisés dans le calcul demandé pourront être utilisés (ASIC, FGPA ou DSP). Enfin pour la visualisation des données, des GPU seront alors nécessaires. L'hétérogénéité est à la fois un challenge majeur des plate-formes Exascale, mais aussi une grande opportunité. 
    
    

\subsection{Développement}
%%%%%%%%%%%%%%%%%%%%%%%%%%%%%%%%%%%%%%%%%%%%%%%

    Comme présente dans la section \ref{X}, l'analyse de performance peut se faire à plusieurs niveaux. En fonction de l'objectif défini, le niveau et les outils utilisés doivent être adaptés. Notre analyse porte sur la performance d'une partie du code intéressante, un hot spot. Les outils développé permettent pour le moment de mettre en avant des problèmes de performances dus au système mémoire ou au processeur. Les problèmes de performances liés au réseaux ou au système d'exploitation ne sont pas la priorité de la méthodologie, bien qu'ils puissent être décelé.
    
    Du fait de la complexité des architectures, le travail de portage et d'optimisation peut être très difficile. Ainsi, notre démarche s'adresse aux programmeurs ayant de solides connaissances des micro-architectures. Pour améliorer la précision de l'analyse et des outils il est préférable d'avoir accès au code source. 
    
    Contrairement à des solutions existante comme VTune \cite{vtune}, nous avons choisi de développer plusieurs outils indépendant répondant chacun à une question précise. Nous espérons qu'en réduisant la complexité de l'outillage, l'adoption des outils auprès des programmeurs sera plus grande. Les outils n'ont pas vocation d'automatiser entièrement les tâches du programmeur. La puissance de ces outils vient de leur utilisation complémentaire. 
    
    Les outils utilisés sont disponibles en Open Source et ne sont pas exhaustifs. Certains outils utilisés ont été développé durant la thèse, d'autres répondant à nos critères n'ont pas eu à être développés de nouveau. Cette méthodologie et les outils l'accompagnant sont présentés pour partager notre philosophie d'analyse de performance, mais le travail doit être poursuivi pour ajouter de nouveaux outils. De plus, le contexte de leur utilisation est de profiter de l'hétérogénéité arrivant dans les centre de données, ils nécessiteront d'être portés sur ces architectures.



    \begin{figure}
        \center
        \includegraphics[width=10cm]{images/analyse.png}
        \caption{\label{pic_analyse} Délimitation de l'analyse proposée.}
    \end{figure}

    
    
\subsection{Contributions}
%%%%%%%%%%%%%%%%%%%%%%%%%%%%%%%%%%%%%%%%%%%%%%%

    Dans ce chapitre, nous présentons une méthodologie simple en 5 étapes permettant aux utilisateurs de modéliser les performances de leur code, de les projeter sur de nouvelles architectures et de les optimiser (voir \autoref{pic:methodologie_step_2}). Nous proposons un modèle de performance simple basé sur les caractéristiques du sous-système mémoire. L'objectif est de créer pour chaque \textit{hot spot} un modèle de ses performances dans le but de projeter ses performances sur d'autres architectures mais aussi de valider ses performances. Nous cherchons à prouver la bonne utilisation ou non du système mémoire, ressource critique pour la performance des applications sur les architectures modernes. Enfin, lorsque les performances de l'application ne sont pas celles attendues par notre modèle, nous proposons un cheminement pour comprendre, optimiser et transformer le code pour parvenir aux performances ultimes.

    \begin{figure}
    \center
    \includegraphics[width=14cm]{images/methodologie_step.png}
    \caption{\label{pic:methodologie_step_2} Méthodologie en 5 étapes pour caractériser et optimiser une application sur une nouvelle architecture.}
    \end{figure}
            
    
    Pour illustrer les différentes étapes, nous appliquons la méthodologie à l'étude des performances de la fonction \textit{triadd} (voir extrait de code \ref{lst:triadd}) du benchmark Stream \cite{McCalpin1995} sur un processeur Intel\textit{ Xeon Gold 6148} possédant 20 coeurs. Les matrices utilisées mesurent chacune 19.6 GB. Cet exercice nous permet de montrer que même pour un code aussi simple et en apparence optimisée, l'approche et les outils utilisés permettent de comprendre et d'optimiser ses performances.
    
\begin{lstlisting}[language=c,caption=Fonction Triadd extraite du benchmark Stream \ref{McCalpin1995},label={lst:triadd}, 
  basicstyle=\footnotesize, frame=tb,
  xleftmargin=.065\textwidth, xrightmargin=.065\textwidth]
for (j=0; j < STREAM_ARRAY_SIZE; j++)
    A[j] = B[j] + scalar * C[j];
\end{lstlisting}

