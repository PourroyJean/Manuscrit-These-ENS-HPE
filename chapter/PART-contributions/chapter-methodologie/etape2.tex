\iffalse 
%%%%%%%%%%%%%%%%%%%%%%%%%%%%%%%%%%%%%%%%%%%%%%%%%%%%%%%%%%%%%%%%%%%%%%%%%%%%%%%%%%%%%%%%
%%%%%%%%%%%%%%%%%%%%%%%%%%%%%%%%%%%%%%%%%%%%%%%%%%%%%%%%%%%%%%%%%%%%%%%%%%%%%%%%%%%%%%%%
%%_______ .___________.    ___      .______    _______                      ___       %%
%%|   ____||           |   /   \     |   _  \  |   ____|                   |__ \      %%
%%|  |__   `---|  |----`  /  ^  \    |  |_)  | |  |__          ______         ) |     %%
%%|   __|      |  |      /  /_\  \   |   ___/  |   __|        |______|       / /      %%
%%|  |____     |  |     /  _____  \  |  |      |  |____                     / /_      %%
%%|_______|    |__|    /__/     \__\ | _|      |_______|                   |____|     %%
%%%%%%%%%%%%%%%%%%%%%%%%%%%%%%%%%%%%%%%%%%%%%%%%%%%%%%%%%%%%%%%%%%%%%%%%%%%%%%%%%%%%%%%%
%%%%%%%%%%%%%%%%%%%%%%%%%%%%%%%%%%%%%%%%%%%%%%%%%%%%%%%%%%%%%%%%%%%%%%%%%%%%%%%%%%%%%%%%
\fi


%%%%%%%%%%%%%%%%%%%%%%%%%%%%%%%%%%%%%%%%%%%%%%%%%%%%%%%%%%%%%%%%%%%
\section{Étape 2: Caractérisation des plate-formes}
%%%%%%%%%%%%%%%%%%%%%%%%%%%%%%%%%%%%%%%%%%%%%%%%%%%%%%%%%%%%%%%%%%%

%%%%%%%%%%%%%%%%%
\subsection{Motivations et objectifs}
%%%%%%%%%%%%%%%%%

Pour pouvoir estimer la bonne ou mauvaise performance d'un code sur une plate-forme il est nécessaire de connaître la performance crête de cette dernière. Cette approche est utilisée par le modèle du \textit{roof line}. L'approche présentée dans cette thèse nécessite s'intéresse aux mêmes caractéristiques qui sont: la bande passante mémoire ($GB/s$) et la capacité calculatrice ($FLOP/s$). Il y a deux façons d'obtenir ces performances crêtes. La première méthode est de la calculer à partir des caractéristiques techniques de la plate-forme. Cette méthode à été utilisé lors de l'étape 1 permet de catégoriser les plate-formes qui pourront être envisagées pour porter l'application. La deuxième méthode est de la mesurer lors de l'exécution de l'application ou de benchmark. Pour la réalisation de cette étape, il faut avoir accès aux matériels pour pouvoir y réaliser les mesures. Des simulateurs peuvent aussi être utilisés bien que cette étape servent à mesurer les performances réelles de l'architecture.





%%%%%%%%%%%%%%%%%
\subsection{Mesure effective}
%%%%%%%%%%%%%%%%%
Une application industrielle peut être difficile à porter sur une nouvelle architecture dans le simple but de la caractériser. Il est préférable d'utiliser des benchmarks, plus court et plus facilement portable. Ces codes, souvent simple, permettent de caractériser une ou plusieurs parties d'une architecture. Cette valeur valeur est souvent inférieur à la performance théorique du fait de la complexité des architectures ou de la qualité du code généré. Par exemple pour mesurer la bande passante mémoire maximale atteignable on peut utiliser le benchmark \textit{Stream} \cite{McCalpin1995}, les latences des différents niveaux de caches \textit{lmbench} \cite{Staelin2002}. Pour mesurer le nombre maximum d'opération sur un nombre flottant exécutables par seconde, on peut utiliser le benchmark Linpack. 
L'avantage de cette approche est sa facilité. Il suffit de compiler le benchmark voulu et de l'exécuter pour obtenir les résultats. Si le code est optimale, cette méthode permet de mesurer la performance maximale réelle atteignable par l'architecture. Ainsi pourra comparer les performances de l'application avec les performances maximales atteignables par la plate-forme. 

L'inconvénient est que la mesure est dépendant de la qualité de code et du compilateur. Il peut arriver qu'en voulant mesurer spécifiquement un composant, les performances soient dégradées par une autre partie du sous système, empêchant ainsi la caractérisation exacte dont nous avons besoin. Par exemple, si l'on cherche à mesurer la bande passante maximale atteignable par un seul coeur avec un code simple lisant un tableau de données. Sur un processeur récent tel qu'un Xeon Skylake, on s'attendrait à obtenir une valeur proche du maximum théorique de 128 GB/s, calculé dans la partie précédente. Cependant, à cause de la Loi de Little \cite{little2008little} et de la taille de la queue de chargement (\textit{outstanding load queue}), il faut plus de dix coeurs actifs pour saturer la bande passante \cite{JohnMcCalpin2010}. Il faut donc une certaine expérience des outils et des micro-architectures pour apprécier les résultats mesurées. Il peut ainsi être nécessaire d'avoir plusieurs codes de benchmark à exécuter pour valider de différentes façon la micro-architecture. Nous avons développé un benchmark mémoire (section \ref{aaa}) et un benchmark de calcul (section \ref{bbbb}) pour générer différentes versions et évaluer une plate-forme le plus précisément possible.





%%%%%%%%%%%%%%%%%
\subsection{Application au processeur Intel Xeon 6148}
%%%%%%%%%%%%%%%%%
Nous mesurons les performances crête du bus mémoire, $\text{MEMORY}_{max}$ en GB/s, et du processeur $FLOP_{seconde\_max}$ en GFLOP/s, en utilisant deux contributions de la thèse: le benchmark mémoire \textit{dml\_mem} et le générateur de kernel, \textit{kernel\_generator}.


%%%%%%%%%%%%%%%%%
\subsubsection{Performances mémoires}


\textbf{Bande passante mesurée}
Nous utilisons le benchmark mémoire présenté dans \autoref{sec:dml}. A cause de la loi de Little \cite{little2008little}, nous utilisons la version MPI du benchmark sur les 20 coeurs pour s'assurer de saturer le bus mémoire grâce à la commande présentée dans \autoref{lst:dmlmem_xeon}. Nous obtenons une bande passante mémoire $\text{MEMORY}_{max}$ de 105 GB/s que nous avons vérifié avec l'outil \textit{YAMB}.\\

\begin{lstlisting}[caption=Commande utilisée pour obtenir la bande passante maximale avec le benchmark \textit{DML\_MEM}, label={lst:dmlmem_xeon},
  basicstyle=\footnotesize, frame=tb,
  xleftmargin=.065\textwidth, xrightmargin=.065\textwidth]
mpirun -np 20 numactl dml_mem  --steplog 0 --unroll 8 
       --type read --stride 64 --matrixsize 5000
\end{lstlisting}


%%%%%%%%%%%%%%%%%
\subsubsection{Performance de calculs}


\textbf{Performance de calculs mesurée}: nous utilisons le générateur de benchmark \textit{Kernel\_Generator} (voir \autoref{seq:kg}) pour évaluer le nombre maximum d'opérations FMA AVX-512 réalisable sur un nombre flottant à double précision. Pour cela nous avons utilisé la commande présentée dans l'\autoref{code:kg_512}. Nous générons un kernel de 14 instructions FMA pour réduire le coût de gestion de la boucle (incrémentation et comparaison) et masquer la latence des instructions.\\

\begin{lstlisting}[caption=Commande utilisée pour obtenir la performance crête du processeur (GFLOP/s) avec le générateur de benchmark \textit{Kernel\_Generator}., label={code:kg_512},
  basicstyle=\footnotesize, frame=tb,
  xleftmargin=.065\textwidth, xrightmargin=.065\textwidth]
./kg -W 512 -O ffffffffffffff -P double -S 100 -L 120000000
\end{lstlisting}


Le benchmark généré ne possède pas de version multi-coeurs, nous lançons donc indépendamment 20 exécutions du binaire que l'on accroche à un coeur différent grâce à un paramètre du benchmark. Pour cette expérience nous avons commencé par désactiver le turbo du processeur, et limité sa fréquence à 1.6GHz. La performance mesurée est de 998.57 GFLOP/s. Ensuite nous avons activé le turbo, et laissé le processeur choisir lui même sa fréquence. L'\autoref{code:kg_512_output} montre les résultats donnés par le benchmark pour l'exécution sur un des 20 coeurs. Le benchmark mesure sa fréquence effective et trouve bien la valeur de 2.2 GHz renseigné par Intel dans sa documentation. La performance crête d'un coeur est de 69.2 GFLOP/s, approchant le maximum théorique de 70 GFlop/s.


\begin{lstlisting}[caption=Résultat de l'exécution du benchmark sur un coeur avec le turbo activé, label={code:kg_512_output},
  basicstyle=\footnotesize, frame=tb,
  xleftmargin=.005\textwidth, xrightmargin=.005\textwidth]
------------------  INSTRUCTIONS SUMMARY ------------------------------
_label_|   NB INSTRUCTIONS      Time    FREQUENCY    inst/sec       IPC
_value_|      168000000000      38.8          2.2     4.33e+9      2.01
----------------------  FLOP SUMMARY  ---------------------------------
 PRECISION     FLOP/cycle         FLOP/second
    Single              0                   0
    Double           32.0            6.92e+10
-----------------------------------------------------------------------
\end{lstlisting}

Pour vérifier que les 20 coeurs effectuaient bien le même travail, un outil de profilage développé en interne, appelé \textit{mygflops.sh}, a été utilisé. Il permet de compter les instructions flottantes simple et double précision exécutées sur un processeur. Le résultat est présenté dans l'\autoref{code:mygflops_512_output}. La performance crête mesurée du processeur est de 1372.78 GFlop/s, proche du maximum théorique calculé précédemment de 1408 GFLOP/s.

\begin{lstlisting}[caption=Résultat de l'outil \textit{myflops.sh} utilisé pour compter les instructions flottante exécutées sur un processeur., label={code:mygflops_512_output},
  basicstyle=\footnotesize, frame=tb,
  xleftmargin=.005\textwidth, xrightmargin=.005\textwidth]
Single-precision SSE/AVX :       0.00 GFlop/s  --   0.0% of Flops
Double-precision SSE/AVX :     (*\bfseries 1372.78 GFlop/s*)  -- 100.0% of Flops
   0.0% scalar  64-bit SSE/AVX instructions (  0.0% of fp instructions)
   0.0% packed 128-bit SSE/AVX instructions (  0.0% of fp instructions)
   0.0% packed 256-bit AVX instructions     (  0.0% of fp instructions)
 100.0% packed 512-bit AVX instructions     (100.0% of fp instructions)
\end{lstlisting}