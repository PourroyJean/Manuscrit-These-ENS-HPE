This work is part of this vision where the need for computing power is constantly evolving. The infrastructures to be produced must provide it without exceeding the electrical power already achieved by the largest clusters (envelope between 20 and 30 MW). One of the most viable solutions is to optimize the codes to use the most of the available computational power. But today, there is a need to have a methodology to tackle this task and to have the appropriate tools to do the work.

Most of the decisions these objects will have to make will have to be intelligent and made in real time. The entire information system will have to be redesigned if these problems are to be addressed. And new innovations such as those presented in this section will appear: non-volatile memories (RRAM, MRAM, STTRAM), very heterogeneous processors optimized for a workload, and both connected by new photonics networks.


Although still more powerful, these infrastructures do not use all the power at their disposal. As the top500 list shows \cite{Top500}, most supercomputers rarely achieve 80\% efficiency on a simple application like Linpack \cite{Dongarra2003}. For real applications this efficiency is even lower, sometimes less than 10\% \cite{Oliker2005}. There is a lot of work to be done to give applications the ability to access all the computing power that is present but not used. To carry out this work, it is necessary to know what are the capacities of these architectures, how they react according to the applications used and if the performance is optimal.


Optimizing the performance of an application has an impact on its execution speed, but it  does not translate into an economic gain. The construction of a supercomputer costs millions of dollars, but once it has been built it implies large operational costs, of which the biggest part is the energy it uses. PathForward published the technical requirements for the development of an exascale supercomputer \cite{Ang2016}. This report established a power envelope of 20 to 30 Megawatts. Simply put, if the price of a kilowatt-hour is 0.10\$, a budget of 20 million dollars per year will be needed to power an infrastructure consuming 20 MWatt.