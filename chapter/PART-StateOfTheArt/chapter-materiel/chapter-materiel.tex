

\chapter{État de l'art - Matériel}
\label{chap:sota:materiel}
\minitoc

Dans le chapitre précedent nous avons vu...\\
Dans ce chapitre la nous étudions\\
Ce chapitre est long car l'étude et l'analyse de performance nécessite une compréhension fine des architectures.
Grâce aux concepts abordés, le chapitre suivant aborde les X\\

\section{Historique des calculateurs}\label{sec:von}


Bien que leur architecture des processeurs ait beaucoup évolué depuis leur création, l’organisation globale n’a pas changé. Aujourd’hui la majorité d’entre eux sont basés sur une architecture qui porte le nom  du scientifique à l’origine des premiers schémas : l’architecture \textit{Von Neumann}.

\subsection{Les premiers calculateurs}

L'origine des premiers calculateurs remonte bien avant l'invention des systèmes électroniques. Les premières machines à calculer sont apparues à la fin du 16e siècle, conçues par Blaise Pascal et Gottfried Wilhelm Leibnitz \cite{Vie1996}. Les opérations en base 10 étaient à réaliser par l’utilisateur au moyen de roues dentées. 
Pour automatiser le recensement des 62 millions d’Américains vivant aux États Unis en 1890, le gouvernement lança un appel d’offres pour construire un système de traitement automatique. Herman Hollerith proposa alors d’utiliser un système de cartes perforées utilisé par des sociétés ferroviaires produit par la société Computing-Tabulating-Recording.  C’est en 1924 que la société fut renommée  International Business Machines (IBM). DE 1937 à 1943, IBM construit pour l’Université de Harvard un calculateur géant capable de multiplier deux nombres de 23 chiffres en six secondes appelé  Automatic Sequence Controlled Calculator (ASCC) \cite{cortada2016computer}. Cet ordinateur fonctionnait à partir de dispositifs mécaniques et électromécaniques issus des machines à cartes perforées.

Un autre projet de recherche à l'origine de l'ordinateur est celui mené par Allan Turing durant la Seconde Guerre mondiale. Utilisés pour décoder les messages chiffrés d'Enigma utilisée par l'armée allemande. Le câblage de ces premiers automates devait être refait lorsqu'on voulait appliquer un changement dans le traitement des opérations. On parlait alors de programmes extérieurs. Le calculateur d'Allan Turing avait une seule utilité, non des moindres, celle de déchiffrer des messages. Il ne pouvait cependant pas être utilisé pour réaliser d'autres calculs.

En 1945, Von Neumann proposa les 3 premiers principes qui sont à l'origine des ordinateurs actuels. Le premier voulait que ces machines soient universelles, c’est-à-dire qu’elles puissent exécuter différents types de calculs. Le deuxième concernait leur programmation: on parle alors d’instructions organisées dans des programmes qui, comme les résultats intermédiaires, peuvent être sauvés en mémoire. On parle alors de programmes enregistrés. Le troisième principe est l’implémentation de la rupture de séquence. Lors de l’exécution d’un programme, l’automate décide des instructions à exécuter, réalise des tests et des comparaisons pour faire des sauts dans le programme. 


\subsection{L'architecture Von Neumann}

C'est en 1945 que cette architecture a été présentée pour la première fois par John von Neumann dans un papier qu'il n'aura pas le temps de finir \textit{First Draft of a Report on the EDVAC}. Cependant, sur ce papier ne sont pas mentionnés les deux autres contributeurs que sont J. Presper Eckert et John Mauchly. Malheureusement pour eux, l'histoire ne retiendra que le nom du premier. Leur papier décrit une façon d'organiser un système électronique dans le but d'exécuter des instructions. La principale idée de cette architecture est la présence de trois modules principaux (voir \autoref{pic_cpu_von}):
 \begin{itemize}
    \item Une unité de traitement (CPU) qui contient une unité arithmétique et logique (ALU) qui s'occupe d'exécuter les opérations de bases (addition, multiplication, etc.) ainsi que de registres pour mémoriser les données utilisés pour leur exécution.
    \item L'unité de contrôle qui lit les instructions et organise leur exécution en s'occupant de demander les données nécessaires à la mémoire.
    \item Une mémoire qui contient toutes les données, mais aussi le code à exécuter contrairement aux anciennes architectures qui lisaient les programmes sur des cartes perforées, rubans, ou tableaux de connexion.
 \end{itemize}


\begin{figure}
    \centering
    \begin{subfigure}[b]{0.45\linewidth}
        \includegraphics[width=\linewidth]{images/cpu_von1.png}
        \caption{Architecture Von Neuman initiale}
        \label{pic_cpu_von}
    \end{subfigure}
    ~ %add desired spacing between images, e. g. ~, \quad, \qquad, \hfill etc. 
      %(or a blank line to force the subfigure onto a new line)
    \begin{subfigure}[b]{0.45\linewidth}
        \includegraphics[width=\linewidth]{images/cpu_von_new.png}
        \caption{Quelques évolutions de l'architecture}
        \label{pic_cpu_von2}
    \end{subfigure}
    \caption{Architecture Von Neumann\protect\footnotemark }\label{fig:cpu_archi_von}
\end{figure}

\footnotetext{source: https://interstices.info/le-modele-darchitecture-de-von-neumann}

Bien que l’implémentation de cette architecture ait évolué avec l’apparition de nouveaux matériaux, les architectures modernes des processeurs sont toujours construites sur ce modèle. Deux évolutions de l’architecture initiale ont cependant été appliquées (\autoref{pic_cpu_von2}). Les entrées et sorties ne sont plus gérées directement par l’unité centrale, mais par des microprocesseurs dédiés. L’unité centrale de traitement n’est plus unique depuis l’apparition des processeurs multicoeurs. Ces évolutions ne remettent pas en cause les principes de bases énoncés par Von Neumann. 


\subsection{Architecture Harvard}

Dans leur papier \cite{238389}, Von Neuman et al. précisent que la façon dont sont stockées les instructions et les données en mémoires doit être la même. Cette configuration la différencie de sa principale concurrente, l'architecture  Harvard (figure \ref{pic_neumannHarvard}) qui utilise deux bus pour accéder à deux mémoires réservées l'une pour les instructions, l'autre pour les données. 
Cette configuration permet à l'architecture Harvard d'accéder en parallèle aux données et aux instructions. De plus, comme les instructions et les données sont séparées, elles peuvent être stockées sur des supports de différentes performances. On peut ainsi utiliser un support plus cher pour stocker les données avec des mémoires très rapides (SRAM) et stocker les instructions sur des mémoires moins chers de type ROM. De plus, l’architecture Harvard apporte une sécurité en empêchant les processeurs d’exécuter des instructions provenant du stockage réservé aux données. Enfin la nécessité d’avoir deux bus rend les puces Harvard plus chères et les performances sont souvent moins bonnes. Un code qui aurait beaucoup d’accès mémoire ne pourrait pas profiter de la disponibilité du canal allant à la mémoire des instructions. Dans une architecture Von Neumann à deux canaux, les bus peuvent être utilisés aussi bien pour des instructions que pour les données. Cependant la performance de ce bus mémoire est à l’origine du déséquilibre de performance des architectures modernes. Ainsi, il est courant d’utiliser le terme de \textit{goulot de Von Neumann} pour désigner ce point faible des architectures modernes.

\begin{figure}
    \center
    \includegraphics[width=10cm]{images/Chapitre1/neumannHarvard.png}
    \caption{\label{pic_neumannHarvard} Les deux principales architectures de processeurs: Von Neumann et Harvard. }
\end{figure}

\section{L'architecture} \label{sec:micro}
%%%%%%%%%%%%%%%%%%%%%%%%%%%%%%%%%%%%%%%%%%%%%%%%%%%%%%%%%%%%%%%%%%%
%%%%%%%%%%%%%%%%%%%%%%%%%%%%%%%%%%%%%%%%%%%%%%%%%%%%%%%%%%%%%%%%%%%

Dans la section \autoref{sec:logique} sont présentés les transistors, les éléments de bases des ordinateurs. Les transistors sont groupés en portes logiques pour construire des circuits électroniques. Cette section présente comment ces circuits sont utilisés pour construire la micro-architecture d'un ordinateur, capable d'exécuter les instructions défini par une \textit{Instruction Set Architecture} (ISA). L'objectif n'est pas de présenter la totalité de l'architecture mais seulement les éléments importants nécessaires pour la suite de la thèse.


Afin d'éviter toutes confusions, nous rappelons la définition de la micro-architecture et de l'ISA car ces termes sont souvent confondus dans la littérature:

\begin{itemize}
    \item  \textbf{La couche ISA} (\textit{Instruction Set Architecture}) regroupe les instructions, leur mode (système ou utilisateur), les registres utilisables, l'organisation du système mémoire (alignement, espace d'adressage) ... 
    C'est une spécification formelle établie, qui peut être utilisée par plusieurs fabricants de micro-architecture \cite{tanenbaum2016structured}. Intel publie fréquemment la documentation de l'ISA x86 \cite{guide2011intel}. Elle forme l'interface entre le matériel et le logiciel et permet la compatibilité de programme sur des micro-architectures de différents constructeurs. 
    Grâce à la couche ISA, différents langages de programmation peuvent être utilisés pour écrire les application. Le compilateur s'occupe alors de les traduire dans un langage bas niveau pouvant utiliser l'ISA (langage assembleur). Ce langage est tellement proche de la couche ISA que les deux termes sont souvent mélangés. Les \textit{ISA} existantes sont listées dans la \autoref{sec:isa}. 

    \item \textbf{La micro-architecture} correspond à l'implémentation matériel de l'\textit{ISA} est implémentée matériellement par la micro-architecture. Ce sont deux couches distinctes, la seconde n'ayant pas forcément besoin d'avoir connaissance de la première (bien que pour des raisons de performances cela soit préférable). En ayant connaissance de la micro-architecture, le compilateur pourra réordonner ou modifier des instructions pour tirer parti du pipeline ou d'un processeur vectoriel. La conception d'une nouvelle micro-architecture doit commencer par choisir l'ISA à implémenter (si possible existante pour permettre la compatibilité des programmes). Les différences principales entre deux micro-architectures implémentant la même ISA sont leur différence de performances et de coût. Les processeurs Intel et AMD implémente la même \textit{ISA x86}. La performance et le nombre d'instructions supportés par les deux architectures est cependant différent.
    
    \item Le terme d'\textbf{architecture} est souvent employé à la place du terme \textit{ISA}, notamment par IBM en 1964 \cite{amdahl1964architecture}.  Aujourd'hui il est souvent utilisé pour faire référence à la fois à l'\textit{ISA} et à la \textit{micro-architecture}. Il est courant d'entendre parler d'\textit{architecture x86} pour faire référence à une micro-architecture implémentant l'\textit{ISA x86}.\\ 
\end{itemize}



\subsection{Performance d'une architecture}
%%%%%%%%%%%%%%%%%%%%%%%%%%%%%%%%%%%%%%%%%%%%%%%%%%%%%%%%%%%%%%%%%%%


Le développement d'une nouvelle micro-architecture doit prendre en compte plusieurs facteurs qui peuvent impacter son implémentation (vitesse de traitement des instructions, le coût de fabrication, la fiabilité, consommation électrique, taille). Ces différents facteurs font pression sur les architectes de processeurs qui doivent redoubler d'inventivité pour les satisfaire en implémentant des optimisation matériels.

Les améliorations ont pour but d'améliorer la performance de l'architecture, la plus part du temps de façon transparente pour l'utilisateur. Cependant, un programmeur n'étant pas avertis de ces optimisations, pourrait écrire des applications inefficaces voire contre-productive. Les développeurs d'applications \textit{HPC} étant généralement des scientifiques experts dans leur domaine (physique, chimie, mécanique…), il est fréquent de voir des codes peu efficaces. Cette section, liste les principales améliorations des micro-architectures qu'il faut connaître et utiliser pour exploiter le maximum des performances disponibles. 


Pour améliorer la vitesses d'exécution d'un programme, quatre moyens peuvent être utilisés:
\begin{itemize}
    \item Utiliser une nouvelle technologie, plus rapide, consommant moins d'énergie, moins cher ou plus dense (technologie mémoire, fibre optique...). Les nouvelles technologies sont nombreuses et les principales sont listées et discutées dans la \autoref{sec:opportunité}. 
    \item Améliorer l'efficacité des instructions: \autoref{sec:efficacite} 
    \item Accélérer l'exécution des instructions: \autoref{sec:accelerer}
    \item Exécuter les instructions en parallèle: \autoref{sec:para}
\end{itemize}

Quand elles ne sont pas matérielles, ces différentes améliorations peuvent être réalisés par deux techniques: dynamique ou statique. Les méthodes statiques interviennent avant que le code soit exécuté. Généralement, c'est le compilateur qui les implémente lors de la génération du code. Les techniques dynamiques sont mis en oeuvre au fil des exécutions d'instructions. Elles nécessitent donc du matériel spécialisé supplémentaire et sont donc très coûteuses. La section présente les principales techniques et matériels utilisés pour l'amélioration des performances (dynamique et statique). Deux améliorations majeures sont présentées dans deux section distinctes: la hiérarchie mémoire (\autoref{sec:hierarchie} et la mémoire virtuelle (\autoref{sec:memoire_virtuelle}).







%%%%%%%%%%%%%%%%%%%%%%%%%%%%%%%%%%%%%%%%%%%%%%%%%%%%%%%%%%%%%%%%%%%
\subsection{Améliorer l'efficacité de l'exécution} \label{sec:efficacite}
%%%%%%%%%%%%%%%%%%%%%%%%%%%%%%%%%%%%%%%%%%%%%%%%%%%%%%%%%%%%%%%%%%%




\subsubsection{Jeu d'instructions ISA } \label{sec:isa}
%%%%%%%%%%%%%%%%%%%%%%%%%%%%%%%%%%%%%%%%%%%%%%%%%%%%%%%%%%%%%%%%%%%

Le choix de l'\textit{ISA} à implémenter est le premier choix à réaliser lors du développement d'une nouvelle micro-architecture. L'\textit{ISA} utilisée à un impact sur la performance, la facilité de programmation, les applications compatibles...

Les jeux de d'instructions existant sont séparés en deux grandes familles d'ISA: les jeux d'instructions CICS pour \textit{Complex Instruction Set Computing} et les instructions RISC pour \textit{Reduced Instruction Set Computing}. La principale différence entre les deux est la complexité de leurs instructions. 

    \paragraph{CISC} est la première famille d'instructions à avoir été utilisé massivement. Ces instructions sont dites complexes car une seule instructions peut à elle seule demander plusieurs opérations à réaliser. Par exemple une addition CISC s'occuperait de charger les données depuis la mémoire, d'exécuter l'addition ensuite et sauver le résultat. A l'origine, beaucoup de codes étaient écrit en assembleur, et ce genre d'instructions permettaient au programmeur d'éviter d'écrire de nombreuses lignes de codes souvent redondantes. De plus les codes générés en CISC sont plus petit et nécessitent donc moins de mémoire, qui à l'origine manquait énormément.

    \paragraph{RISC} regroupe les ISA dites \textit{simple}. En 1970, John Cocke, alors ingénieur chez IBM, proposa de réduire le nombre d'instructions CISC \cite{cocke1990evolution}. Le terme \textit{réduit} fait référence au nombre d'instructions plus petit que celles de CISC mais aussi pour signifier que le travail à réaliser par une instructions était moindre que pour une instruction CISC. Pour réaliser une multiplications entre deux données, on devra alors explicitement charger la première donnée, puis la deuxième et enfin écrire l'instruction qui correspond à la multiplication. Les instructions étant plus nombreuses le travail des compilateurs est augmenté mais souvent l'exécution des codes résultantes en est réduite. Le RISC fut une réponse apporté a la lenteur de décodage du CISC.  Toutes les instructions font la même taille, les architectures nécessitent donc moins de transistors pour les analyser. Les micro-architecture sont donc moins coûteuses et peuvent atteindre des fréquences plus élevées.

Ces deux familles d'instructions ont toutes deux leurs avantages et leurs inconvénients et les puces actuelles comportent des parties qui exécutent des instructions RISC et d'autres en CISC. L'ISA la plus rependu est le x86 qui se veut être un jeu d'instruction CISC. Le RISC augmente le nombre d'instructions lues séparément par le micro-processeur, si cela à le désavantage de consommer plus de mémoire, cela à aussi d'autre avantages, comme de pouvoir optimiser leur exécution avec la mise en place d'un \textit{pipeline}. Les \textit{ISA RISC} les plus connus sont les \textit{ISA} ARM, \textit{MIPS} (utilisé dans le domaine universitaire pour apprendre le langage assembleur), PA-RISC (Hewlett-Packard) et RISC-V. L'\textit{ISA CISC} la plus utilisée dans les super-calculateurs aujourd'hui est \textit{x86} (processeurs Intel et AMD).



\paragraph{Extensions vectorielles}.
Pour tirer partie de la puissance de calculs des processeurs et des unités de calculs vectorielles, les \textit{ISA} ont reçu de nombreuses extensions (\autoref{tab_simd}).
C'est en 1995 que Sun Microsystem introduit son premier jeu d'instruction vectoriel, le Visual Instruction Set auquel Intel répondra en 1997 avec son processeur Pentium MMX et le jeu d'instructions du même nom. Ainsi ces instructions pouvaient réaliser des opérations sur les jeux de données tel que les images ou vidéos de facon tres performante. En 1997 AMD viendra ameliorer le jeu d'instruction MMX avec l'implémentions des instructions 3DNow ! qui rendait alors possible les operations vectoriels sur les nombre flottant. La possibilite de faire une operation sur deux nombre flottant par cycle doublait alors la performance des processeurs. Intel repondit alors a cette avance avec le jeu d'instructions Streaming SIMD Extensions (SSE) en 1999. Il est alors facile d'augmenter la puissance des processeurs : il suffit d'augmenter la taille des vecteurs.
L'\textit{ISA x86} reu plus de dix extensions dans les vingts dernires annes pour s'adapter l'agrandissement des units vectorielles. Les principales sont \textit{MMX} (1996), \textit{3DNOW!} (1998), 6 versions de \textit{SSE} de 1999 2008 et enfin \textit{AVX-2} et \textit{AVX-512} en 2013 et 2015. videmment, les codes ne peuvent pas profiter de ces instructions sans une micro-architecture capable de les excuter. Ces micro-architectures sont prsentes dans la sous-section suivante. 


\begin{table}[]
\centering
\caption{Évolutions principales des instructions SIMD $x86$}
\label{tab_simd}
\begin{tabular}{|l|l|l|l|l|l|}
\hline
\multicolumn{1}{|c|}{\textbf{Année}} & \multicolumn{1}{c|}{\textbf{Nom}} & \multicolumn{1}{c|}{\textbf{Nb registres}} & \multicolumn{1}{c|}{\textbf{Taille (bit)}} & \multicolumn{1}{c|}{\textbf{Registres}} & \multicolumn{1}{c|}{\textbf{Commentaires}} \\ \hline
1996                                 & MMX                               & 8                                          & 64                                         & MM0                                     & Nombre entiers
\\ \hline
1999                                 & SSE                               & 8                                          & 128                                        & XMM0                                    & 120 instr., simple precisions              \\ \hline
2006                                 & SSSE3                             & 8                                          & 128                                        & XMM0                                    & 300 instr., double précisions              \\ \hline
2008                                 & AVX                               & 16                                         & 128                                        & XMM0                                    & FMA4, op. a 3 opérandes                    \\ \hline
2011                                 & AVX2                              &  16                                          & 256                                        & YMM0                                    &                                            \\ \hline
2013                                 & AVX512                            & 32                                         & 512                                        & ZMM                                     & FMA3                                       \\ \hline
\end{tabular}
\end{table}

\begin{figure}
    \center
    \includegraphics[width=7cm]{images/Chapitre1/simd_registres.png}
    \caption{\label{pic_simd_registres} Découpage d'un registre de 512 bits}
\end{figure}

\subsubsection{Les processeurs vectoriels } \label{sec:cpu_vectoriel}
%%%%%%%%%%%%%%%%%%%%%%%%%%%%%%%%%%%%%%%%%%%%%%%%%%%%%%%%%%%%%%%%%%%

Dans les années 90, les ordinateurs personnels commençaient à être utilisés pour le multimédia (film, image, musique). Dans le but d'augmenter la performance des applications les architectes ont voulu améliorer l'efficacité des instructions en implémentant des architectures SIMD (Single Instruction Multiple Data de la taxonomie de Flynn ). 


L'architecture vectorielle repose sur les même fondement que l'architecture superscalaire: réduire le temps d'exécution d'un code en utilisant le parallélisme et diminuer le coût en transistors de la micro-architecture par la mise en commun du matériel entre plusieurs unités de calculs (\autoref{pic_cpu_simd}). La pression que subit l'unité responsable du \textit{fetch} et du \textit{decode} est donc réduire car avec une seule instruction vectorielle, le processeur réalise le travaille de plusieurs instructions scalaires.  Contrairement au processeur scalaire qui exécutent une instructions sur une seule donnée, les processeurs vectoriels sont capables d'exécuter une instructions sur plusieurs données simultanément (\autoref{pic_simd2}).


Il en résulte que doubler largeur des instructions revient à doubler le nombre de FLOP réalisable par le processeur alors que la consommation électrique augmentera d'un facteur inférieur à 2. Beaucoup d'efforts ont donc été réalisé pour être capable d'utiliser cette technologie, par exemple la majorité des compilateurs est capable de détecter les zones de codes parallélisable pouvant bénéficier d'instructions SIMD.

Cependant, dans la pratique les codes ne sont pas aussi parallèle que nous le souhaiterions et certains aspect du code empêche de tirer la totalité de la performance disponible. En effet, si les données sont indépendantes les unes des autres, il est alors impossible de les calculer simultanément rendant la partie vectorielle inutilisable. Aussi, les performances peuvent être réduites si les données accédées ne sont pas continues en mémoire. Cela demande donc un travail supplémentaire pour repenser les structures de données et s'assurer que les donnes transférées sur le bus mémoire sont des données utiles (voir le concept de ligne de cache dans la \autoref{sec:cache}).

\begin{figure}
    \begin{subfigure}[]{0.5\linewidth}\centering
        \vspace{1.6cm}
        \includegraphics[width=.5\linewidth]{images/cpu_simd.png}
        \label{pic_cpu_simd}
        \caption{Un processeur vectoriel exécute une même instruction sur différentes données}
    \end{subfigure}%
    ~ %space
    \begin{subfigure}[]{0.5\linewidth}\centering
        \includegraphics[width=\linewidth]{images/Chapitre1/simd.png}
        \caption{Schéma de fonctionnement d'une multiplication vectorielle\protect\footnotemark}
        \label{pic_simd2}
    \end{subfigure}
   
    \caption{Un processeur vectoriel travaille sur un groupe de données indépendantes. Les opérandes sont alors stockées dans des registres vectoriels capable de charger un vecteur pour y appliquer une même opération.}
    \label{pic_simd}
\end{figure}
\footnotetext{source:\url{https://software.intel.com/en-us/articles/ticker-tape-part-2}}


%%%%%%%%%%%%%%%%%%%%%%%%%%%%%%%%%%%%%%%%%%%%%%%%%%%%%%%%%%%%%%%%%%%
\subsection{Accélérer l'exécution des instructions} \label{sec:accelerer}
%%%%%%%%%%%%%%%%%%%%%%%%%%%%%%%%%%%%%%%%%%%%%%%%%%%%%%%%%%%%%%%%%%%


\subsubsection{Lien entre fréquence et performance}
Les processeurs sont des circuits électronique dits \textit{synchrone}. Leur fonctionnement est cadencé par une horloge donnant des impulsions régulières aux composants du circuit pour organiser leur synchronisation. Lorsqu'une instruction est exécutée par le processeur les passages dans les différentes étapes de la micro-architecture et notamment du pipeline sont régis par cette horloge. Au plus l'horloge est rapide, au plus l'exécution l'est aussi. Cependant, la durée séparant deux signaux (un cycle) ne peut pas être réduite autant que tout architecte le souhaiterait. En effet, les vitesses d'horloges sont si rapide, qu'il est nécessaire de laisser suffisamment de temps au signal électrique de se propager dans le circuit. La taille du processeur et sa fréquence sont donc liées, c'est notamment pour cette raison qu'il n'existe pas de processeur mesurant plusieurs dizaines de centimètres. À une fréquence de 3 GHz, le signal électrique, qui se déplace à la vitesse de la lumière, ne peut parcourir que 15 centimètres entre deux cycles. Au temps de propagation il faut aussi prévoir le temps de préparer et recevoir la communication. L'augmentation de la fréquence des processeurs à donc une première limite physique infranchissable bien que d'autres limites présentées dans cette partie soient encore plus contraignantes.

La fréquence des processeurs a beaucoup évoluée depuis les premiers processeurs (\autoref{pic_cpu_frequency}). Les premiers processeurs \textit{Pentium} d'Intel utilisaient des fréquences de 60 MHz en 1993. Pendant plus de dix ans, les fréquences ont évoluées chaque année d'un facteur de 40\%  \textbf{tous les deux ans ou un ans ???}pour atteindre des vitesses de plusieurs Gigahertz. En 2000, Intel annonçait la fabrication de processeurs cadencés à 10 GHz\footnote{source \url{https://www.clubic.com/actualite-1791-des-processeurs-intel-10-ghz-pour-2005.html}} pour les années 2005. Pourtant, en 2019, nous sommes encore loin d’utiliser des processeurs avec de telles cadences. Mise à part les systèmes surcadencés, utilisant des système de refroidissement liquide, il est rare de voir des systèmes utiliser des processeurs récents à plus de 5 Ghz.

\begin{figure}
    \center
    \includegraphics[width=10cm]{images/cpu_frequency.jpg}
    \caption{\label{pic_cpu_frequency} Évolution de la fréquence des processeurs\protect\footnotemark.}
\end{figure}
\footnotetext{source: \url{https://en.wikipedia.org/wiki/Beyond_CMOS}}

Le but de cette section est comprendre pourquoi l'évolution de la fréquence des processeurs s'est arrêtée brusquement autour des 4 GHz. Pour cela, nous expliquons comment la loi de Moore a permis d'en arriver là, et quelles sont les limites physiques qui empêchent de poursuivre cette évolution.

\subsubsection{Loi de Dennard}
Comme constaté dans la section précédente, la fréquence de l'horloge d'un processeur est limitée par le temps nécessaire au signal de se propager entre deux éléments d'un circuit. Au plus les transistors seront proches, au plus la vitesse de propagation sera faible. La loi de Moore prévoit que le nombre de transistors double tout les deux ans pour une surface donnée. Cela est rendu possible par l'affinement de la gravure permettant de réaliser des transistors toujours plus petits et donc plus proches. Avec plus de transistors, les processeurs peuvent utiliser des pipelines plus complexes, entre divisant les étapes les plus longues permettant aussi l'utilisation de fréquence plus élevée. Nous expliquons comment l'évolution de la fréquence des processeurs est intimement liée avec la loi de Moore en utilisant la formule de la consommation électrique d'un circuit CMOS \cite{martin2014post}:
\begin{equation}
P = QfCV^{2} +  VI_{leakage}
\label{eq:power}
\end{equation}

L'intérêt de cette formule est d'apprécier comment la puissance et la fréquence d'un processeur est impactée par la loi de Moore. Pour cela, nous étudions sa variation entre deux générations de processeurs, c'est à dire une période de deux ans.

\paragraph{Le nombre de transistors $Q$.} D'après la loi de Moore, le nombre de transistors double tous les deux ans pour un circuit de même surface. Ainsi la surface des transistors est divisée par deux. Leurs longueur et largeur est ainsi réduite d'un facteur $\sqrt{2}$. Cette valeur est appelé facteur de \textit{scaling} et a une valeur de $1.44$.

\paragraph{La capacité du circuit $C$.} La capacité d'un transistor peut être calculée par la formule suivante $C = \frac{S \times e}{d}$. Avec $S$ la surface du transistor, $e$ la permitivité électrique du matériau utilisé (pouvant être considéré comme fixe entre deux générations), et $d$ la distance séparant la grille et le semi-conducteur (isolant). Comme vu précédemment, la surface $S$ d'un transistor est divisé par $2$ entre deux générations. La distance $d$ est elle réduite par un facteur $\sqrt{2}$. Tous les deux ans, la capacité d'un transistor est donc réduite de $\sqrt{2}$. 

\paragraph{La fréquence du circuit $f$.} La fréquence d'utilisation d'un transistor dépend essentiellement de la vitesse à laquelle la grille peut être chargée ou déchargée. Diminuer sa capacité diminue du même facteur ce temps de remplissage. Entre deux générations, la fréquence est supposée augmenter d'un facteur $\sqrt{2}$. Cette valeur correspond bien à l'augmentation de $40\%$ constatée dans l'introduction de cette section. 

\paragraph{La tension de fonctionnement $V$.} La tension est proportionnelle à la finesse de grave utilisée. Diviser la finesse de gravure pas un coefficient de $\sqrt{2}$ rien à diviser la tension de fonctionnement par le même facteur. 

\paragraph{Les courants de fuites $I_{leakage}$.} Les courants de fuites sont considérés comme négligeables (pour le moment). 

\paragraph{Variation de $P$.} A une tension de fonctionnement $V$ égale entre deux générations de processeurs et en reprenant les variations des différentes valeurs, nous constatons que $P$ ne varie pas entre deux générations. La baisse de capacité compenser l'augmentation de la fréquence (facteur $\sqrt{2}$). La baisse de V NTM 


% Please add the following required packages to your document preamble:
% \usepackage{graphicx}
% \usepackage[table,xcdraw]{xcolor}
% If you use beamer only pass "xcolor=table" option, i.e. \documentclass[xcolor=table]{beamer}
\begin{table}[]
\centering
\resizebox{\textwidth}{!}{%
\begin{tabular}{lc}
\hline
\rowcolor[HTML]{EFEFEF} 
{\color[HTML]{000000} Paramètre} & {\color[HTML]{000000} Coefficient multiplicateur (tous les deux ans)} \\ \hline
Finesse de gravure & $\frac{1}{\sqrt{2}}$ \\
Nombre de transistors par unité de surface & $2$ \\
Tension d'alimentation & $\frac{1}{\sqrt{2}}$ \\
Capacité d'un transistor & $\frac{1}{\sqrt{2}}$ \\
Fréquence & $\sqrt{2}$ \\ \hline
\end{tabular}%
}
\caption{ Résumé des impacts de la diminution de la finesse de gravure d'un facteur $\sqrt{2}$\protect\footnotemark.}
\label{tab:dennard}
\end{table}

\footnotetext{source: \url{https://fr.wikibooks.org/wiki/Fonctionnement_d\%27un_ordinateur/La_consommation_d\%27\%C3\%A9nergie_d\%27un_ordinateur}}

Le \autoref{tab:dennard} résume les différents impacts que la diminution de la finesse de gravure à sur les différentes propriétés d'un circuit. On remarque deux choses concernant l'évolution de $P$ entre deux générations de circuit. La première est que l'augmentation de la fréquence est compensée par la baisse de la capacité du circuit. La deuxième est que le baisse de tension compense le nombre de transistors. Ainsi, la consommation électrique d'un processeur de varie pas entre deux génération bien que la fréquence augmente de $40\%$ et que le nombre de transistor soit doublé. Cette propriété est connue sous le nom de Loi de Dennard \cite{Dennard1974} qui assurait en 1974 que la densité énergétique resterait constante entre deux générations de processeurs. Cette propriété est restée vrai durant 30 ans, permettant l'augmentation de la performance des processeurs sans augmenter drastiquement leur consommation électrique. 




\subsubsection{Fin de Dennard}
Les équations de Dennard se sont appliquées pendant plus de 30 ans, voyant la fréquence des processeurs augmenter de 40\% tous les 2 ans (voir \autoref{pic_cpu_evolution}). Cependant, l'\autoref{eq:power} ignorait les courants de fuite $I_{leakage}$ alors peu significatifs. Cependant, avec la miniaturisation des transistors, ces fuites augmente exponentiellement pour des tailles de gravure inférieur à 65nm \cite{martin2014post}. Alors que la loi de Dennard prévoyait une consommation électrique constante entre deux générations de processeur, ces fuites de courant vont faire augmenter la consommation des puces d'un facteur 2. Ceci entraîne une forte évolution de la densité électrique à chaque nouvelle génération (voir \autoref{pic_cpu_leakage}) impactant la consommation électrique des processeurs (voir \autoref{pic_cpu_evolution}). Au plus la fréquence des processeurs est élevée, au plus les transistors sont activés augmentant d'autant les courants de fuite. La chaleur dégagée par les puces n'est alors plus soutenable et l'évolution de la fréquence des processeurs s'arrête ainsi autour des années 2005.

\begin{figure}
    \center
    \includegraphics[width=10cm]{images/cpu_evolution.png}
    \caption{\label{pic_cpu_evolution} Évolution des caractéristiques des processeurs \cite{rupp40years}.}
\end{figure}

\begin{figure}
    \center
    \includegraphics[width=10cm]{images/cpu_leakage.png}
    \caption{\label{pic_cpu_leakage} La miniaturisation de l'isolant nécessaire au fonctionnement des transistors permet le passage de courant de \textit{fuites}.}
\end{figure}

Bien que la fréquence des processeur n'augmentent plus depuis plus de 10 ans, la \autoref{pic_cpu_evolution} montre que la performance des processeurs continue bien d'augmenter. En effet, si la loi de Dennard n'est plus valide, la loi de Moore elle l'est encore après 2005. Les processeurs reçoivent toujours plus de transistors permettant d'augmenter la complexité des pipelines. Ces transistors sont alors utiliser pour construire des processeurs a plusieurs coeurs, présentés dans la \autoref{sec:multicore} (on remarquera que l'augmentation du nombre de coeurs commence exactement quand la fréquence n'augmente plus). Les processeurs sont alors capable de réguler la chaleur d'un processeur en plaçant les processus peu actifs sur les points chauds de la puce.
Pour limiter la consommation électrique, des techniques de management d'alimentation sont alors mises en place pour éteindre certaines parties du processeur inutilisées.
L'utilisation de différentes fréquences est aussi largement utilisée. Les processeurs possèdent des fréquences dites \textit{turbo}, leur permettant d'atteindre des fréquences très élevées pendant un cours laps de temps ou lorsque le processeur n'est pas pleinement utilisé (coeurs et calculs vectoriels inutilisés).



\subsubsection{FPU}\label{sec:fpu}


L'unité de calcul en virgule flottante (FPU pour \textit{floating-point  unit}) est un composant du processeurs permettant de réaliser les opérations sur les nombres à virgule flottante. A l'origine ce module était séparé du processeur et il convenait à l'utilisateur de choisir si il voulait ou non en brancher une sur la carte mère dans l'emplacement qui lui était alors dédié. Pour des questions de coûts d'intégration et de performance, la FPU est depuis 1989, avec la sortie du processeur Intel 80486,  intégrée au processeur. 

Au fil des années la complexité de la FPU a augmenté, quand elle n'était capable d'exécuter que de simples opérations à l'origine, elle peut désormais réaliser des opérations complexes (division, racine carrée, exponentielles ou des fonctions trigonométriques).  De plus, des instructions fusionnées ont fait leur apparition en 2013 dans les processeurs \textit{Haswell} de Intel et Piledriver pour AMD. Elles ont la particularité de réaliser 2 opérations en un seul cycle d'horloge du processeur. Connues sous le nom de FMA pour \textit{fused multiply-add} elles sont capables d'exécuter l'instruction $a \leftarrow b * c + d$ en un seul cycle. 

Enfin, les unités de calcul modernes sont capable d'exécuter une opération sur plusieurs données à la fois ce type d'instructions est dit \textit{vectoriel}. Ces instructions sont performantes pour les algorithmes qui doivent exécuter une même opération sur plusieurs données (un vecteur par exemple). L'évolution de leurs caractéristiques sont abordées dans la partie \ref{sub_taxonomie}.
Les différentes évolution de la FPU sont responsable de la forte augmentation de la puissance de processeur. En effet, usuellement la puissance d'un processeur est donnée par le nombre de calculs flottant qu'il peut exécuter par cycle. Le tableau \ref{tab_FPU} montre comment le nombre de FLOP par cycle évolue: Pour Intel, cette performance a été multipliée par deux à chaque nouvelle version de l'architecture et les FPU modernes exécutent 8 fois plus de calculs qu'en 2008.

\begin{table}[]
\centering
\caption{Evolution de la performance des FPU}
\label{my-label}
\begin{tabular}{|l|l|l|l|l|l|}
\hline
\multicolumn{1}{|c|}{\textbf{Année}} & \multicolumn{2}{c|}{\textbf{Architecture Intel / AMD}}     & \multicolumn{1}{c|}{\textbf{Simpe p.}} & \multicolumn{1}{c|}{\textbf{Double p.}} & \multicolumn{1}{c|}{\textbf{Opérations}} \\ \hline
2008                                 & Nehalem                          & K10                     & 8                                      & 4                                       & 4 add. et 4 mul.         \\ \hline
2011                                 & Sandy Bridge                     & Bulldozer               & 16                                     & 8                                       & 8 add. et 8 mul.         \\ \hline
2013                                 & Haswell \& Skylake               &                         & 32                                     & 16                                      & 8 FMA (mult. + add.)        \\ \hline
2016                                 & Xeon Phi KNL &                         & 64                                     & 32                                      & 8 FMA (mult. + add.)        \\ \hline
\end{tabular}
\label{tab_FPU}
     \vspace{1ex}

     \raggedright Avec Haswell le nombre d'instructions exécutable n'évolue pas mais c'es le type d'instructions exécuté qui sont des FMA (une multiplication et une addition en un cyle d'horloge) et qui correspond à deux FLOP.
\end{table}

\textbf{TODO} latence des differentes instructions; skylake FMA 4 cycles %http://agner.org/optimize/blog/read.php?i=415 (calc_fma() combien de FMA en parallele)







\subsubsection{Optimisations}

Dans les sous-parties précédentes est présenté comment l'évolution de la fréquence et l'utilisation de matériel spécialisé comme une FPU permet d'accélérer l'exécution d'une instruction. Pour différentes raisons (dépendance, manque d'une opérande), les instructions ne peuvent pas être exécutées à la vitesse maximale théorique prévue par le processeur. Pour maximiser le nombre d'instructions exécutées chaque cycle, les processeurs ont reçu de nombreuses optimisations. 

\paragraph{Exécution dans le désordre.}\label{sec:out_of_order}

Le but de cet optimisation est de cacher l'attente de donnée du processeur de la mémoire. Cette avancée est apparue sur les processeurs Intel en 1995 avec le \textit{Pentium P6}. Le principe de l'exécution dans le désordre est d'exécuter les instructions dans un ordre différent que celui donné par le code source.  Ainsi lorsqu'une instruction doit attendre une donnée, au lieu de perdre des cycles à attendre ces données, le processeur va exécuter les instructions qui suivent et finira d'exécuter la première quand la donné sera chargée dans un registre. 
Cependant ne pas exécuter le programme dans l'ordre initial peut fausser les résultats. C'est alors au processeur de s'assurer que les instructions permutées ne sont pas dépendantes. Il existe trois types de dépendances. La première est une \textit{lecture après écriture} (Read After Write ou RAW): une instruction lit une donnée écrite par une instruction la précédent. La deuxième est une \textit{écriture après lecture} (WAR): une première instruction lit une donnée qui est modifiée par une instruction la suivant. Enfin, la dernière dépendance est une \textit{écriture après écriture} (WAW): deux instructions écrivent sur la même donnée. Pour éliminer ces deux dernières dépendances, les processeurs possèdent plus de registres que ceux adressable par le programme. Il les utilise pour éliminer les dépendances WAR et WAW grâce à des techniques de renommage de registres \cite{903248}. 
L'extrait de code \ref{code_dependances} donne un exemple pour chaque type de dépendances. Dans chaque cas, la valeur de B n'est pas la même si les deux assignations ne sont pas exécutées dans le même ordre, suivant cet ordre, B peut valoir 10 ou 15. Pour ce faire, le processeur possède une fenêtre de plusieurs instructions, aussi appelée (à tord) l'\textit{execution queue}. En effet, cette liste n'a pas vocation à être executé dans l'ordre, c'est un rassemblement d'instructions qui ont des dépendances entre elles. Le processeur vient mettre à jour cette liste pour essayer d'exécuter des instructions qui n'ont plus de dépendances avec une donné ou une autre instruction.
Pour pouvoir bénéficier de l'exécution dans le désordre, le processeurs doit donc détecter si les instructions sont dépendantes pour pouvoir les réordonner. Aussi, le programmeur peut aider le processeur dans son travail en évitant au maximum les dépendances entre les instructions. 
La complexité apporté par le système d'exécution dans le désordre, peut être la source d'attaque comme la récente faille \textit{Meltdown} des processeurs Intel \cite{DBLP:journals/corr/abs-1801-01207}.


\begin{lstlisting}[language=C, caption=Exemples de dépendances entre deux instructions., float,floatplacement=H, label=code_dependances]
//----------- Read after Write -----------
int A, B = 0
A = 5;
B = A + 10;
//Result: B == 10 ou B == 15

//----------- Write after Read -----------
int A, B = 0
B = A + 10;
A = 5;
//Result: B == 10 ou B == 15 
 
//----------- Write after Write -----------
int B = 0
B = 10;
B = 15;
//Result: B == 10 ou B == 15
\end{lstlisting}






\paragraph{Prédiction de branchement}\label{sec:branch_predictor}
L'exécution dans le désordre fonctionne tant que suffisamment d'instructions sont disponible pour être exécutées. Cependant, lorsqu'un programme contient un branchement conditionnel, le processeur doit attendre que sa condition soit testée. Si cette condition utilise une variable modifiée dans les instructions précédentes le processeur doit attendre d'avoir le résultat du test pour connaître les futures instructions à exécuter. Pour maximiser l'utilisation du pipeline, le processeur peut essayer de prédire le résultat du test et ainsi continuer l'exécution. S'il s'est trompé sur la prédiction, il doit alors annuler les instructions déjà exécutées et reprendre l'exécution des autres instructions. Le temps nécessaire pour la reprise de l'exécution après une mauvaise prédiction dépend donc de la taille du \textit{pipeline} utilisé (plusieurs dizaines de cycles).
La prédiction de branchement peut être implémentée par deux méthodes.
Le processeur peut posséder un matériel spécifique, appelé prédicteur de branchement (\textit{branch predictor}), qui utilise des méthodes de statistiques. Lorsqu’un branchement est faux plusieurs fois d’affilés, il peut estimer qu'il y a une grande probabilité qu'il le soit aussi à l’itération suivante et éviter d’attendre le résultat du test pour continuer l’exécution. Le processeur peut aussi, à partir de l'adresse de destination, comprendre si la condition et le saut est utilisé dans une boucle ou si c'est un retour de fonction. Une mauvaise prédiction pouvant fortement impacter l'exécution, les processeurs implémentent des prédicteurs de branchement toujours plus complexes. Ce matériel représente une part non négligeable du processeur, et des travaux ont pour objectif d'en comprendre leur fonctionnement \cite{Milenkovic2002}.
La deuxième façon d'implémenter la prédiction est réalisé statiquement, par le compilateur. À la lecture du code, le compilateur peut deviner qu'une boucle ne verra sont branchement vrai qu'a la fin de son parcours. Il peut alors calculer le nombre d’itération à réaliser et éviter le test à chaque itération.
Comme pour le mécanisme d'exécution dans le désordre, la complexité apportée par le prédicteur de branchement a donné lieu à une importante faille de sécurité découverte en 2018 par les chercheurs de Google appelée Spectre \cite{kocher2018spectre}.






%%%%%%%%%%%%%%%%%%%%%%%%%%%%%%%%%%%%%%%%%%%%%%%%%%%%%%%%%%%%%%%%%%%
\subsection{Exécuter les instructions en parallèles} \label{sec:para}
%%%%%%%%%%%%%%%%%%%%%%%%%%%%%%%%%%%%%%%%%%%%%%%%%%%%%%%%%%%%%%%%%%%

La loi de Moore à assurer aux processeurs un gain constant de transistors chaque année. Ils peuvent être utilisés pour implémenter de nouvelles fonctionnalités matériels permettant d'exécuter les instructions en parallèles pour accélérer les applications. Les processeurs ont reçu de nombreuses améliorations dont les principales sont présentées dans cette section: 
\begin{itemize}
    \item Le pipeline
    \item Les processeurs superscalaire
    \item Les coeurs
\end{itemize}


\subsubsection{Le pipeline} \label{sec:pipeline}
%%%%%%%%%%%%%%%%%%%%%%


\paragraph{Motivations.} 

L'utilisation d'instructions CISC toujours plus complexes, a pour effet d'allonger le temps nécessaire à leur exécution qui dure alors plusieurs cycles. Les instructions complexes nécessitent plusieurs opérations: le chargement depuis la mémoire (\textit{fetch}), le décodage (\textit{decode}), le chargement des données nécessaire (\textit{memory}), l'exécution (\textit{execute}) et l'enregistrement du résultat (\textit{write-back}). Pendant ces différentes étapes, la totalité de l'unité d'exécution ne peut pas être utilisée simultanément et l'unité d'exécution n'est pas disponible pour les instructions suivantes. 

La chaîne de traitement du processeur, ou \textit{pipeline}, est une implémentation matériel d'un module qui permet de découper l'exécution d'une instructions en plusieurs étapes (\autoref{pic_pipeline_simple}). Cette technique peut être vu pas analogie à l'utilisation de chaîne de montage. Datant de plus d'un sciècle, la technique de la chaîne de montage a été abondamment utilisée par des industriels tels que Louis Renault et Henry Ford \cite{wolff1957entrepreneurs}.



\begin{figure}
    \center
    \includegraphics[width=10cm]{images/Chapitre1/Neumann.png}
    \caption{\label{pic_pipeline_simple} Représentation simplifié d'un pipeline de 5 étapes.}
\end{figure}


\paragraph{Implémentation}

En informatique, la technique de \textit{pipeline} est utilisée pour exploiter la parallélisme d'instructions (ILP) (\autoref{pic_pipeline}). Il est commun de présenter la notion de pipeline avec un pipeline de 5 niveaux:

\begin{itemize}
    \item \textbf{Recherche de l'instruction} ou \textit{fetch}: cette première étape charge l'instruction à exécuter depuis la mémoire principale dans un registre du processeur. Grâce à un compteur interne, le registre \textit{Program Counter}, le processeur connaît l'adresse mémoire de la prochaine instruction à charger. Pour améliorer le temps d'accès aux instructions, le processeur possède un tampon (\textit{instruction buffer}) contenant plusieurs instructions d'avance. Ce tampon permet l'implémentation d'optimisations tel que l'exécution dans le désordre (voir \autoref{sec:out_of_order}).
    \item \textbf{Décodage} ou \textit{décode}: une fois que l'instruction est chargée elle est décodée pour déterminer l'action à exécuter et les données nécessaires.
    \item \textbf{Execution} ou \textit{execute}: en fonction du décodage réalisé, l'instruction est exécutée: utiliser l'ALU pour faire une opération ou calculer une adresse.
    \item \textbf{Accès mémoire} ou \textit{memory}: réalise un accès mémoire (\textit{load} ou \textit{store}) lorsqu'une instructions le nécessite.
    \item \textbf{Ecriture du résultat} ou \textit{write back}: Enfin le processeur doit enregistrer le résultat produit par l'étape \textit{execute}. Si c'est un branchement, il modifie le registre \textit{Program Counter} (\textit{branch}). Si c'est une opération arithmétique il sauvegarde le résultat dans l'adresse destinataire décodé par la deuxième étape.
\end{itemize}

Le fait de partager l'exécution d'une instruction en sous étapes permet de commencer l'exécution de la suivante pendant que l'instruction actuelle est encore dans la chaîne d'exécution (voir figure ~\ref{pic_pipeline}). Son utilisation ne réduit pas le temps d'exécution d'une instruction (5 cycles sur la \autoref{pic_pip_no}). Celles-ci doivent tout de même passer une à une par chaque étape de la chaîne. Le \textit{pipeline} permet d'améliorer la cadence d'exécution en maximisant l'utilisation de chaque ressource à un même moment (\autoref{pic_pip_yes}). On peut par exemple commencer à charger la prochaine instruction (étape \textit{fetch}), alors que l'instruction actuelle est en train d'être exécutée (étape \textit{execute}). Sur la \autoref{pic_pip_yes} on assiste à l'exécution de 5 instructions, au premier temps un seul instruction est exécutée, à l'étape $IF$ pour \textit{instruction fetch}. Ensuite (ligne suivante) une nouvelle instruction est chargé (opération $IF$) pendant que la première est passé à l'étape suivante (opération $ID$). Ainsi au bout de 5 cycles, chaque étape du pipeline est utilisée (partie en verte). 


\begin{figure}
    \begin{subfigure}[]{0.5\linewidth}\centering
        \vspace{1.6cm}
        \includegraphics[width=\linewidth]{images/Chapitre1/pipelineNo.png}
        \label{pic_pip_no}
        \caption{Processeur sans pipeline}
    \end{subfigure}%
    ~ %space
    \begin{subfigure}[]{0.5\linewidth}\centering
        \includegraphics[width=.7\linewidth]{images/Chapitre1/pipelineYes.png}
        \caption{Processeur avec un pipeline à 5 étages}
        \label{pic_pip_yes}
    \end{subfigure}
    
    \caption{Pipeline: en séquençant les instructions le processeur est capable d'exécuter des étapes différentes en parallèles (\textit{IF: instruction fetch, ID: instruction decode, EX: execution, MEM: memory, WB: write back}). Le nombre de cycle nécessaire pour l'exécution de 3 instructions passe alors de 15 à 8 cycles (source: \url{https://fr.wikipedia.org/wiki/Pipeline_(architecture_des_processeurs)} }
    \label{pic_pipeline}
\end{figure}




\paragraph{Taille du pipeline.}
En 1939 IBM conçoit le premier processeur avec pipeline. Ce n'est qu'en 1989 qu'Intel produira le siens (Intel 80486). Le nombre d'étapes, ou profondeur du pipeline, était de 2 à l'origine et a augmenté au fil du temps atteignant 31 étapes pour l'architecture du Pentium 4 Prescott d'Intel en 2004. 

\paragraph{Complexité de la gestion du pipeline.}
L'utilisation d'un \textit{pipeline} n'est pas toujours optimale et plusieurs facteurs peuvent affecter sa performance. le principe du pipeline repose sur le concept de commencer à exécuter des instructions avant que la précédente ne soit terminée. Cela peut être rendu impossible par la dépendance entre deux instructions et par l'utilisation de branchements conditionnels \cite{emma1987characterization}. Lors de l'évaluation d'un tel branchement, le pipeline ne peut pas commencer à exécuter les instructions suivantes sans connaître son résultat. Le processeur doit alors attendre (\textit{stall}) plusieurs cycle avant de continuer. Une optimisation de prédiction de branchement a été implémentée pour éviter ces états de \textit{stall} (voir \autoref{sec:branch_predictor}). De plus, pour permettre la bonne utilisation du pipeline, des mémoires tampons doivent être disposées entre chaque étapes pour mémoriser les différents résultats intermédiaires. Lorsque le processeur exécute plusieurs processus, il doit veiller à terminer l'exécution des instructions avant de commencer celles du processus suivant. La complexité de sa gestion le rend vulnérable aux attaques informatiques (voir \autoref{sec:out_of_order}). 












\subsubsection{Processeur superscalaire} \label{sec:superscalar}
%%%%%%%%%%%%%%%%%%%%%%

Le pipeline apporte un niveau de parallélisation horizontale. Les processeurs ont reçus une autre amélioration apportant au pipeline une parallélisation verticale. 

\paragraph{Principe}
Un processeur est dit \textit{superscalaire} s'il est capable d'exécuter plus d'une instruction simultanément. Le nombre d'instruction par cycle d'holorge (IPC) peut alors être supérieur à 1. Le principe est d'implémenter un second pipeline (ou plus) capable d'exécuter les instructions tel que sur la \autoref{pic_pip_yes}. Intel proposa son premier processeur superscalaire en 1989 avec le processeur Intel 80486. Il possédait un pipeline à cinq étages proche de celui présenté dans la section précédente. 
Pour pouvoir l'utiliser, le processeur doit déterminer si deux instructions peuvent être exécutées en parallèle (sans dépendance et utilisant des ressources matériels différentes). La \autoref{pic_superscalar} montre comment une implémentation superscalaire du pipeline fonctionne. 

\begin{figure}
    \center
    \includegraphics[width=8cm]{images/Chapitre1/superscalar.png}
    \caption[Processeur superscalaire]{Fonctionnement d'un processeur superscalaire possédant deux pipelines.\protect\footnotemark. \label{pic_superscalar} }
\end{figure}
\footnotetext{source: \url{https://fr.wikipedia.org/wiki/Processeur_superscalaire}}

\paragraph{Implémentation}
Il existe deux façons de transformer un processeur scalaire en superscalaire. La première est de dupliquer matériellement chaque étape pour obtenir deux pipelines distincts (processeurs \textit{superpipeline}). On peut citer le processeur Intel Pentium dont la totalité du pipeline n'est pas dupliqué. Le processeurs à une fenêtre de plusieurs instructions prêtes à être exécutées. Seule la phase d'exécution est dupliquée.  Il possédait deux unités d'exécution \textit{u} et \textit{v} qui pouvait exécuter des instructions de types différents (opérations flottantes ou entières) en parallèle. 
Le deuxième moyen d'implémenter le parallélisme d'instruction d'un processeur superscalaire repose sur le fait qu'une étape peut nécessiter moins d'un demi cycle d'horloge pour être exécutée. Une étape (ou micro-instruction) du pipeline peut donc s'occuper de deux instructions différentes pendant un cycle d'horloge, en utilisant sa propre horloge interne. Cette méthode à le bénéfice de ne pas avoir a dupliquer le pipeline matériellement. 

Les principales limitations à l'implémentation d'un pipeline sont les dépendances et les conflits. Cela peut être une dépendance entre les données de plusieurs instructions ou un conflits d'accès à une même ressource (ALU, FPU). Le processeur est alors en charge d'orchestrer les différentes instructions pour rendre possible le parallélisme en utilisant des \textit{stratégie d'iméssion} \cite{johnson1989super} des instructions. Cette stratégie doit veiller à conserver la validité du programme en veillant aux ordres: de lecture des instructions, de leur exécution et de leur actualisation des registres (ou de la mémoire). 


\paragraph{Exemple du Pentium 4}
Pour bien comprendre le déroulement de l'exécution du pipeline d'un processeur superscalaire, nous choisissons de détailler le fonctionnement de processeur Intel Pentium 4 \cite{stallings2003organisation} donc le schéma de la micro-architecture est présenté sur la \autoref{cpu_superscalar_pentium}. Le processeur exécute les \textit{micro-ops} en utilisant un pipeline d'au moins 20 étages en veillant à respecter les dépendances, dont les principales étapes sont décrites ici: 

\begin{enumerate}
    \item Le processeur lit les instructions (CISC) depuis le cache L2 par groupe de 64 octets dans l'ordre du programme pour profiter de l'effet de localité. Bien que la prédiction de branchement puisse modifier cet ordre. 
    \item Chaque instruction (pouvant être de taille différente) est décodé et traduite en une à quatre instructions RISC de 118 bits (\textit{micro-ops}). \item Ces micro-ops sont ensuite stockées dans un buffer (\textit{Trace Cache}) permettant l'utilisation de l'exécution dans le désordre (voir \autoref{sec:out_of_order}).
    \item Ensuite, le processeur possède au renommage des registres. Il existe 16 registres architecturaux (utilisable pas le code) mais 128 registres physique sont réellement implémentés. Les \textit{micro-ops} peuvent ensuite être stockée dans deux listes d'attente distinctes utilisant une discipline \textit{FIFO}.
    \item L'ordonnanceur choisi ensuite dans les deux files les instructions qui possèdent leurs opérandes et dont l'exécution peut être réalisée. Suivant le type d'instruction elle sont envoyées (jusqu'à 6 à la fois) vers l'unité d'exécution correspondante (calcul entier ou flottant) en utilisant les différents ports. 
\end{enumerate}
     

\begin{figure}
    \center
    \includegraphics[width=13cm]{images/cpu_superscalar_pentium.png}
    \caption[Diagramme en bloc du Pentium 4]{Diagramme en bloc du Pentium 4 \cite{stallings2003organisation}
    \label{cpu_superscalar_pentium}}
\end{figure}



Les architectures actuelles otn beaucoup évoluées depuis le premier pentium. Le détail de la micro-architecture Skylake d'Intel peut être consultée sur la \autoref{pic:cpu_skylake_architecture}. L'unité d'exécution peut être utilisée par 8 \textit{ports} différents. Chaque port est relié à des composants différents de l'ALU qui peut exécuter jusqu'à 4 instructions par cycles (ou 2 opérations flottantes).






\subsubsection{Processeur multi-coeurs} \label{sec:multicore}
%%%%%%%%%%%%%%%%%%%%%%


\paragraph{Motivation.}
La vitesse de calcul d'un processeur est lié à sa fréquence qui a largement contribué à l'évolution de leur performance. Cependant, certaines limites physiques empêchent l'augmentation infinie des fréquences (discuté dans la partie \ref{sec:frequence}). Il a donc fallu trouver d'autres moyens d'améliorer la performance des processeurs, sans pouvoir accélérer leur fréquence. L'apparition des processeurs multi-coeurs est une réponse à ce challenge. Pour comprendre leur intérêt l'analogie suivante peut être utilisée \cite{tanenbaum2016structured}: la construction d'un processeur avec une fréquence de 1000 GHz est probablement impossible. Par contre l'utilisation de 1000 processeurs avec une fréquence de 1 Ghz est possible pour obtenir la même performance. Ce gain de performance peut alors être utilisé pour réduire la fréquence des processeurs. En réduisant la fréquence de 30\%, l'énergie nécessaire est elle réduite de 35\% \cite{mattsson2014haven}. En utilisant deux coeurs à 70\% de la fréquence initiale permet cependant d'obtenir un gain de 140\% de la puissance de calcul. C'est ce constat qui motive l'utilisation du parallélisme dans toute l'architecture d'un supercalculateur (voir \autoref{sec:parallelisme}). Ainsi, avant l'apparition des processeurs mutli-coeurs, l'utilisation du multitraitement symétrique (SMP) utilisant plusieurs processeurs en parallèles était le principale moyen d'accéder au parallélisme. Cependant, les serveurs devenant toujours plus gros, et avec le désir d'avoir des processeurs plus puissant pour les ordinateurs personnelles et les téléphones, les processeurs multi-coeurs ont été inventés.


\paragraph{Multi-coeur.}
Le terme de processeur multi-coeur est employé pour désigner tout processeur possédant entre deux et quelques dizaines de coeurs (on parle de processeurs \textit{manycore} au delà). Les différents coeurs sont disposés sur la même puce d'où l'autre appellation utilisée pour désigner ces processeurs de Chip Multiprocessor (CMP).
Les différents coeurs sont généralement identiques en tout point (processeur homogène) bien qu'ils puissent être différents (processeur hétérogène). Les processeurs homogènes sont plus faciles à utiliser car tous les coeurs peuvent répondre au même besoin et leur design est plus simple. Les processeurs hétérogènes sont cependant plus performant pour certaines application. 
Généralement, chaque coeur du processeur n'est pas relié directement à la mémoire. Un niveau de cache est généralement interposé entre le coeur et la mémoire. La hiérarchie mémoire est présentée dans la \autoref{sec:hierarchie}.

\begin{figure}
    \center
    \includegraphics[width=10cm]{images/cpu_multicore.jpg}
    \caption{\label{processeur_archi} Exemple de processeur multi-coeur (Intel Core i7-2600K)  partageant le troisième niveau de cache.\protect\footnotemark}
\end{figure}

\footnotetext{source: \url{https://www.anandtech.com/show/4083/the-sandy-bridge-review-intel-core-i7-2600k-i5-2500k-core-i3-2100-tested}}




L’avantage principale de dupliquer un coeur plutôt que de doubler la fréquence d’un seul coeur est la consommation électrique et donc la puissance dégagée par effet Joule. En effet, la puissance dissipée est quadruplée quand la fréquence est doublé alors qu'elle ne fait que doubler lorsque que le nombre de coeur est doublé. On obtient ainsi un processeur utilisant moins d’énergie et nécessitant moins de refroidissement pour une même performance qu'un processeur plus rapide.
Comme pour le pipeline (voir \autoref{sec:pipeline}), le gain de performance apporté par l’ajout de coeur s’appuie sur l’amélioration du parallélisme d'instruction (Instruction Level Parallélisme (ILP)).

La difficulté d’utilisation de processeurs multi-coeurs vient des programmes qui ne sont pas capables par nature d’utiliser ce niveau de parallélisme. Ils doivent donc être programmés pour pouvoir en profiter. Cette tâche, quoique difficile à ses début, est aujourd'hui facilitée par l’utilisation de librairies prévues telles que \textit{Pthread} ou \textit{OpenMP}. 

Les coeurs partageant des niveaux communs de cache (généralement le dernier), la bonne programmation des applications et l'implémentation d'une micro-architecture efficace sont alors des facteurs déterminant de la performance obtenue:
\begin{itemize}
    \item Pour maximiser l'utilisation des caches, il est primordiale de prévoir leur partages entre les différents coeurs pour minimiser les conflits. Des méthodes de placements plus ou moins efficaces peuvent alors être utilisées \cite{mazouz2011performance}: laisser le système d'exploitation peut donner des performances très variable entre deux exécutions identiques, alors que le placement manuel permet d'obtenir les meilleurs résultats. 
    \item Le choix du réseaux utilisé par les coeurs est alors important pour la performance des codes et repose principalement sur quatre paramètres \cite{peh2009chip} : la topologie, les algorithmes de routages, le protocole de contrôle de flux et le routage de la micro-architecture. La topologie indique comment les coeurs sont connectés et quels sont les chemins empruntable pas un message pour rejoindre sa destination. Ce choix est réalisé par l’algorithme de routage. Le contrôle de flux s’occupe de l'envoi des messages (ordre et date d’envoie) qui est ensuite réalisé par la micro-architecture. Les choix réalisés pour ces quatre paramètres ont un impact sur la performance du processeur (latence, bande passante) et sur son prix. 

\end{itemize}


Le nombre de coeurs par processeur a beaucoup évolué dans les quinze dernières années. Si les premiers processeurs multi-coeurs n'en possédaient que deux, il n'est pas rare que les supercalculateur utilisent des processeurs avec plus de vingt coeurs. Intel à même annoncé en 2019 un nouveau processeur doté de 56 coeurs \footnote{\url{https://ark.intel.com/content/www/us/en/ark/products/194146/intel-xeon-platinum-9282-processor-77m-cache-2-60-ghz.html}}. Cependant, l'ajout de coeurs supplémentaires n'est pas forcément bénéfique pour les applications à cause de la vitesse des mémoires qui peinent à évoluer au même rythme. Ce constat est discuté dans la \autoref{sec:memorygap}.





























%%%%%%%%%%%%%%%%%%%%%%%%%%%%%%%%%%%%%%%%%%%%%%%%%%%%%%%%%%%%%%%%%%%
\section{Hiérarchie mémoire} \label{sec:hierarchie}
%%%%%%%%%%%%%%%%%%%%%%%%%%%%%%%%%%%%%%%%%%%%%%%%%%%%%%%%%%%%%%%%%%%

\begin{fancyquotes}
Idéalement, on souhaiterait disposer d'une capacité de mémoire indéfiniment grande, de sorte qu'un agrégat particulier de 40 chiffres binaires, ou mot (cf. 2.3), soit immédiatement disponible, c'est-à-dire dans un temps légèrement ou considérablement plus court que le temps de fonctionnement d'un multiplicateur électronique rapide. On peut supposer que cela est pratique au niveau d'environ 100 m sec. Par conséquent, le temps de disponibilité d'un mot dans la mémoire doit être de 5 à 50 ms. Il est également souhaitable que les mots puissent être remplacés par de nouveaux mots à peu près au même rythme. Il ne semble pas physiquement possible d'atteindre une telle capacité. Nous sommes donc obligés de reconnaître la possibilité de construire une hiérarchie de mémoires, chacune d'entre elles ayant une plus grande capacité que la précédente mais moins rapidement accessible.\\
Traduit de \cite{burks1946preliminary}.

\end{fancyquotes}


\subsection{Motivations}
%%%%%%%%%%%%%%%%%%%%%%%%%%%%%%%%%%%%%%%%%%%%%%%%%%%%%%%%%%%%%%%%%%%


Dès 1946, les architectes des processeurs avaient anticipé que les applications seraient demandeuses de mémoires très performantes. A cause de la forte évolution du processeur, l'écart avec la performance des mémoires s'est creusé au fil des années (voir \autoref{pic:cpuvsmemory}). L'incapacité du processeur d'accéder suffisamment rapidement à la mémoire est inhérent à l'architecture actuelle des processeurs qui partage le même bus pour accéder aux instructions et aux données stockées en mémoire. Ce \textit{goulot d'étranglement} ou \textit{bottleneck}, a été nommé d'après l'un des architectes de l'architecture: le \textit{bottleneck Von Neumann}.


\begin{figure}
    \center
    \includegraphics[width=10cm]{images/cpu_vs_memory.png}
    \caption{\label{pic:cpuvsmemory} Progression de la performance des processeurs et des mémoires. Les processeurs ont vu leur performance évoluer de 50\% chaque année, contre 7\% pour les mémoires. L'écart de performance entre les deux matériels s'est creusé de 50\%  chaque année depuis les années 2000 (graphique extrait de \cite{AliSalehi2012}).}
\end{figure}

La réponse naïve à ce problème est de construire de grande mémoire à partir de SRAM, très performante et qui consomme peu d'énergie. Cependant, des contraintes économiques et technique sont à prendre en compte et rendent impossible cette solution. La mémoire SRAM est très cher à produire et nécessite l'utilisation de six transistors pour fonctionner, empêchant la construction de modules denses. Les constructeurs de processeurs ont dû élaborer une solution en prenant en compte la vitesse, la densité et le coût de chacune des technologies mémoires. Au plus une mémoire est rapide, au plus sont coût est élevé. Au plus la densité est élevée, plus le prix par bit stocké est réduit. Au plus la densité est élevé, au plus le temps d'accès est élevé. 






\subsection{Hiérarchie mémoire sur les processeurs récents}
%%%%%%%%%%%%%%%%%%%%%%%%%%%%%%%%%%%%%%%%%%%%%%%%%%%%%%%%%%%%%%%%%%%

La hiérarchie mémoire est la réponse économique et technique apportée aux contraintes évoquées ci dessus. Elle consiste en l'utilisation de différents modules mémoires de tailles, de technologies et de performances différentes. Son objectif peut être résumé à trois points: réduire le coût par bit stocké, augmenter la capacité de la mémoire la plus rapide, améliorer le temps d'accès aux données. La solution est de placer au plus proche du processeur des mémoires très rapide pouvant répondre instantanément aux accès mémoire. Au plus on s'éloigne des unités de calcul, au plus la latence d'accès aux modules mémoires augmente, mais au plus leur prix diminue rendant possible l'utilisation de module de plus grande capacité.


Les différents niveaux de mémoire peuvent être imbriqués, une donnée qui se trouve dans le premier niveau sera aussi stockée dans les modules de niveaux supérieurs.

Lorsque le processeur souhaite accéder à une donnée, il vérifie qu'elle se trouve dans son premier niveau de mémoire et, si ce n'est pas le cas, remonte la hiérarchie jusqu'à la trouver. La performance des applications varie fortement si les données nécessaires sont présentes ou non dans les mémoires proches du processeur. Les programmes doivent essayer de profiter du concept de localité présenté dans la section \autoref{sec:localite}.
\begin{figure}
    \center
    \includegraphics[width=14cm]{images/memory_hierarchy.png}
    \caption{\label{pic:cpuvsmemory} Hiérarchie mémoire}
\end{figure}





\subsection{Communication entre les différents niveaux}
%%%%%%%%%%%%%%%%%%%%%%%%%%%%%%%%%%%%%%%%%%%%%%%%%%%%%%%%%%%%%%%%%%%
Pour communiquer entre les différents niveaux de la hierarchie les données sont transmises par bloc de données de tailles différentes. L'avantage de tranférer les données par bloc et non une par une est d'amlélirer la performance des codes en tirant parti du principe de localité spatiale (voir \autoref{sec:localite}).
Pour accéder à un mot, le processeur a besoin que celle ci se trouve dans le niveau de cache L1. Lorsqu'elle si trouve, le processeur peut charger un mot directement dans ces registres pour y effectuer les opérations nécessaires. C'est la granularité de transfert la plus petite dans un ordinateur. 

Entre les différents niveaux de caches et entre le cache et la mémoire les données sont transférées par blocs appellées \textit{lignes de cache} ou \textit{cache line}. La ligne de cache contient une copie des données de la mémoire, un tag contenant des informations sur l'adresse mémoire du bloc de donnée et un \textit{flag} contenant des informations sur la validité de la ligne (voir \autoref{pic:cacheline}). La taille d'une ligne de cache peut varier d'une architecture à l'autre mais il est courant d'utiliser des tailles de 32, 64 ou 128 bytes. Une ligne de cache d'un processeur Intel récent mesure 64 bytes. Elle contient ainsi entre 8 et 16 éléments en double précision. 

\begin{figure}
    \center
    \includegraphics[width=8cm]{images/cacheline_def.png}
    \caption{\label{pic:cacheline} Représentation d'une ligne de cache.}
\end{figure}


Entre la mémoire, les blocs de données transférés sont de la taille d'une page (voir \autoref{sec:page}) qui mesure généralement entre 4 KiB et 2 MiB.



\subsection{Registres}
%%%%%%%%%%%%%%%%%%%%%%%%%%%%%%%%%%%%%%%%%%%%%%%%%%%%%%%%%%%%%%%%%%%
Les registres du processeurs sont situés au plus proche des unités de calculs. Pour permettre un accès rapide, 1 cycle, ils sont réalisés en SRAM. On compte entre x et y registres sur les processeurs récents. Leur taille est variable en fonction des instructions exécutables par les unités logiques arithmétiques. Par exemple, un processeur pouvant exécuter des instructions vectorielles AVX-512, possède des registres de 512 bits (registre \textit{ZMM}). Il existe différents types de registres, certains sont utilisés pour stocker des données et des résultats intermédiaires, tandis que d’autres ont une signification précise. Le registre de pointeurs de piles stocke l’adresse de la première adresse mémoire responsable de l’instruction actuellement exécutée et permettent de réaliser des appels et des retours de fonctions. Les registres des drapeaux (Flag Register) stockent des informations nécessaire à l’éxécution d’instructions. Par exmple, lorsqu’une retenue est générée par un calcul, ou qu’un branchement conditionnelle a été évalué à vrai. Les processeurs récents dupliquent certains registres pour pouvoir utiliser des techniques de renommage \cite{moudgill1993register} et d'exécution spéculative \cite{chou2004efficient}. Lors de l'exécution d'une instruction nécessitant plusieurs cycles, le processeur, si aucune dépendance n'est détectée, commence à exécuter les instructions suivantes sans attendre le résultat de la première instruction.






%%%%%%%%%%%%%%%%%%%%%%%%%%%%%%%%%%%%%%%%%%%%%%%%%%%%%%%%%%%%%%%%%%%
%%%%%%%%%%%%%%%%%%%%%%%%%%%%%%%%%%%%%%%%%%%%%%%%%%%%%%%%%%%%%%%%%%%
\subsection{Caches} \label{sec:cache}
%%%%%%%%%%%%%%%%%%%%%%%%%%%%%%%%%%%%%%%%%%%%%%%%%%%%%%%%%%%%%%%%%%%
%%%%%%%%%%%%%%%%%%%%%%%%%%%%%%%%%%%%%%%%%%%%%%%%%%%%%%%%%%%%%%%%%%%
%intro
C’est en 1965 que les premières mémoires caches sont présentées sous le nom de \textit{slave memory} \cite{wilkes1965slave}. Leur temps d’accès est de 4 à 20 fois plus rapide que celui de la mémoire principale servant alors de mémoire tampon. Cependant leur taille est très réduite (quelques MiB) comparée à celle de la mémoire principale (plusieurs GiB). Les cache utilisent généralement de la mémoire SRAM.

%performance
Ce niveau de mémoire a été implémenté pour réduire, du point de vu du processeur, l’écart de performance entre ses unités de calculs et celles de la mémoire centrale. Leur apport de performance vient de la capacité des programmes à réutiliser des données déjà présentes, évitant une accès à la mémoire centrale beaucoup plus long. Lorsque le processeur doit accéder à une donnée, il commence par la chercher dans le premier niveau de cache, si elle si trouve, son temps d'accés est très rapide (évenement \textit{cache hit}). Si ce n’est pas le cas (évenement \textit{chache miss}), il réalise alors une copie de la zone mémoire la contenant dans le cache. La zone mémoire copié est appelé \textit{ligne de cache}. Si par la suite, cette donnée ou une donnée appartement à la même ligne de cache devait être à nouveau accédée, leur temps d’accès serait alors drastiquement réduit. Ce mécanisme est transparent pour l’utilisateur, bien que pour des questions de performances il doive être conscient de son existence (voir \autoref{sec:localite}). 

%taille vs performance
La taille de chaque niveau de cache varie pour les raisons expliquées en introduction de cette partie. A cela vient s'ajouter la notion de performance qui est liée à leur taille. Un cache de grande capacité aura plus de chance de contenir la donnée dont le processeur à besoin, améliorant ainsi la performance moyenne du programme. Un cache est une mémoire associative. Pour accéder à son contenu il faut utiliser une clef associative. La clef est constitué de l’adresse en mémoire centrale de l’instruction ou de la donnée. Un cache plus grand nécessite plus de comparaisons pour vérifier si une donnée s'y trouve ou non. Pour allier les avantages et contourner les inconvénient, les processeurs utilisent non pas un, mais plusieurs niveaux de caches de tailles différentes.
Le premier niveau de cache est généralement séparé en deux zones mémoire: l'une contenant les instructions et l'autre les données. C'est le seul niveau de la hiérarchie qui stocke différemment les données et les instructions. Sur les processeurs récents, le premier et le deuxième niveau de cache est privé à chaque coeur. Un troisième, et parfois un quatrième niveau de cache est partagé entre les différents coeurs du processeur. La \autoref{pic:cache_hierarchy} représente une telle architecture pour un processeur à 4 coeurs. Le partage d’un ou plusieurs niveaux de caches entre différents coeurs à certains bénéfices en programmation parallèle. La communication entre les coeurs est plus rapide, ainsi que la migration d’un thread entre deux coeurs partageant un même niveau de cache. Cependant, cela introduit de la complexité pour la cohérence des caches (voir \autoref{sec:cache_coherence}).

\begin{figure}
    \center
    \includegraphics[width=8cm]{images/cache_hierarchy.png}
    \caption{\label{pic:cache_hierarchy} Organisation d'une hiérarchie de cache à trois niveau sur un processeur à 4 coeurs (source \cite{putigny2014benchmark}).}
\end{figure}




\subsubsection{Propriété d'inclusion}
%%%%%%%%%%%%%%%%%%%%%%%%%%%%%%%%%%%%%%%%%%%%%%%%%%%%%%%%%%%%%%%%%%%
Lorsqu'une donnée est chargé depuis la mémoire, le processeur doit la stocker dans le cache qui varie en fonction de la propriété d'inclusion du processeur pouvant être inclusive ou exclusive ou non-inclusive (voir \autoref{pic:cacheinclusionpolicy}).

Un cache est dit inclusif, si lorsqu'une donnée se trouve à un niveau de la hiérarchie, tous les caches des niveaux supérieur contiennent eux aussi une copie de la donnée (voir \autoref{pic:InclusivePolicy}). Cette politique d'inclusion a un désavantage lorsqu'elle est utilisée sur des système multi-coeurs. En effet, lorsqu'une donnée doit être retiré d'un niveau de cache, le processeur doit aussi l'enlever des niveaux de caches inférieurs. 
Un cache non-inclusif permet qu'une ligne du cache de niveau 1 ne soit pas forcément dans le cache de niveau 2. Cela permet d'augmenter la capacité de la hiérarchie de cache. Les processeurs récents implémente les deux politiques d'inclusion. Le processeur Intel Sandy Bridge a un cache L3 inclusif tandis que les cache L1 et L2 sont non-inclusif.

Pour une politique d'exclusion, une donnée qui se trouve à un niveau du cache, ne peut pas se trouver dans un autre niveau de cache au même moment (voir \autoref{fig:ExclusivePolicy}). L'avantage des caches exclusif est leur capacité de stocker plus de données car une ligne de cache ne se trouve jamais à deux endroits à la fois de la hiérarchie de cache. Cependant, lors d'un \textit{hit} dans le cache L2, le processeur doit échanger la ligne entre les deux niveau de cache L1 et L2, plus long qu'une simple copie.



\begin{figure}
    \centering
    \begin{subfigure}[b]{0.45\linewidth}
        \includegraphics[width=\linewidth]{images/InclusivePolicy.png}
        \caption{Inclusive}
        \label{pic:InclusivePolicy}
    \end{subfigure}
    ~ %add desired spacing between images, e. g. ~, \quad, \qquad, \hfill etc. 
      %(or a blank line to force the subfigure onto a new line)
    \begin{subfigure}[b]{0.45\linewidth}
        \includegraphics[width=0.85\linewidth]{images/ExclusivePolicy.png}
        \caption{Exclusive}
        \label{pic:ExclusivePolicy}
    \end{subfigure}
    \caption{Exemple de deux propriétés d'inclusion de la hiérarchie de cache (source \cite{wikipedia_2019}). }\label{fig:cacheinclusionpolicy}
\end{figure}




\subsubsection{Politique de placement: associativité}
%%%%%%%%%%%%%%%%%%%%%%%%%%%%%%%%%%%%%%%%%%%%%%%%%%%%%%%%%%%%%%%%%%%
La performance d'un cache ne vient pas seulement de la technologie utilisée pour sa construction. En effet, lorsqu'une donnée est accédée, le cache doit vérifier si la donnée est présente ou non dans le niveau de cache demandé. Il faut que l'algorithme de comparaison permettant de la trouver soit le plus rapide possible. Pour cela, les architectes utilisent généralement une fonction de \textit{hash} permettant d'attribuer un emplacement dans le cache en fonction de l'adresse mémoire de la \textit{cache line}. Si la \textit{cache line} ne se trouve pas à l'emplacement calculé, c'est quelle n'est pas présente dans ce niveau de cache.


\begin{figure}
    \centering
    \begin{subfigure}[b]{0.45\linewidth}
        \includegraphics[width=\linewidth]{images/cache_calcul.png}
        \caption{Calcul de l'emplacement (index de la ligne et décalage) pour un mappage direct}
        \label{pic:cache_calcul}
    \end{subfigure}
    ~ %add desired spacing between images, e. g. ~, \quad, \qquad, \hfill etc. 
      %(or a blank line to force the subfigure onto a new line)
    \begin{subfigure}[b]{0.45\linewidth}
        \includegraphics[width=\linewidth]{images/cache_direct.png}
        \caption{Emplacement de la ligne calculée dans le cache ou sera stockée la \textit{cache line}.}
        \label{pic:cache_direct}
    \end{subfigure}
    \caption{Exemple du calcul de l'emplacement de la ligne de cache lors d'un mappage direct à partir de l'adresse de la \textit{cache line} à stocker (source \cite{Meunier2017}). }\label{fig:cacheinclusionpolicy}
\end{figure}





Les trois politiques de placement les plus utilisées sont: le mappage  \textit{direct }, le mappage \textit{fully associative} et le mappage \textit{set assiociative} (voir \autoref{pic:cache_associativite}).


\begin{figure}
    \center
    \includegraphics[width=12cm]{images/cache_associativite.png}
    \caption{\label{pic:cache_associativite} En fonction de la politique de remplacement utilisée, une \textit{cache line} sera associée à une ligne du cache différente (source \cite{Meunier2017})}
\end{figure}

\paragraph{Cache à correspondance directe (\textit{direct-mapped cache})} utilise une fonction simple pour déterminer l'emplacement (index et offset) du cache à utiliser. Une partie des bits le l'adresse de la \textit{cache line} est utilisée pour déterminer la ligne à utiliser (\textit{index}) par exemple à l'aide d'une opération modulo. L'autre partie est utilisée pour déterminer le décalage dans cette ligne (\textit{offset}). Cette méthode est très rapide mais peut avoir des performances catastrophiques. Un algorithme faisant des sauts en mémoire d'une certaine taille pourrait n'utiliser qu'une seule ligne du cache, le rendant totalement inefficace. Comparé aux caches associatifs, le mappage direct est plus simple à implémenter car un seul comparateur est nécessaire pour déterminer la ligne de cache à utiliser (voir \autoref{fig:cache_schema}).

\paragraph{Cache pleinement associatif (\textit{fully associative cache})} remédie à ce problème en permettant à une \textit{cache line} d'être stockée à n'importe quel emplacement dans le cache. Cependant, cette technique à le désavantage d'être très lente. En effet, une \textit{cache line} pouvant se trouver à n'importe quelle ligne du cache, il faut toutes les comparer pour vérifier sa présence ou non. Pour faire cette comparaison en parallèle, il faudrait implémenter autant de comparateurs que de ligne dans le cache, complexifiant grandement le cache.


\paragraph{Cache N-associatif (\textit{N-way set associative cache})} permet de réduire le nombre de comparateurs nécessaires en regroupant les lignes du caches potentiellement adressable pour une \textit{cache line} en groupe (\textit{set}). L'exemple de la \autoref{pic:cache_associativite} utilise un cache à 4 set, appelé \textit{4-way associative}. Il ne faut plus que 4 comparateurs pour déterminer si une ligne appartient à un des \textit{set}. Les mappage par association sont plus lents que le mappage direct, car il faut trouver où se trouve la \textit{cache line} (si présente) dans un sous ensemble de ligne de cache plus ou moins grand. Pour accélérer la recherche de la présence ou non d'une ligne de cache, le traitement peut être réalisé en parallèle dans les différents \textit{sets} par l'utilisation de plusieurs comparateurs (4 dans la \autoref{pic:cache_circuit-set-associative}). 

La propriété principale d'un cache est sa capacité à conserver les données pour de futurs accès. Un cache de 8 MB à 2 set associatifs peu sauver jusqu'à 44\% des \textit{miss} comparé à un cache à correspondance directe \cite{Drepper2007}. Sur les architectures récentes Intel Skylake, les caches utilisent entre 8 et 16 associativités, pouvant varier entre les différents niveaux. 

\begin{figure}
    \begin{subfigure}[t]{\linewidth}\centering
        \includegraphics[width=0.7\linewidth, trim = 0cm 10cm 0cm 2cm]{images/cache_circuit-direct.png}
        \caption{Le cache pleinement associatif nécessite d'avoir un comparateur pour chacune des lignes du cache}
        \label{pic:cache_circuit-direct}
        \vspace{1cm}
    \end{subfigure}
    
    \begin{subfigure}[t]{\linewidth}\centering
        \includegraphics[width=0.7\linewidth]{images/cache_circuit-fully-associative.png}
        \caption{Le cache pleinement associatif nécessite d'avoir un comparateur pour chacune des lignes du cache}
        \label{pic:cache_circuit-fully-associative}
        \vspace{.5cm}
    \end{subfigure}
    
    \begin{subfigure}[t]{\linewidth}\centering
        \includegraphics[width=0.7\linewidth]{images/cache_circuit-set-associative.png}
        \caption{Le cache N-associatif associe les deux architectures des caches direct et pleinement associatif. Ce sont plusieurs caches directs montés en parallèles.}
        \label{pic:cache_circuit-set-associative}
    \end{subfigure}
    
    \caption{Schéma des trois modèles de cache utilisés (tirés de l'ouvrage \cite{Blanchet2013}).}\label{fig:cache_schema}
\end{figure}




\subsubsection{Politique de remplacement}
%%%%%%%%%%%%%%%%%%%%%%%%%%%%%%%%%%%%%%%%%%%%%%%%%%%%%%%%%%%%%%%%%%%
Que ce soit pour le mappage \textit{fully associative} ou \textit{set associative}, une \textit{cache line} peut être stockée dans plusieurs ligne du cache. Pour déterminer laquelle choisir pour y placer la \textit{cache line}, différentes stratégies peuvent être utilisées, appelées \textit{politique de remplacement}. L'objectif de ces politiques est de maximiser l'utilisation du cache en prévoyant et en anticipant les futures lignes à être accéder pour ne pas les supprimer du cache. Il existe de nombreuses politiques de remplacement, chacune ayant ses avantages et ses inconvénients \cite{wikipedia2_2019}: FIFO, LIFO, LRU, TLRU, MRU, PLRU, RR, SLRU, LFU, LFRU, LFUDA, LIRS, ARC, CAR, MQ. La politique choisie à un réelle impacte sur les performances de l'application. Pour la choisir, il faut trouver un compromis entre la performance la complexité apportée par la mise en place de la politique choisie. Dans le cas du mappage \textit{direct}, la ligne de cache présente à l'emplacement calculé est forcément remplacée et ne nécessite pas d'avoir une politique de remplacement. Il existe deux familles de politiques de remplacement. La première famille regroupe les politiques de remplacements qui tiennent compte de l'utilisation des \textit{cache line} (LRU, FIFO). Ce sont généralement les politiques les plus efficaces. La deuxième famille est celle des politiques aléatoires (\textit{random}, \textit{round robin}) qui ne tiennent pas compte de l'utilisation des données et choisissent une ligne aléatoirement à remplacer. Ces politiques sont performantes en termes de rapidité d'exécution car le choix ne se fait que sur une fonction aléatoire. Cependant, \cite{Al-Zoubi:2004:PEC:986537.986601} montrent que ces techniques utilisent 22\% moins bien le cache, impactant fortement les performances des applications.


\paragraph{Least Recently Used} ou LRU remplace la ligne de cache la moins récemment utilisée. Les numéros des lignes utilisée sont stockée dans une pile suivant la date de leur dernière utilisation. La pile est mise à jour lorsqu'une nouvelle donnée est stockée dans le cache en empilant son adresse au sommet de la pile. De même, lors d'un \textit{hit}, la ligne de cache référencé est stockée elle aussi en sommet de pile. Cette méthode à un inconvénient pour certains type d'accès, notamment les parcours de tableau. Imaginons un cache pouvant contenir 4 lignes. Le double parcours d'un tableau mesurant 5 ligne de caches a,b,c,d,e aura les accès mémoire suivant: a,b,c,d,e a,b,c,d,e. Le deuxième accès au tableau ne profitera pas du cache car chaque ligne de cache est remplacée au fur et mesure du parcours du tableau. Des améliorations ont été apportées pour corriger ce problème, comme l'introduction d'un répertoire image \cite{Stone:1987:HCA:31845} qui garde une trace des groupes de lignes de cache utilisées ensemble pour prévoir les accès similaires et anticiper leur accès.






\subsubsection{Stratégie de cache: lecture et écriture}
%%%%%%%%%%%%%%%%%%%%%%%%%%%%%%%%%%%%%%%%%%%%%%%%%%%%%%%%%%%%%%%%%%%

Le cache est une zone mémoire qui évolue en fonction des accès mémoires. Son fonctionnement lors d'un accès (en lecture ou écriture) peut varier en fonction de la présence (\textit{hit}) ou non (\textit{miss}) de la donnée et de la stratégie de cache implémentée.

\paragraph{Lecture:}  Lorsque le processeur accède à une donnée, il vérifie qu’elle n’est pas présente dans ses différents niveaux de cache. Si la donnée est présente, son accès est très rapide. Si ce n’est pas le cas, il réalise alors une copie de la zone mémoire la contenant dans le cache (la taille de la zone est une ligne de cache).

Lire en fonction des inclusion exclusion

\paragraph{Écriture:} la comportement du cache lors d'une écriture dépend de la présence ou non de la \textit{cache line} et la politique employée.

Si la ligne de cache n’est pas présente (\textit{miss}) dans le cache, deux solutions sont possibles. La première est de charger la ligne depuis la mémoire et d’y apporter les modifications (politique \textit{Write-Allocate}). La deuxième solution est d’écrire la ligne de cache sans la charger (politique \textit{No-Write-Allocate}). La ligne de cache ne sera chargé que lors d’un miss lors d’une lecture, sauf si la totalité de la ligne a été écrite, les données originales n'ayant plus de valeur utile. Cette option peut être intéressante si un algorithme ne fait qu’écrire dans un structure de donnée sans ne jamais la lire.

Si la ligne de cache est présente (\textit{hit}) dans le cache, deux solutions sont possibles. La première est de mettre à jour la \textit{cache line} dans le cache et en mémoire pour que le changement soit répercuter sur l'ensemble de la hiérarchie mémoire (politique \textit{Write-Through}). Cette politique peut être pénalisante si le processeur effectue consécutivement la mise à jour d'une donnée (par exemple un compteur, ou un index de bouble).  La seconde solution est de différer l'écriture à plus tard (politique \textit{Write-Back}). L’écriture est effectuée seulement dans le cache et ne sera effective en mémoire seulement lorsque la ligne de cache modifiée sera évincé du cache. La ligne de cache modifiée est alors indiqué grâce à un bit indicatif (\textit{dirty bit}). Comparé à la première méthode, celle ci utilise moins de bande passante car les mises à jour en mémoire sont moins fréquentes. Cependant, si plusieurs coeurs utilisent la même donnée, sa valeur pourrait alors être différente entre leurs caches respectifs (donnée périmée). Il faut alors implémenter un protocole de cohérence de cache entre les différents cache et la mémoire.

Les politiques utilisées lors d'un \textit{miss} ou d'un \textit{hit} peuvent être associés. Les combinaisons les plus utilisées sont \textit{Write-Through} + \textit{No-Write-Allocate} et \textit{Write-Back} + \textit{Write-Allocate}.






\subsubsection{Cohérence de cache} \label{sec:cache_coherence}
%%%%%%%%%%%%%%%%%%%%%%%%%%%%%%%%%%%%%%%%%%%%%%%%%%%%%%%%%%%%%%%%%%%

La stratégie employée lors de la modification d'une donnée introduit une challenge majeur des architectures multi-coeurs qui est de garantir la cohérence des données entres les différentes zones mémoires. Lors d'un accès mémoire, on souhaite accéder à la valeur sa plus récente, qui aura pû être modifiée par un autre coeur, ou processeur. La gestion de la cohérence d'un processeur à un seul coeur est plus simple, bien qu'elle doive tout de même être implémentée. Les opérations d'entrée-sortie peuvent affecter des données en mémoire qui se trouve aussi dans les caches.

Le protocole de cohérence de cache est responsable de vérifier qu'une même ligne de cache présente à plusieurs emplacement de la mémoire soit identique. Il doit pour cela garantir trois trois points. Le premier est de partager le changement d'une valeur à tous les coeurs d'un processeur pour que l'ordre des opérations affecté à une valeur soit vu dans le même ordre par tous les coeurs/processeurs. Le deuxième point est d'assurer que le résultat ne dépende que de l'ordre des instructions du programme assembleur et non de l'ordre de leur exécution par les différents coeurs. Enfin, le protocole doit assurer à un coeur qui lit une donnée que sa valeur est bien la dernière qui a été écrite (par un autre coeur ou autre processeur). La notion d'ancienneté peut être défini de plusieurs façons et le protocole doit la définir précisément pour assurer la validité des résultats \cite{Blanchet2013}. En effet, l'ordre peut faire référence à l'ordre des instructions dans le programme source. L'ordre peut aussi faire référence à celui de la fin des exécutions des résultats (avant que la donnée soit effectivement écrites). Enfin, ce peut être l'ordre des écritures mémoires. Comme la durée de propagation des écritures n'est pas constante dans le système, des erreurs peuvent apparaître si un protocole venait à utiliser ce dernier.

Comme le résume \cite{Blanchet2013}, les deux propriétés principales d'un protocole de cohérence sont sa simplicité de mise en oeuvre et sa performance. Pour assurer la cohérence, deux familles de protocoles existent, suivant si la gestion de cohérence est répartie sur les différents caches (\textit{locale}), ou si elle est centralisée (\textit{globale}). 



\paragraph{Protocoles locaux - cohérence répartie}

Les protocoles \textit{locaux} utilisent des outils de scrutation (\textit{snooping}) et de signalisation (\textit{broadcasting}). Implémentés directement dans les caches, ils ne nécessitent pas la modification ni de la mémoire ni du processeur. Lorsqu'une \textit{cache line} est modifié dans un cache, il obtient la copie exclusive de celle ci en invalidant ses copies dans d'autres caches (\textit{Write-Invalidate}). Une seconde option vise à simplement signaler la modification de cette ligne aux autres caches pour qu'ils mettent à jour leur structure de donnée (\textit{Write-Update}).  Différents protocoles de cohérence ont été implémentés et ont évolué. Les plus connues sont les protocoles MESI (ou \textit{Illinois}) \cite{papamarcos1984low} et MOESI. Mais il en existe beaucoup d'autres: MSI, MOSI, MERSI, MESIF. MESI et MOESI sont notamment très utilisés dans les processeurs multi-coeurs car il implémente des stratégies à écriture différée, minimisant le traffic mémoire.
\\
Nous présentons le protocole \textit{MOESI} à titre d'exemple. \textit{MOESI}  permet à une ligne de cache d'avoir cinq états différents. Le passage entre les différents état est résumé dans la \autoref{pic:moesi}. Chaque coeur surveille toutes les commandes effectuées sur le bus pour mettre à jour l'état de ses lignes ou les communiquer quand il en est propriétaire.

L'état $M$ (\textit{modified)} indique que la ligne est valide et qu'elle a été modifiée dans ce niveau de cache et qu'elle est seulement présente dans ce cache. La valeur en mémoire n'est pas cohérente, la ligne doit alors être copiée en mémoire lors de son remplacement. 

L'état $O$ (\textit{owned} ou \textit{shared-modified}) indique que cette ligne est valide est qu'elle est présente dans au moins un autre niveau de cache. Le cache actuel est \textit{propiétaire} de cette ligne, il doit informer les autres caches lors de sa modification. La ligne modifiée peut ensuite être communiqué à un autre niveau de cache, sans avoir à passer par la mémoire. Cet état est la principale amélioration apporté par le protocole \textit{MOESI} au protocole \textit{MESI}.


L'état $E$ (\textit{exclusive}) indique que la ligne est valide uniquement dans ce niveau de cache. Cela évite l'émission d'invalidation aux autres caches qui ne détiennent pas cette ligne. De plus, sur un autre cache y accède, la ligne de cache peut directement être transférée depuis le cache sans accès mémoire. La ligne dans le premier cache passera alors de l'état $E$ à $O$. Dans le deuxième cache la ligne sera en état $S$.

L'état $S$ (\textit{shared}) indique que la ligne est valide dans le cache courant et dans au moins un autre cache. Le cache actuel n'est pas propriétaire de la ligne (état $O$). La cohérence avec la mémoire n'est pas assumée. 

L'état $I$ (\textit{invalid}) indique que la ligne n'est pas valide. La lecture de cette ligne est interdite.


\begin{figure}
    \center
    \includegraphics[width=10cm]{images/moesi.png}
    \caption{\label{pic:moesi} Fonctionnement du protocole MOESI (source \cite{Sayin2014})}
\end{figure}





\paragraph{Protocole globaux - Cohérence par répertoire (\textit{directory based coherence}}

La seconde famille regroupe les protocoles dits \textit{globaux} utilisent des répertoires et des controleurs émettant les commandes de transferts des lignes de cache (entre les caches ou avec la mémoire)\cite{tang1976cache}. Toutes les informations nécessaire à la gestion de la cohérence sont enregistrées dans un répertoire. Leur performance est meilleure que les protocoles utilisant des techniques de \textit{snooping} et \textit{broadcasting} car ils génèrent moins de trafic. Bien que les protocoles tel que MOESI réduise le trafic mémoire en utilisant des écritures différés, la gestion des cinq états est complexe. Et le trafic généré par la cohérence de cache augmente fortement avec le nombre de coeurs utilisés et peu rapidement voir ses performances s'effondrer \cite{liu2016protocoles}. Les futures architectures à mémoire partagée nécessiteront d'implémenter des protocole de cohérence de cache très performant  \cite{al2010snoopy}.


















%%%%%%%%%%%%%%%%%%%%%%%%%%%%%%%%%%%%%%%%%%%%%%%%%%%%%%%%%%%%%%%%%%%
%%%%%%%%%%%%%%%%%%%%%%%%%%%%%%%%%%%%%%%%%%%%%%%%%%%%%%%%%%%%%%%%%%%
\subsection{Mémoire principale}
%%%%%%%%%%%%%%%%%%%%%%%%%%%%%%%%%%%%%%%%%%%%%%%%%%%%%%%%%%%%%%%%%%%
%%%%%%%%%%%%%%%%%%%%%%%%%%%%%%%%%%%%%%%%%%%%%%%%%%%%%%%%%%%%%%%%%%%












%%%%%%%%%%%%%%%%%%%%%%%%%%%%%%%%%%%%%%%%%%%%%%%%%%%%%%%%%%%%%%%%%%%
\section{Mémoire virtuelle} \label{sec:memoire_virtuelle}
%%%%%%%%%%%%%%%%%%%%%%%%%%%%%%%%%%%%%%%%%%%%%%%%%%%%%%%%%%%%%%%%%%%


%%%%%%%%%%%%%%%%%%%%%%%%%%%%%%%%%%%%%%%%%%%%%%%%%%%%%%%%%%%%%%%%%%%
\subsection{Contexte et historique de la gestion mémoire}
%%%%%%%%%%%%%%%%%%%%%%%%%%%%%%%%%%%%%%%%%%%%%%%%%%%%%%%%%%%%%%%%%%%

La mémoire est une des ressources les plus importantes des architectures modernes. Sa gestion doit être la plus performante possible  si l'on souhaite minimser au maximum le trou de performance séparant les mémoires et les processeurs. Si la hiérarchie de mémoire est la réponse matérielle à ce challenge la mémoire virtuelle est une réponse logicielle. Avant de la présenter en détail, cette introduction à pour but de motiver son utilité.

\subsubsection{Utilité de l'abstraction de la mémoire}
%%%%%%%%%%%%%%%%%%%%%%%%

Sans abstraction mémoire, tous les programmes et le système d'exploitation partagerai le même espace d'adressage. Cette implémentation, utilisée par les premières architectures, a deux inconvénients majeurs. Le premier concerne la sécurité de l'exécution d'un programme. S'il venait à écrire dans une zone mémoire réservée au système d'exploitation, un arrêt brutal du système pourrait survenir. De plus, lors de l'exécution de plusieurs processus sur le même processeur, deux programmes différents pourraient accéder et/ou modifier des données ne lui appartenant pas. Une solution pour contourner ce problème est d'alterner l'exécution de chaque processus en vidant et chargeant ses données depuis le stockage, engendrant le deuxième inconvénient d'un système sans abstraction mémoire: la performance. Bien que des threads puisse tout de même être utilisés (ils appartiennent au même processus et ont accès au même espace mémoire) l'utilisation de cette architecture serait très impactée. Par exemple, un utilisateur ne pourrait pas avoir plusieurs fenêtre exécutant des programmes différents en parallèle. L'utilisation de serveurs multi-utilisateurs ne serait alors même pas envisageable. L'absence d'abstraction mémoire, ou adresse direct, ne trouve d'application aujourd'hui que dans les systèmes embarqués. Le constructeur du système est généralement le seul utilisateur du processeur et est donc maître de son utilisation et peut réaliser des allocations mémoires manuellement.

\paragraph{L'abstraction par réallocation statique} a été implémentée sur l'ordinateur IBM 260 en 1965 \cite{Britannica} pour permettre l'exécution simultanée de plusieurs processus. Le système d'exploitation alloue une adresse de base à chaque processus. Lorsqu'une processus réalise un accès mémoire, un matériel s'occupait de décaler tous ses accès mémoires à partir de l'adresse de base. Ce mécanisme, invisible pour le programmeur était fonctionnelle mais impactait les performances du programme. Elle pouvait s'avérer complexe à mettre en place car il fallait distinguer les adresses à convertir et celle ne le nécessitant pas (un saut en mémoire par exemple).

\paragraph{L'abstraction de l'espace d'adressage} permet de donner son propre espace d'adresses à chaque processus indépendant les uns des autres. L'allocation dynamique permet de mapper l'espace mémoire d'un processus à un espace physique de la mémoire en utilisant deux registres \textit{base} et \textit{limite} comme sur le processeur Intel  8088. La méthode de \textit{va-et-vient} ou \textit{swapping} peut être utilisée pour gérer les déplacements des processus entre la mémoire et le stockage. Cette méthode est illustrée dans la \autoref{pic:memory_swapping}. L'inconvénient de cette méthode est la création de trous dans la mémoire, empêchant son utilisation optimale.  Des techniques de compactage ont été alors élaborées, mais était souvent très coûteuses (5s pour compacter 1GB de mémoire \cite{tanenbaum2008systeme}). De plus cette méthode ne permet pas de gérer les grands logiciels dont la taille ne permet pas d'être stockés en intégralité. Bien que des techniques utilisant les segments de recouvrement (\textit{overlays}) \cite{sherman1992method} aient permis  d'adapter le \textit{va-et-vient} a ces grand processus, la technique adoptée depuis est connue sous le terme de \textit{mémoire virtuelle}.


\begin{figure}
    \center
    \includegraphics[width=10cm]{images/memory_swapping.png}
    \caption{\label{pic:memory_swapping} Technique de va-et-vient pour gérer la mémoire dynamiquement. Les processus A, B, et C sont créés dans les étapes (a), (b) et (c). Lors de la création d'un processus D à l'étape (d), le système d'exploitation doit enlever un processus de la mémoire pour lui faire de la place. Lors de l'étape (f) et (g), le processus B laisse sa place pour que A puisse continuer son exécution. Entre l'étape (c) et (g) le processus A est exécuté à partir de deux espace d'adressage physique différents (graphique extrait de \cite{tanenbaum2008systeme})}.
\end{figure}





%%%%%%%%%%%%%%%%%%%%%%%%%%%%%%%%%%%%%%%%%%%%%%%%%%%%%%%%%%%%%%%%%%%
\subsection{La pagination}
%%%%%%%%%%%%%%%%%%%%%%%%%%%%%%%%%%%%%%%%%%%%%%%%%%%%%%%%%%%%%%%%%%%


\subsubsection{Motivations}
%%%%%%%%%%%%%%%%%%%%%%%%
La mémoire virtuelle a été implémentée pour gérer de façon efficace des processus dont la taille est plus grande que l'espace mémoire disponible. La seconde motivation était de gérer efficacement la mémoire lorsque plusieurs processus donc la somme des tailles dépasse l'espace mémoire  disponible. En d'autre terme, il fallait un mécanisme permettant l'exécution d'une programme sans qu'il soit chargé en totalité en mémoire. La solution devait aussi permettre de gérer facilement les changements de taille des processus de façon efficace, sans avoir à recopier la totalité du programme lors d'une allocation mémoire (\textit{malloc)}. Enfin, la mémoire virtuelle doit assurer la sécurité de l'exécution de plusieurs programme sur une même architecture en évitant les bugs et les vols de données.


\subsubsection{Les pages}
%%%%%%%%%%%%%%%%%%%%%%%%
Le principe de la mémoire virtuelle repose sur le principe de donner à chaque processus sont propre espace d'adressage mémoire. Chaque processus peut travailler sur l'adresse \textit{0x100}, car en réalité le mécanisme de mémoire virtuelle fait correspondre cette \textbf{adresse virtuelle} à différentes \textbf{adresse physique}. Pour cela, son \textbf{espace d'adressage virtuelle} est découpé en petites entités appelées \textbf{pages} qui contiennent un \textbf{espace d'adressage physique} contiguës. Chaque page est \textit{mappée} sur des adresses physiques (aussi contiguës) formant un \textbf{cadre de page} (\textit{page frame}). Une page et son cadre de page associé contiennent le même nombre d'adresses. Deux pages contiguës ne correspondent pas forcément à deux cadres de pages contiguës. Les pages et les cadres de pages ont la même taille qui est choisi par le système d'exploitation à son démarrage. Ces concepts sont résumés dans la \autoref{pic:memory_page_frame}. La page 2 contient les adresse virtuelles allant de l'adresse $0$ à l'adresse $4095$. Lorsque le processus propriétaire de cette page réalise un accès à cette adresse virtuelle, il réalise sans le savoir un accès aux adresses se trouvant entre $8192$ et $12287$. Ni la mémoire, ni le processeur n'ont connaissances de cette traduction qui est réalisée par un module matériel indépendant appelé \textit{Memory Management Unit} (MMU) (voir \autoref{sec:mmu}). 

\begin{figure}
    \center
    \includegraphics[width=10cm]{images/memory_page_frame.png}
    \caption{\label{pic:memory_page_frame} Correspondance entre les adresse virtuelles, stockées dans des pages, et les adresses physiques, stockées dans des cadres de pages.  \cite{tanenbaum2008systeme})}.
\end{figure}


\subsubsection{La taille des pages}
%%%%%%%%%%%%%%%%%%%%%%%%%%%%%%%%%%%
Les transferts de données entre la mémoire et le stockage se font par page. Ainsi une page ne peut se trouver à la fois en mémoire et sur le disque. En fonction des applications et de l'algorithme de remplacement de pages (voir \autoref{sec:deplacement_page}) ces transferts peuvent être fréquents. Le choix de la taille de page doit alors être pris en considération pour obtenir les performances de l'application attendus.
Plus la taille des pages est petite, plus l'utilisation effective de la mémoire sera proche de la quantité mémoire disponible. Avec de grandes pages, les processus n'en utilisant qu'une faible partie, réduise la mémoire disponible pour les autres processus. Si les \textit{défauts de pages} sont fréquent, la quantité de mémoire à déplacer entre la mémoire et le stockage et d'autant plus grande.
Aujourd'hui, les systèmes d'exploitation utilisent des page mesurant $4 KiB$. Cette taille est un bon compromis entre la gestion complexe de la table des pages qui doit être parcouru le moins souvent possible, et la meilleure gestion de la mémoire possible. 
Cependant, les systèmes d'exploitation récents permettent d'utiliser des tailles de pages plus grandes pour certaines applications qui pourraient en bénéficier. Ces grandes pages ou \textit{large pages} ou \textit{huge pages}, sont des pages de taille allant de 2 MiB à plusieurs GiB et peuvent être allouées de deux façons. La première est transparente pour l'utilisation. C'est le système d'exploitation qui analyse les accès et \textit{comprend} que le jeux de données accédé est grand et que l'application pourrait profiter de l'utilisation de grande page. Ce mécanisme est appelé \textit{Transparent Huge Pages} (THP) car il est géré automatiquement par le système \cite{TODO}. Le deuxième façon d'allouer des grandes pages est de la réaliser manuellement dans le code:
https://github.com/torvalds/linux/blob/master/tools/testing/selftests/vm/hugepage-shm.c








%%%%%%%%%%%%%%%%%%%%%%%%%%%%%%%%%%%%%%%%%%%%%%%%%%%%%%%%%%%%%%%%%%%
\subsection{Memory Management Unit (MMU)} \label{sec:mmu}
%%%%%%%%%%%%%%%%%%%%%%%%%%%%%%%%%%%%%%%%%%%%%%%%%%%%%%%%%%%%%%%%%%%




\subsubsection{Déplacement de page} \label{sec:deplacement_page}
%%%%%%%%%%%%%%%%%%%%%%%%%%%%%%%%%%%
Lorsque la demande mémoire est supérieur à l'espace disponible, la totalité des processus ne peut pas y être stocké. Ainsi, la totalité des pages allouées ne se trouvent pas toute en mémoire à un instant donné. Pour cela, la MMU tient une liste des pages se trouvant en mémoire en utilisant pour chaque page un bit de présence/absence. Dans l'exemple de la \autoref{pic:memory_page_frame}, si un processus à accéde à l'adresse $33000$, la MMU consuable dl

 sa te page et constate que la page n'est pas présente en mémoire. Cet évenement est appelé un \textbf{défaut de page} (\textit{page fault}). La système doit alors déplacer une page de la mémoire vers le stockage pour faire de la place pour cette nouvelle page. Le choix de la page à déplacer se fait grâce à un algorithme de remplacement de page. De la même façon que pour la gestion des lignes de cache, il faut une méthode efficace de remplacement pour éviter de remplacer une page qui sera accédée par l'instruction suivante. Plusieurs algorithmes existent: First In First Out (FIFO) remplace la page la plus vieille, Not Recently Used (NRU) remplace la page non utilisée depuis longtemps, Seconde Chance implémente l'algorithme FIFO mais cherche en priorité une page non-référencée. Pour des applications réalisant des accès mémoire sur plusieurs pages, l'algorithme choisi peut avoir un fort impacte sur sa performance.


te

\input{chapter/PART-StateOfTheArt/chapter-materiel/section-accelerateur}

\input{chapter/PART-StateOfTheArt/chapter-materiel/section-cluster}

\section{Discussion et conclusion}\label{sec:materiel_conclusion}


\textbf{todo: faire la conclusion}


    \begin{lstlisting}
    
	Forte evolution des performances des processeurs
		La complexite amene les failles
			Branch predictor
			Out of order.
		Schema avec l'evolution des transistors, frequence, nombre de coeurs etc...
	Difference de performance entre le processeur et la memoire
	Les problematiques techniques
		Mur de la memoire
		Fin de Moore
		Loi de Denard
	Besoin de disruption
		Repenser les architectures
		Nouvelles memoires
		Motiver et faire le lien avec le chapitre presentant ces opportunite
			GenZ
			Nouvelles memoires
    La memoire est la ressource limitante
		Des techniques permettent de profiter de la localite des donnees
		La bonne utilisation du bus est primordiale

	\end{lstlisting}
