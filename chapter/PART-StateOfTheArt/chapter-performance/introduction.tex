\section{Introduction}\label{sec:edl_perf_intro}

 
https://patents.google.com/patent/US9111032B2/en  
    -A bottleneck is a region of a program (e.g., program code) where significant execution time is spent. Typically, software developers use a profiling tool to collect program execution profiles such as the timing information for each method, routine, process, etc. With the help of such a profiling tool, the developer can sort, e.g., methods by the time spent on them. The methods that consume an amount of time greater than a threshold defined by, e.g., the developer may be treated as bottlenecks. However, the effectiveness of this approach depends on the program's runtime characteristics. Many programs, such as large enterprise commercial applications, do not have obvious bottlenecks. Therefore, their profiles contain a large amount of routines, processes, or methods where the execution time is spent relatively evenly (e.g., within a threshold). This type of profile is often referred to as a “flat profile” because no method dominates the execution time.
    

%% ANALYSE DE PERF %%

STATIC VS DYNAMIQUE

Intel propose différents outils tel que Intel Architecture Code Analyzer (IACA) \cite{Hirsh2012} qui permet de réaliser une analyse statique d'un code grâce à l'ajout de marqueur dans le code source. Il permet, pou un noyau identifié, de détecter la présence de dépendance entre plusieurs itérations de boucle et de donner une estimation des performances (débit d'instructions, saturation des ports de l'ALU...). Cepedant, il est nécessaire d'avoir identifié les zones de \textit{hot spots} pour les annoter, et il ne fonctionne que pour des architectures Intel. Le projet a été abandonné en avril 2019\footnote{Intel IACA - \url{https://software.intel.com/en-us/articles/intel-architecture-code-analyzer}}

--> on préfère dynamique à cause de la complexité des architectures qui rend défificile l'estimatique en statique
