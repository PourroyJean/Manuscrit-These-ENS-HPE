\section{Introduction au HPC}\label{sec:hpc}







%% ANALYSE DE PERF %%

STATIC VS DYNAMIQUE

Intel propose différents outils tel que Intel Architecture Code Analyzer (IACA) \cite{Hirsh2012} qui permet de réaliser une analyse statique d'un code grâce à l'ajout de marqueur dans le code source. Il permet, pou un noyau identifié, de détecter la présence de dépendance entre plusieurs itérations de boucle et de donner une estimation des performances (débit d'instructions, saturation des ports de l'ALU...). Cepedant, il est nécessaire d'avoir identifié les zones de \textit{hot spots} pour les annoter, et il ne fonctionne que pour des architectures Intel. Le projet a été abandonné en avril 2019\footnote{Intel IACA - \url{https://software.intel.com/en-us/articles/intel-architecture-code-analyzer}}

--> on préfère dynamique à cause de la complexité des architectures qui rend défificile l'estimatique en statique
