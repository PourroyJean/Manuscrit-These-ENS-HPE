\section{Conlusion}\label{sec:conclusion-hpc}





\textbf{TODO}
 
 - L'interconnexion de toutes ces ressources est réalisée dans le seul but de réduire le temps nécessaire à la résolution d'un problème. En effet, pour une expérience donnée si l'on possède un cluster de 1000 machines on ne voudra pas réaliser l'expérience un millier de fois mais plutôt réduire le temps d'une expérience par un facteur 1000 pour ensuite analyser les résultats, changer les paramètres et pouvoir lancer une nouvelle expérimentation. La calcul parallèle est un ensemble de moyens, logiciel et matériel qui permettent de réaliser des instructions simultanément. L'idée principale du calcul parallèle est de réduire le temps de calcul d'un programme en divisant le travail à réaliser, le partager en sous-problèmes qui peuvent être résolu de façon indépendante par plusieurs ressources de calcul, comme des processeurs. Un exemple concret de résolution d'un problème grâce à la programmation parallèle est donnée dans la section \ref{sub_reso_partage}

